\setcounter{section}{-1}

\section{Preliminaries}

\subsection{Basics}

\begin{defi}[Basic Objects]
    \label{defi0.1}
    \begin{itemize}
        \item The \textbf{order} or \textbf{cardinality} of a set $A$ will be denoted with $|A|$.
        \item The \textbf{Cartesian product} of two sets $A$ and $B$ is the collection of elements
        \[A \times B := \{(a, b) \mid a \in A, b \in B\}\]
        \item The following are common sets:
        \begin{itemize}
            \item $\z$ is the set of integers $\set{\dots, -2, -1, 0, 1, 2, \ldots}$.
            \item $\q$ is the set of rationals $\set{p/q \mid p \in \z, q \in \z\nz}$.
            \item $\r$ is the set of real numbers whose decimal expansions cannot be written as a rational.
            \item $\c$ is the set of complex numbers $\set{a + bi \mid a, b \in \r, i^2 = -1}$.
        \end{itemize}
        Moreover, we denote $\zp, \q^+$, and $\r^+$ to be the positive sets of $\z, \q,$ and $\r$ respectively.
    \end{itemize}
\end{defi}

\begin{defi}[Function Terminology]
    \label{defi0.2}
    \begin{itemize}
        \item A \textbf{function} $f$ from $A$ to $B$ is denoted as $f : A \to B$, where the value of $f$ at some $a \in A$ is denoted by $f(a)$.
        \item The \textbf{domain} of $f$ is $A$, and the \textbf{codomain} of $f$ is $B$. Moreover, we use the notation $f : a \to b$ or $a \mapsto b$ to denote that $f(a) = b$, or that the function is specified on elements.
        \item The \textbf{range} or \textbf{image} of $f$ (also known as the \textbf{image of $A$ under $f$} is given by
        \[f(A) := \{b \in B \mid f(a) = b \text{ for some } a \in A\}\]
        \item The \textbf{preimage} or \textbf{inverse image} of some subset $C$ of $B$ is given by
        \[f\inv(C) := \{a \in A \mid f(a) \in C\}\]
        Moreover, the \textbf{fiber of $f$ over $b$} for some $b \in B$ is the preimage of $b$ wherein we may replace $C$ with $b$ above.
        \item Let $f : A \to B$ and $g : B \to C$ be functions. Then the \textbf{composite map} $g \circ f : A \to C$ is given by
        \[(g \circ f)(a) := g(f(a))\]
        \item $f$ is \textbf{injective} or an \textbf{injection} whenever $a_1 \neq a_2$ implies $f(a_1) \neq f(a_2)$.
        \item $f$ is \textbf{surjective} or a \textbf{surjection} if for every $b \in B$ there exists $a \in A$ such that $f(a) = b$, i.e., $f(A) = B$.
        \item $f$ is \textbf{bijective} or a \textbf{bijection} when it is both injective and surjective.
        \item A \textbf{left inverse} of $f$ is a function $g : B \to A$ such that $g \circ f : A \to A$ is the identity map on $A$, i.e., $(g \circ f)(a) = a$ for every $a \in A$.
        \item A \textbf{right inverse} of $f$ is a function $h: B \to A$ such that $f \circ h : B \to B$ is the identity map on $B$.
    \end{itemize}
\end{defi}

\begin{prop}[Implication of Function Qualities]
    \label{prop0.1}
    Let $f: A \to B$.
    \begin{enumerate}
        \item $f$ is injective if and only if $f$ has a left inverse.
        \item $f$ is surjective if and only if $f$ has a right inverse.
        \item $f$ is a bijection if and only if there exists $g: B \to A$ such that $f \circ g$ is the identity map on $B$ and $g \circ f$ the identity map on $A$.
        \item If $A$ and $B$ are finite sets where $|A| = |B|$, then $f: A \to B$ is bijective if and only if $f$ is injective if and only if $f$ is surjective.
    \end{enumerate}
    \begin{proofnum}
        \item \rightimp We first note that $A$ must be nonempty, for otherwise if $A = \varnothing$, then $f : \varnothing \to B$ is vacuously injective. However, no left inverse $g : B \to \varnothing$ of $f$ may exist, since there is no element for $b \in B$ to be mapped to in $\varnothing$ as it contains no elements, by definition. Now suppose $f$ is injective. To construct a left inverse, note that $B$ can be split up into two subsets: $f(A)$ and its complement, $B \bl f(A)$. If $b \in f(A)$, by definition, there is some $a \in A$ where $f(a) = b$. If $b \not\in f(A)$, we may simply choose some element in $A$ to map to. Explicitly, we define a function $g : B \to A$ as follows, fixing some $a^*$
        \[g(b) = 
        \begin{cases}
            a & a \in A \text{ such that } f(a) = b \\
            a^* & b \in B \bl f(A)
        \end{cases}\]
        This is clearly a left inverse, because for any $a \in A$, we have $f(a) = b$ for some $b \in B$ so that $g(f(a)) = g(b) = a$.
        
        \leftimp Suppose now that $f$ has a left inverse $g$, and let $x, y \in A$ such that $f(x) = f(y)$. Then $g(f(x)) = g(f(y))$. Since $g$ is a left inverse, then $x = y$.
        \item \rightimp Suppose $f$ is surjective. Then each fiber of $f$ over every $b \in B$ is nonempty. We may then arbitrarily associate each $b$ with a singular $a_b \in A$ such that $f(a_b) = b$. We may then define a mapping $h : B \to A$ as follows: for any $b \in B$, then $h(b) = a_b$. Then $f(h(b)) = f(a_b) = b$ so that $h$ is a right inverse of $f$.
        
        \noindent \leftimp Suppose that $f$ has a right inverse $h$, and let $b \in B$. Then for any $h(b) \in h(B)$, we have $f(h(b)) = b$ so that $f$ is surjective.
        \item \rightimp Suppose $f$ is bijective, and let $b \in B$. Since $f$ is surjective, there exists some $a \in A$ such that $f(a) = b$. Moreover, this $a$ is unique because $f$ is injective. Define the function $g : B \to A$ by $g(b) = a$. Then for any $a \in A$, we have $g(f(a)) = g(b) = a$. For any $b \in B$, we have $f(g(b)) = f(a) = b$ so that $g \circ f$ is the identity map on $A$, and $f \circ g$ is the identity map on $B$.
        
        \noindent \leftimp Suppose there exists $g : B \to A$ such that $f \circ g$ is the identity map on $B$, and $g\circ f$ is the identity map on $A$. Then $g$ is a left inverse and a right inverse of $f$, hence $f$ is injective and surjective. Then $f$ is bijective, by definition.
        \item Suppose $f$ is bijective. It is then injective.
        
        Suppose $f$ is injective. Then distinct elements of $A$ map to distinct elements of $B$. Since $|A| = |B|$, then all elements of $B$ are mapped to, hence $f$ is surjective.
        
        Suppose $f$ is surjective. Then each element of $B$ has at least one element of $A$ that maps to it. Since $f$ is well-defined, then no element of $A$ maps to more than one element in $B$. Because $|A| = |B|$, then each element in $A$ maps to a distinct element in $B$, hence $f$ is injective. Then it is bijective, by definition. 
    \end{proofnum}
\end{prop}

\begin{multicols}{2}
    \begin{defi}[Types of Functions]
        \begin{itemize}
            \item In (3) of \autoref{prop0.1}, the map $g$ is unique and is called the \textbf{2-sided inverse} of $f$.
            \item A \textbf{permutation} of a set $A$ is a bijection from $A$ to itself.
            \item Let $A \subseteq B$ with $f: B \to C$. Then the \textbf{restriction} of $f$ to $A$ is denoted by $f|_A$.
            \item Let $A \subseteq B$ with $g: A \to C$. Let $f : B \to C$ such that $f|_A = g$. Then $f$ is an \textbf{extension} of $g$ to $B$, where $f$ does not need to exist nor be unique.
        \end{itemize}
    \end{defi}
    
    \begin{defi}[Relations]
        \label{0.6}
        Let $A$ be a nonempty set.
        \begin{itemize}
            \item A \textbf{binary relation} on a set $A$ is a subset $R$ of $A \times A$, and we write $a \sim b$ when $(a, b) \in R$.
            \item We say the relation $\sim$ on $A$ is:
            \begin{enumerate}
                \item \textbf{reflexive} if $a \sim a$ for every $a \in A$,
                \item \textbf{symmetric} if $a \sim b$ implies $b \sim a$ for every $a, b \in A$,
                \item \textbf{transitive} if $a \sim b$ and $b \sim c$ imply $a \sim c$ for all $a, b, c \in A$.
            \end{enumerate}
            \item A relation is an \textbf{equivalence relation} when it is reflexive, symmetric, and transitive.
            \item Let $\sim$ be an equivalence relation on $A$. Then the \textbf{equivalence class} of $a \in A$ is the set $\{x \in A \mid x \sim a\}$. Moreover, the elements of this set are said to be \textbf{equivalent} to $a$. If $B$ is an equivalence class, then any $b \in B$ is a \textbf{representative} of $B$.
            \item A \textbf{partition} of $A$ is some collection $\{A_i \mid i \in I\}$ of nonempty subsets, for some indexing set $I$, such that
            \begin{enumerate}
                \item $A = \bigcup_{i \in I} A_i$,
                \item $A_i \cap A_j = \varnothing$ for every $i, j \in I$ where $i \neq j$.
            \end{enumerate}
        \end{itemize}
    \end{defi}
\end{multicols}

\begin{prop}[Fundamental Theorem of Equivalence Relations]
    \label{prop0.2}
    Let $A$ be a nonempty set.
    \begin{enumerate}
        \item If $\sim$ is an equivalence relation on $A$, then the set of equivalence classes of $\sim$ form a partition of $A$.
        \item If $\{A_i \mid i \in I\}$ is a partition of $A$, then there is an equivalence class relation on $A$ whose equivalence classes are the sets $A_i$ for $i \in I$.
    \end{enumerate}
    \begin{proofnum}
        \item Let $I$ be an indexing set, put $B = \bigcup_{i \in I}A_i$, and let $\set{A_i \mid i \in I}$ be the set of equivalence classes of $\sim$. Since $A_i \subseteq A$ for each $i \in I$, then $B \subseteq A$. Pick some $a \in A$. Since $\sim$ is reflexive, then $a \sim a$ so that $a \in A_j$ for some $j \in I$. Then $a \in B$ so that $A \subseteq B$, hence $A = B$. To show that these equivalence classes are pairwise disjoint, it suffices to show that they if they are not, then they must be equal. To that end, suppose $A_i \cap A_j$ is not empty for some $i, j \in I$, let $a$ and $b$ be representatives of $A_i$ and $A_j$ respectively, and let $c \in A_i \cap A_j$. By definition, $a \sim c$ and $c \sim b$. By transitivity of $\sim$, we have $a \sim b$. Suppose some $x \in A_i$. Then $x \sim a$ so that $x \sim b$, and $x \in A_j$. Similarly, $y \in A_j$ implies $b \sim y$ so that $a \sim y$, and $y \in A_i$. Then $A_i = A_j$ so that that any two equivalence classes are either distinct or equal.
        \item Define a relation on $A$ as follows: $a \sim b$ if and only if $a, b \in A_i$ for some $i \in I$. Clearly, $a \sim a$ so that $\sim$ is reflexive. If $a \sim b$, then $a, b\in A_i$ so that $b, a \in A_i$. Hence, $b \sim a$, and $\sim$ is symmetric. Moreover, if $a \sim b$ and $b \sim c$, then $a, b \in A_i$ and $b, c \in A_j$. Since $A_i \cap A_j$ is nonempty, then $A_i = A_j$, hence $a, c \in A_i$ so that $a \sim c$. Then $\sim$ is transitive, hence it is an equivalence relation.
    \end{proofnum}
\end{prop}

\subsection{Properties of the Integers}

\begin{defi}[Integer Properties]
    \begin{itemize}
        \item \textbf{Well Ordering of $\z$}: If $A$ is a nonempty subset of $\zp$, then there exists $m \in A$ such that $m \leq a$ for all $a \in A$. We say $m$ is the \textbf{minimal element} of $A$.
        \item For $a, b \in \z$ with $a \neq 0$, then $a$ \textbf{divides} $b$ if there is some $c \in \z$ such that $ac = b$. We write $a \mid b$. Otherwise, if $a$ does not divide $b$, then we write $a \nmid b$.
        \item If $a, b \in \z\nz$, there exists a unique positive integer $d$ called the \textbf{greatest common divisor} of $a$ and $b$ where
        \begin{enumerate}
            \item $d \mid a$ and $d \mid b$ so that $d$ is a common divisor, 
            \item If $e$ is another element such that $e \mid a$ and $e \mid b$, then $e \mid d$ so that $d$ is the \textit{greatest} common divisor.
        \end{enumerate}
        We denote $d$ as $(a, b)$. Moreover. if $(a, b) = 1$, we say $a$ and $b$ are \textbf{relatively prime}.
        \item If $a, b \in \z\nz$, there exists a unique positive integer $\ell$ called the \textbf{least common multiple} of $a$ and $b$ where
        \begin{enumerate}
            \item $a \mid \ell$ and $b \mid \ell$ so that $\ell$ is a common multiple,
            \item If $k$ is another positive integer such that $a \mid k$ and $b \mid k$, then $\ell \mid k$ so that $\ell$ is the \textit{least} common multiple.
        \end{enumerate}
        Moreover, for any two elements $a, b$ with a greatest common divisor $d$ and least common multiple $\ell$, then $ab = d\ell$. 
    \end{itemize}
\end{defi}

\begin{defi}[Division and Euclidean Algorithms]
    \label{defi0.6}
    \begin{itemize}
        \item \textbf{Division Algorithm}: For $a, b \in \z\nz$, there exists unique $q, r \in \z$ such that
        $$a = bq + r, 0 \leq r < |b|$$
        where $q$ is the \textbf{quotient} and $r$ is the \textbf{remainder}.
        \item \textbf{Euclidean Algorithm}: a procedure that produces a greatest common divisor between two integers $a$ and $b$ done by iterating the Division Algorithm. For $a, b \in \z\nz$, we get a sequence of quotients and remainders:
        \begin{align*}
            a & = q_0b + r_0 \\
            b & = q_1r_0 + r_1 \\
            r_0 & = q_2r_1 + r_2 \\
            r_1 & = q_3r_2 + r_3 \\
            \vdots \\
            r_{n - 2} & = q_nr_{n - 1} + r_n \\
            r_{n - 1} & = q_{n + 1}r_n
        \end{align*}
        Note that the sequence of remainders $r_i$ terminates eventually, since they are a decreasing sequence of nonnegative integers. We then have that $(a, b) = r_n$.
        \item A consequence of the Euclidean Algorithm is the \textbf{$\z$-linear combination of $a$ and $b$}. For some $a, b \in \z\nz$, there exists $x, y \in \z$ such that
        $$(a, b) = ax + by$$
        We can obtain this by recursively solving for $r_n$ in the equations above until we get $r_n$ in terms of $a$ and $b$ again.
    \end{itemize}
\end{defi}

\begin{ex}[Applying the Euclidean Algorithm and Finding the $\z$-Linear Combination]
    Let $a = 57970$ and $b = 10353$. Applying the Euclidean Algorithm, we obtain
    \begin{align*}
        57970 & = (5)10353 + 6205 \\
        10353 & = (1)6205 + 4148 \\
        6205 & = (1)4148 + 2057 \\
        4148 & = (2)2057 + 34 \\
        2057 & = (60)34 + 17 \\
        34 & = (2)17
    \end{align*}
    so that $(57950, 10353) = 17$. We may then find the $\z$-linear combination of $a$ and $b$ as follows:
    \begin{align*}
        17 & = 2057 - (60)34 \\
        & = 2057 - 60(4148 - 2(2057)) = -60(4148) + 121(2057) \\
        & = -60(4148) + 121(6205 - 4148) = 121(6205) - 181(4148) \\
        & = 121(6205) - 181(10353 - 6205) = -181(10353) + 302(6205) \\
        & = -181((10353) + 302(57970 - 5(10353)) = 302(57970) - 1691(10353)
    \end{align*}
    so that $x = 302$ and $y = -1691$. Note that $x$ and $y$ are not unique.
\end{ex}

\begin{defi}[Primality]
    \label{defi0.7}
    We say $p \in \n$ is \textbf{prime} when $p > 1$ and the only positive divisors of $p$ are $p$ and 1 itself. An integer that is not prime is \textbf{composite}. Moreover, if $p$ is prime and $p \mid ab$ for $a, b \in \z$, then $p \mid a$ or $p \mid b$.
    \begin{proof}
        Let $p$ be some prime, and let $a, b \in \z$ such that $p \mid ab$. If $p \mid a$, we are done. Suppose now that $p \nmid a$. There must exist some $k \in \z$ such that $kp = ab$. Since $(p, a) = 1$, then there exist integers $x, y \in \z$ such that 
        \begin{align*}
            px + ay & = 1 \\
            \intertext{Multiplying both sides by $b$, we obtain:}
            bpx + aby & = b \\
            \intertext{Recall that $kp = ab$. Then}
            bpx + kpy & = b \\
            p(bx + ky) & = b
        \end{align*}
        hence $p \mid b$.
    \end{proof}
\end{defi}

\begin{defi}[Fundamental Theorem of Arithmetic]
    \label{defi0.8}
    Any $n \in \zp$ can be uniquely factored into a product of primes $p_1, p_2, \ldots, p_s$ with $\alpha_1, \alpha_2, \ldots, \alpha_s \in \zp$ such that:
    \[n = \prod_{i = 1}^s p_i^{\alpha_i}\]
    \begin{proof}
        We proceed by induction. Since 1 and 2 are divisible by 1 and itself only, the base cases are covered. Let $n \in \zp$ be given, and suppose any integer below $n$ can be factored into a product of primes. Then for $n$ itself, we have two cases: if $n$ is prime, we are done. Suppose it is not prime. Then $n$ is composite, and there are $a, b \in \zp$ such that $n = ab$, and $a < n$ and $b < n$. By the inductive hypothesis, both $a$ and $b$ can be decomposed into a product of primes, hence $n$ is a product of primes.
        
        To prove uniqueness, let $n$ be a minimal integer with the decomposition above, and suppose there existed another decomposition with primes $q_1, q_2, \ldots, q_t$ and exponents $\beta_1, \beta_2, \ldots, \beta_t$ such that
        \[n = \prod_{j = 1}^t q_j^{\beta_j}\]
        Using properties of primes, we see that $p_1$ divides the product of all $q_j$. Without loss of generality, we may assume that $p_1 \mid q_1$. Since $p_1$ and $q_1$ are both prime, then $p_1 = q_1$. Dividing both products by $p_1$, we obtain the equality
        \[\prod_{i = 2}^sp_i^{\alpha_i} = \prod_{j = 2}^tq_j^{\beta_j}\]
        which are strictly smaller than $n$, contradicting its minimality. Hence, $n$ has a unique prime factorization.
    \end{proof}
    We may also express the greatest common divisor and least common multiple using prime factorization. Let $a, b \in \zp$ be represented by the following products:
    \[a = \prod_{i = 1}^s p_i^{\alpha_i}, \quad b = \prod_{i = 1}^s p_i^{\beta_i}\]
    so that $a$ and $b$ utilize the same primes but may have dif and only ifering exponents for primes who are not present in their respective factorizations. Letting $[ \cdot, \cdot]$ denote the least common multiple of two integers, we have the following:
    \[(a, b) = \prod_{i = 1}^s p_i^{\min(\alpha_i, \beta_i)}, \quad [a, b] = \prod_{i = 1}^s p_i^{\max(\alpha_i, \beta_i)}\]
\end{defi}

\begin{defi}[Euler $\phi$-Function]
    \label{defi0.9}
    For $n \in \n$, let $\phi(n)$ be the number of positive integers $a \leq n$ with $(a, n) = 1$. For primes $p$ and powers $a \geq 1$, we have the formula
    $$\phi(p^a) = p^{a - 1}(p - 1)$$
    $\phi$ is \textit{multiplicative} in the sense that
    $$\phi(ab) = \phi(a)\phi(b) \text{ when } (a, b) = 1$$
    We can then obtain a general formula applying $\phi$ to some $n = p_1^{\alpha_1}p_2^{\alpha_2}\cdots p_s^{\alpha_s}$ as such:
    \begin{align*}
        \phi(n) & = \prod_{i = 1}^s \phi(p_i^{\alpha_i}) \\
        & = \prod_{i = 1}^s p_i^{\alpha_i - 1}(p_i - 1)
    \end{align*}
\end{defi}

\subsection{\texorpdfstring{$\intmod$: The Integers Modulo $n$}{Z/nZ: The Integers Modulo n}}

\begin{defi}[Modular Congruence and Integers Modulo $n$]
    Let $n \in \z$. Define a relation $\sim$ on $\z$ by
    $$a \sim b \iff n \mid (b - a)$$
    Clearly $a \sim a$ since $n \mid 0$. Moreover, if $a \sim b$, then $nk = b - a \implies -nk = a - b$ so that $n \mid (a - b)$, and $b \sim a$. Moreover, if $a \sim b$ and $b \sim c$, then $nk = b - a$ and $n\ell = c - b$ for some $k, \ell \in \z$. Add those equations to get $n(k + \ell) = c - a$, that $a \sim c$. We conclude that $\sim$ is an equivalence relation, and we write $a \equiv b \bmod n$. For some $k \in \z$, let the equivalence class of $a$ be denoted as $\widebar a$, called the \textbf{congruence class} or \textbf{residue class of $a \bmod n$}. This class consists of the integers dif and only ifering from $a$ by an integer multiple of $n$, or written out:
    $$\widebar a = \{a + kn : k \in \z\}$$
    The set of equivalence classes under a particular $n$ is called the \textbf{integers modulo $n$}, which partition $\z$ into $n$ classes: $\widebar 0, \widebar 1, \ldots, \widebar{n - 1}$. Moreover, we can define both \textbf{modular arithmetic}, or operations between any two elements in $\intmod$ as follows: for any $\widebar a, \widebar b \in \intmod$, their sum and products are defined as:
    $$\widebar a + \widebar b := \widebar{a + b} \text{ and } \widebar a \cdot \widebar b := \widebar{ab}$$
\end{defi}

\begin{theo}[Modular Arithmetic is Well-Defined]
    \label{theo0.3}
    The operations of addition and multiplication on $\intmod$ are well-defined. More precisely, if $s_1, s_2 \in \z$ and $t_1, t_2 \in \z$ where $\widebar{s_1} = \widebar{t_1}$ and $\widebar{s_2} = \widebar{t_2}$, then $\widebar{s_1 + s_2} = \widebar{t_1 + t_2}$ and $\widebar{s_1s_2} = \widebar{t_1t_2}$.
    \begin{proof}
        Since $\widebar{s_1} = \widebar{t_1}$ and $\widebar{s_2} = \widebar{t_2}$, it follows that $s_1 = t_1 + nx$ and $s_2 = t_2 + ny$ for $x, y \in \z$. Then $s_1 + s_2 = t_1 + t_2 + n(x + y)$. Moreover, we have $s_1s_2 = (t_1 + nx)(t_2 + ny) = t_1t_2 + (t_1y + t_2x + nxy)n$.
    \end{proof}
\end{theo}

\begin{ex}[Calculating Last Digits]
    Suppose we wanted to calculate the last two digits of $2^{1000}$. The last two digits of any number is the remainder when dividing by $100$, so the goal is to reduce $2^{1000}$ to $\bmod 100$. There are many ways to calculate this, but one such way is when we consider $2^{10} = 1024 \equiv 24 \bmod 100$. Then $2^{20} \equiv 24^2 \bmod 100 = 576 \bmod 100 \equiv 76 \bmod 100$. Then $2^{40} \equiv 76^2 \bmod 100 = 5776 \bmod 100 = 76 \bmod 100$. It looks like multiplying by $2^{20}$ results in $76 \bmod 100$, and since $1000$ is divisible by $20$, it follows that $2^{1000} \equiv 76 \bmod 100$. 
\end{ex}

\begin{defi}[Primitive Residue Classes Modulo $n$]
    Let $\intmod$ be defined as before. The the \textbf{primitive residue classes modulo $n$} is the subset of $\intmod$ whose elements contain a multiplicative inverse:
    $$\units = \{\widebar a \in \intmod : \exists \widebar c \in \intmod, \widebar a \cdot \widebar c = \widebar 1\}$$
\end{defi}

\begin{prop}[Defining $\units$]
    \label{prop0.4}
    The set $\units \subset \intmod$ is \textit{precisely} the set of residue classes whose representatives are relatively prime to $n$, i.e.,
    $$\units = \{\widebar a \in \intmod : (a, n) = 1\}$$
\end{prop}

\begin{ex}[Example of $\units$ and Calculating Inverses]
    \begin{itemize}
        \item For $n = 9$, we have all elements $x$ such that $(x, 9) = 1$. It is precisely the set $\{\widebar 1, \widebar 2, \widebar 4, \widebar 5, \widebar 7, \widebar 8\}$. Moreover, the inverses of each of those elements is $\widebar 1, \widebar 5, \widebar 7, \widebar 2, \widebar 4, \widebar 8$ respectively. 
        \item Note that $a$ and $n$ being relatively prime allows the Euclidean Algorithm to produce $x, y \in \z$ such that $ax + ny = 1$, or $ax \equiv 1 \bmod n$. It follows that $\widebar x$ would be the multiplicative inverse of $\widebar a$ in $\intmod$.
        \item Consider $n = 60$ and $a  =17$. Using the Euclidean Algorithm, we obtain
        \begin{align*}
            60 &= 3(17) + 9 \\
            17 &= 1(9) + 8 \\
            9 &= 1(8) + 1
        \end{align*}
        so that $(60, 17) = 1$, and $-7(17) + 2(60) = 1$. Then $\widebar{-7} = \widebar{53}$ is the multiplicative inverse of $\widebar{17}$ in $\intmod[60]$.
    \end{itemize}
\end{ex}