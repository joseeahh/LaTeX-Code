\section{Introduction to Groups}

\subsection{Basic Axioms and Examples}

Let $G$ be a group.

\begin{exercise}
    Determine which of the following binary operations are associative:
    \begin{subproblems}
        \item the operation $\star$ on $\z$ defined by $a \star b = a - b$
        \item the operation $\star$ on $\r$ defined by $a \star b = a + b + ab$
        \item the operation $\star$ on $\q$ defined by $a \star b = \dfrac{a + b}{5}$
        \item the operation $\star$ on $\z \times \z$ defined by 
        \[
        (a,b) \star (c,d) = (ad + bc, \, bd)
        \]
        \item the operation $\star$ on $\q - \{0\}$ defined by $a \star b = \dfrac{a}{b}$
    \end{subproblems}
\end{exercise}

\begin{solalph}
    \item Not associative: $(1 \star 0) \star 1 = 0 \neq 2 = 1 \star (0 \star 1)$.
    \item Associative:
    \begin{align*}
        (a \star b) \star c & = (a + b + ab) \star c \\
        & = (a + b + ab) + c + (a + b + ab)c \\
        & = a + b + ab + c + ac + bc + abc \\
        & = a + b + c + bc + ab + ac + abc \\
        & = a + (b + c + bc) + a(b + c + bc) \\
        & = a \star (b + c + bc) \\
        & = a \star (b \star c)
    \end{align*}
    \item Not associative: $(1 \star 0) \star 2 = 11/25 \neq 7/25 = 1 \star (0 \star 2)$.
    \item Associative:
    \begin{align*}
        [(a, b) \star (c, d)] \star (e, f) & = (ad + bc, bd) \star (e, f) \\
        & = ((ad + bc)f + bde, bdf) \\
        & = (adf + bcf + bde, bdf) \\
        & = (adf + b(cf + de), bdf) \\
        & = (a, b) \star (cf + de, df) \\
        & = (a, b) \star [(c, d) \star (e, f)]
    \end{align*}
    \item Not associative: $(1 \star 2) \star 3 = 1/6 \neq 3/2 = 1 \star (2 \star 3)$.
\end{solalph}

\newpage

\begin{exercise}
    Decide which of the binary operations in the preceding exercise are commutative.
    \begin{subproblems}
        \item the operation $\star$ on $\z$ defined by $a \star b = a - b$
        \item the operation $\star$ on $\r$ defined by $a \star b = a + b + ab$
        \item the operation $\star$ on $\q$ defined by $a \star b = \dfrac{a + b}{5}$
        \item the operation $\star$ on $\z \times \z$ defined by $(a,b) \star (c,d) = (ad + bc, \, bd)$
        \item the operation $\star$ on $\q - \{0\}$ defined by $a \star b = \dfrac{a}{b}$
    \end{subproblems}
\end{exercise}

\begin{solalph}
    \item Not commutative: $1 - 0 \neq 0 - 1$.
    \item Commutative: $a \star b = a + b + ab = b + a + ba = b \star a$.
    \item Commutative: $a \star b = (a + b)/5 = (b + a)/5 = b \star a$.
    \item Commutative: $(a, b) \star (c, d) = (ad + bc, bd) = (cb + da, db) = (c, d) \star (a, b)$.
    \item Not commutative: $1/2 \neq 2/1$.
\end{solalph}

\begin{exercise}
    Prove that addition of residue classes in $\intmod$ is associative (you may assume it is well defined).
\end{exercise}

\begin{sol}
    Let $\bar a, \bar b, \bar c \in \intmod$. Then
    \begin{align*}
        (\bar a + \bar b) + \bar c & = \bar{a + b} + \bar c \\
        & = \bar{(a + b) + c} \\
        & = \bar{a + (b + c)} \\
        & = \bar a + \bar{b + c} \\
        & = \bar a + (\bar b + \bar c) \qh
    \end{align*}
\end{sol}

\begin{exercise}
    Prove that multiplication of residue classes in $\intmod$ is associative (you may assume it is well defined).
\end{exercise}

\begin{sol}
    Since $\cdot$ is associative over $\z$, we may use the same argument as in the previous exercise.
\end{sol}

\begin{exercise}
    Prove for all $n > 1$ that $\intmod$ is not a group under multiplication of residue classes.
\end{exercise}

\begin{sol}
    Note that $\bar 0 \in \intmod$, but there is no $\bar a$ such that $\bar 0 \cdot \bar a = \bar 1$. Hence, $\intmod$ does not have an identity under multiplication, so it is not a group.
\end{sol}

\newpage

\begin{exercise}
    Determine which of the following sets are groups under addition:
\begin{subproblems}
    \item the set of rational numbers (including $0 = 0/1$) in lowest terms whose denominators are odd
    \item the set of rational numbers (including $0 = 0/1$) in lowest terms whose denominators are even
    \item the set of rational numbers of absolute value $< 1$
    \item the set of rational numbers of absolute value $\ge 1$ together with $0$
    \item the set of rational numbers with denominators equal to $1$ or $2$
    \item the set of rational numbers with denominators equal to $1$, $2$, or $3$
\end{subproblems}
\end{exercise}

\begin{sol}
    Let $S$ denote the set in each part.
    \begin{subproblems}
        \item Clearly, $0 \in S$. Moreover, for any $s \in S$, then $-s \in S$. Since $\q$ is associative under addition, then so is $S$. To show closure, let $a/b, c/d \in S$, where $b$ and $d$ are odd, and $(a, b) = (c, d) = 1$. Then
        \[\frac ab + \frac cd = \frac{ad + bc}{bd}\]
        where $bd$ is odd. Note that a number is even when at least one of its factors is even. It follows that any common factor of $ad + bc$ and $bd$ must also be odd, hence the fraction in its lowest terms will still have an odd denominator. Thus, $S$ is closed under addition and is a group.
        \item $S$ is not a group, since $1/2 \in S$, but $1/2 + 1/2 = 2/2 = 1/1 \not\in S$.
        \item Same as previous.
        \item $S$ is not a group, since $3/2, 1 \in S$, but $3/2-1 = 1/2 \not\in S$.
        \item Since $0 = 0/1$, then $0 \in S$. Moreover, for any $s \in S$, then $-s \in S$. Since $\q$ is associative under addition, then so is $S$. To show closure, we first note that $s \in S$ has the form
        \[s = a \quad \text{or} \quad s = a + \frac 12\]
        for some $a \in \z$. If any two $s_1, s_2 \in S$ have the same form, then $s_1 + s_2 \in S$. Otherwise, suppose without loss of generality that $s_1$ has the first form and $s_2$ has the second form. Then
        \[s_1 + s_2 = a + \left(b + \frac 12\right) = (a + b) + \frac 12 \in S\]
        so that $S$ is closed under addition and is a group.
        \item $S$ is not a group, since $1/2, 1/3 \in S$, but $1/2 + 1/3 = 5/6 \not\in S$. \qh
    \end{subproblems}
\end{sol}

\newpage

\begin{exercise}
    Let $G = \{ x \in \r \mid 0 \le x < 1 \}$ and for $x, y \in G$ let $x \star y$ be the fractional part of $x + y$ (i.e., $x \star y = x + y - [x + y]$ where $[a]$ is the greatest integer less than or equal to $a$). Prove that $\star$ is a well defined binary operation on $G$ and that $G$ is an abelian group under $\star$ (called the \textit{real numbers mod $I$}).
\end{exercise}

\begin{sol}
    Suppose $x, y \in G$. To show closure, there are two cases:
    \begin{itemize}
        \item Suppose $0 \leq x + y < 1$. Then $[x + y] = 0$ so that $x \star y = x + y - 0 = x + y \in G$.
        \item Suppose $1 \leq x + y < 2$. Then $[x + y] = 1$ so that $x \star y = x + y - 1 \in G$.
    \end{itemize}
    To show associativity, let $x, y, z \in G$. Recall that $[a \pm b] = [a] \pm b$ for all $a \in \r$ and $b \in \z$. Then
    \begin{align*}
        (x \star y) \star z & = (x + y - [x + y]) \star z \\
        & = (x + y - [x + y]) + z - [(x + y - [x + y]) + z] \\
        & = x + y + z - [x + y] - [x + y + z - [x + y]] \\
        & = x + y + z - [x + y] - [x + y + z] + [x + y] \\
        & = x + y + z - [x + y + z] \\
        & = x + y + z - [y + z] - [x + y + z] + [y + z] \\
        & = x + y + z - [y + z] - [x + y + z - [y + z]] \\
        & = x + (y + z - [y + z]) - [x + (y + z - [y + z])] \\
        & = x \star (y + z - [y + z]) \\
        & = x \star (y \star z)
    \end{align*}
    Moreover, $0 \in G$ is the identity since for any $x \in G$, then $x \star 0 = x + 0 - [x + 0] = x$. Additionally, for any $x \in G$, its inverse is $1 - x$ if $x \neq 0$ and $0$ if $x = 0$, since
    \[x \star (1 - x) = x + (1 - x) - [x + (1 - x)] = 1 - 1 = 0\]
    Finally, $\star$ is commutative since $x \star y = x + y - [x + y] = y + x - [y + x] = y \star x$. Hence, $G$ is abelian under $\star$.
\end{sol}

\begin{exercise}
    Let $G = \{z \in \c \mid z^n = 1 \text{ for some } n \in \zp \}$.
    \begin{subproblems}
        \item Prove that $G$ is a group under multiplication (called the group of roots of unity in $\c$).
        \item Prove that $G$ is not a group under addition.
    \end{subproblems}
\end{exercise}

\begin{solalph}
    \item Since $1^1 = 1$, then $1 \in G$. Moreover, for any $z \in G$ such that $z^n = 1$, then its inverse $z\inv$ satisfies $(z\inv)^n = (z^n)\inv = 1\inv = 1$ so that $z\inv \in G$. Since multiplication in $\c$ is associative, then so is $G$. Finally, let $z, w \in G$ such that $z^n = 1$ and $w^m = 1$ for some $n, m \in \zp$. Then
    \[(zw)^{nm} = z^{nm} w^{nm} = (z^n)^m (w^m)^n = 1 \cdot 1 = 1\]
    so that $zw \in G$. Hence, $G$ is a group under multiplication.
    \item Not a group, since $1 \in G$ and $-1 \in G$, but $1 + (-1) = 0 \not\in G$ since $0^n \neq 1$ for all $n \in \zp$.
\end{solalph}

\newpage

\begin{specialexercise}
    Let $G = \{ a + b\sqrt{2} \in \r \mid a, b \in \q \}$.
    \begin{subproblems}
        \item Prove that $G$ is a group under addition.
        \item Prove that the nonzero elements of $G$ are a group under multiplication. [``Rationalize the denominators'' to find multiplicative inverses.]
    \end{subproblems}
\end{specialexercise}

\begin{solalph}
    \item Since $0 + 0 \sqrt 2 \in G$, then $G$ has an identity. Moreover, for any $a + b\sqrt 2 \in G$, its inverse is $-a - b\sqrt 2 \in G$. Since addition in $\r$ is associative, then so is $G$. Finally, let $a + b\sqrt 2, c + d\sqrt 2 \in G$ for some $a, b, c, d \in \q$. Then
    \[(a + b\sqrt 2) + (c + d\sqrt 2) = (a + c) + (b + d)\sqrt 2 \in G\]
    since $a + c, b + d \in \q$. Hence, $G$ is a group under addition.
    \item Since $1 + 0\sqrt 2 \in G$, then $G$ has an identity. Moreover, for any nonzero $a + b\sqrt 2 \in G$, its inverse is
    \[(a + b\sqrt 2)\inv = \frac{1}{a + b\sqrt 2} = \frac{1}{a + b\sqrt 2} \cdot \frac{a - b\sqrt 2}{a - b\sqrt 2} = \frac{a - b\sqrt 2}{a^2 - 2b^2} = \frac a{a^2 - 2b^2} - \frac b{a^2 - 2b^2}\sqrt 2\]
    where $a^2 - 2b^2 \neq 0$ since $a + b\sqrt 2 \neq 0$. Since multiplication in $\r$ is associative, then so is $G$. Finally, let $a + b\sqrt 2, c + d\sqrt 2 \in G$ for some nonzero $a, b, c, d \in \q$. Then
    \[(a + b\sqrt 2)(c + d\sqrt 2) = (ac + 2bd) + (ad + bc)\sqrt 2 \in G\]
    since $ac + 2bd, ad + bc \in \q$. Hence, the nonzero elements of $G$ are a group under multiplication.
\end{solalph}

\begin{exercise}
    Prove that a finite group is abelian if and only if its group table is a symmetric matrix.
\end{exercise}

\begin{sol}
    Let $G$ be a finite group, and let $G = \{g_1, g_2, \ldots, g_n\}$ for some $n \in \zp$ with $g_1 = 1$. Note that $g_ig_j$ is the $(i, j)$-th entry of the group table, and a symmetric matrix implies that the $(i, j)$-th entry is equal to the $(j, i)$-th entry for all $1 \leq i, j \leq n$.

    We have that $G$ is abelian if and only if $g_ig_j = g_jg_i$ for all $1 \leq i, j \leq n$, which holds if and only if the $(i, j)$-th entry is equal to the $(j, i)$-th entry for all $1 \leq i, j \leq n$. This is true if and only if the group table is a symmetric matrix.
\end{sol}

\begin{exercise}
    Find the orders of each element of the additive group $\intmod[12]$.
\end{exercise}

\begin{sol}
    For each $\bar x \in \intmod[12]$, add it to itself until we arrive at $\bar 0$. $|\bar 0| = 1$, and $\bar 1 = 12$ since $12 \cdot \bar 1 = \bar{12} = \bar 0$. In particular, we have:
    $$\begin{array}{c|c|c|c|c|c|c|c|c|c|c|c|c}
        \bar x & \bar 0 & \bar 1 & \bar 2 & \bar 3 & \bar 4 & \bar 5 & \bar 6 & \bar 7 & \bar 8 & \bar 9 & \bar{10} & \bar{11} \\
        \hline
        |\bar x| & 1 & 12 & 6 & 4 & 3 & 12 & 2 & 12 & 3 & 4 & 6 & 12
    \end{array}$$
\end{sol}

\begin{exercise}
    Find the orders of the following elements of the multiplicative group $(\z/12\z)^{\times}$: $\bar{1}, \bar{-1}, \bar{5}, \bar{7}, \bar{-7}, \bar{13}$.
\end{exercise}

\begin{sol}
    $$\begin{array}{c|c|c|c|c|c|c}
        \bar x & \bar{1} & \bar{-1} & \bar{5} & \bar{7} & \bar{-7} & \bar{13} \\
        \hline
        |\bar x| & 1 & 2 & 2 & 2 & 2 & 1
    \end{array}$$
\end{sol}

\begin{exercise}
    Find the orders of the following elements of the additive group $\z/36\z$: $\bar{1}, \bar{2}, \bar{6}, \bar{9}, \bar{10}, \bar{12}, \bar{-1}, \bar{-10}, \bar{-18}$.
\end{exercise}

\begin{sol}
    $$\begin{array}{c|c|c|c|c|c|c|c|c|c}
            \bar x & \bar 1 & \bar 2 & \bar 6 & \bar 9 & \bar{10} & \bar{12} & \bar{-1} & \bar{-10} & \bar{-18} \\
            \hline
            |\bar x| & 36 & 18 & 6 & 4 & 18 & 3 & 36 & 18 & 2
    \end{array}$$
\end{sol}

\begin{exercise}
    Find the orders of the following elements of the multiplicative group $(\z/36\z)^{\times}$: $\bar{1}, \bar{-1}, \bar{5}, \bar{13}, \bar{-13}, \bar{17}$.
\end{exercise}

\begin{sol}
    $$\begin{array}{c|c|c|c|c|c|c}
        \bar x & \bar 1 & \bar{-1} & \bar 5 & \bar{13} & \bar{-13} & \bar{17} \\
        \hline
        |\bar x| & 1 & 2 & 6 & 3 & 6 & 2
    \end{array}$$
\end{sol}

\begin{exercise}
    Prove that $(a_1 a_2 \ldots a_n)^{-1} = a_n^{-1} a_{n-1}^{-1} \ldots a_1^{-1}$ for all $a_1, a_2, \ldots, a_n \in G$.
\end{exercise}

\begin{sol}
    We proceed by induction. For $n = 1$, the result is clear. Suppose it is true for some $n \in \zp$. Then for $n + 1$, we have
    \begin{align*}
        (a_1 a_2 \ldots a_n a_{n + 1})^{-1} & = ((a_1 a_2 \ldots a_n) a_{n + 1})^{-1} \\
        & = a_{n + 1}^{-1} (a_1 a_2 \ldots a_n)^{-1} \\
        & = a_{n + 1}^{-1} (a_n^{-1} a_{n - 1}^{-1} \ldots a_1^{-1}) \\
        & = a_{n + 1}^{-1} a_n^{-1} a_{n - 1}^{-1} \ldots a_1^{-1}
    \end{align*}
    The result follows by induction.
\end{sol}

\begin{exercise}
    Let $x$ be an element of $G$. Prove that $x^2 = 1$ if and only if $|x|$ is either $1$ or $2$.
\end{exercise}

\begin{solitem}
    \item[\rightimp] Suppose $x^2 = 1$. Then $|x| \leq 2$. Since $|x| \in \zp$, then $|x|$ is either $1$ or $2$.
    \item[\leftimp]Suppose $|x|$ is either $1$ or $2$. If $|x| = 1$, then $x = 1$ so that $x^2 = 1$. If $|x| = 2$, then $x^2 = 1$.
\end{solitem}


\begin{exercise}
    Let $x$ be an element of $G$. Prove that if $|x| = n$ for some positive integer $n$, then $x^{-1} = x^{n-1}$.
\end{exercise}

\begin{sol}
    Note that $x^n = x^{n - 1}x = 1$. Then $x^{n - 1} = x\inv$.
\end{sol}

\newpage

\begin{exercise}
    Let $x, y$ be elements of $G$. Prove that $xy = yx$ if and only if $y^{-1}xy = x$ if and only if $x^{-1}y^{-1}xy = 1$.
\end{exercise}

\begin{sol}
    We prove $xy = yx$ if and only if $y\inv xy = x$ first.
    \begin{ifandonlyif}
        \item [\rightimp] Suppose $xy = yx$. Then multiplying both sides on the left by $y\inv$ gives $y\inv xy = y\inv yx = 1x = x$.
        \item [\leftimp] Suppose $y\inv xy = x$. Then multiplying both sides on the left by $y$ gives $xy = yy\inv xy = yx$.
    \end{ifandonlyif}
    Next, we prove $y\inv xy = x$ if and only if $x\inv y\inv xy = 1$.
    \begin{ifandonlyif}
        \item [\rightimp] Suppose $y\inv xy = x$. Then multiplying both sides on the left by $x\inv$ gives $x\inv y\inv xy = x\inv x = 1$.
        \item [\leftimp] Suppose $x\inv y\inv xy = 1$. Then multiplying both sides on the left by $x$ gives $y\inv xy = xx\inv = 1x = x$.
    \end{ifandonlyif}
    The result follows.
\end{sol}

\begin{specialexercise} \label{ex1.1.19}
    Let $x \in G$ and let $a, b \in \zp$.
    \begin{subproblems}
        \item Prove that $x^{a+b} = x^a x^b$ and $(x^a)^b = x^{ab}$.
        \item Prove that $(x^a)^{-1} = x^{-a}$.
        \item Establish part (a) for arbitrary integers $a$ and $b$ (positive, negative, or zero).
    \end{subproblems}
\end{specialexercise}

\begin{solalph}
    \item Since $x^a$ has $a$ amount of $x$ and $x^b$ has $b$ amount of $x$, then $x^a x^b$ has $a + b$ amount of $x$, or $x^{a + b}$. Similarly, $(x^a)^b$ has $ab$ amount of $x$, or $x^{ab}$.
    \item Recall that $x^{-a} = (x\inv)^a$. We proceed by induction. For $a = 1$, we have $x^{-1} = (x\inv)^1$. Suppose $(x^a)\inv = (x\inv)^a$ for some $a \in \zp$. Then for $a + 1$, we have
    \[(x^{a + 1})\inv = (x^ax)\inv = x\inv(x^a)\inv = x\inv(x\inv)^a = (x\inv)^{a + 1}\]
    hence the result follows by induction.
    \item There are three cases to consider:
    \begin{itemize}
        \item If both $a, b \in \zp$, then we proceed as in part (a).
        \item If one of $a$ or $b$ is 0, we may assume without loss of generality that $b = 0$. Then $x^{a + 0} = x^ax^0 = x^a$. Moreover, we have $(x^a)^0 = 1$ by definition, so $x^{a \cdot 0} = x^0 = 1$.
        \item If one of $a$ or $b$ is negative, we may assume without loss of generality that $b < 0$, hence $-b \in \zp$. For the first part of (a), we have
        \[x^a = x^{a + b - b} = x^{a + b} x^{-b} = x^{a + b}(x^b)\inv\]
        so that $x^ax^b = x^{a + b}$. For the second part of (a), we have
        \[(x^a)^b = ((x^a)^{-b})\inv = (x^{-ab})\inv = x^{ab}.\]
        Hence, the result follows. \qh
    \end{itemize}
\end{solalph}

\begin{exercise}
    For $x$ an element in $G$, show that $x$ and $x^{-1}$ have the same order.
\end{exercise}

\begin{sol}
    Suppose $|x| = n \in \zp$. Then
    \[(x\inv)^n = (x^n)\inv = 1\inv = 1\]
    so that $|x\inv| \leq n$ and has finite order. Similarly, suppose $|x\inv| = m \in \zp$. Then
    \[x^m = ((x\inv)^m)\inv = 1\inv = 1\]
    so that $|x| \leq m$ and has finite order. Hence, $|x| = |x\inv|$. Moreover, the above shows that if either $x$ or $x\inv$ has finite order, then so does the other. Hence, they both have finite or infinite order.
\end{sol}

\begin{exercise}
    Let $G$ be a finite group and let $x$ be an element of order $n$. Prove that if $n$ is odd, then $x = (x^2)^k$ for some $k \geq 1$.
\end{exercise}

\begin{sol}
    Since $n$ is odd, then $n = 2m - 1$ for some $m \in \zp$. Since $\abs x = n$, then
    \[1 = x^n = x^{2m - 1} = (x^2)^m x\inv\]
    so that $x = (x^2)^m$.
\end{sol}

\begin{specialexercise} \label{ex1.1.22}
    If $x$ and $g$ are elements of the group $G$, prove that $|x| = |g^{-1} x g|$. Deduce that $|ab| = |ba|$ for all $a, b \in G$.
\end{specialexercise}

\begin{sol}
    We first show that $(g\inv xg)^n = g\inv x^n g$ for all $n \in \zp$ by induction. For $n = 1$, the result is clear. Suppose it holds for some $n \in \zp$. Then for $n + 1$, we have
    \[(g\inv xg)^{n+1} = (g\inv xg)^n (g\inv xg) = (g\inv x^n g)(g\inv x g) = g\inv x^{n+1} g\]
    so the result follows by induction.

    Let $|x| = n \in \zp$. Then
    \[(g\inv x g)^n = g\inv x^n g = g\inv 1 g = 1\]
    so that $|g\inv x g| \leq n$. Similarly, let $|g\inv x g| = m \in \zp$. Then
    \[x^m = (gg\inv) x^m (gg\inv) = g(g\inv xg)^m g\inv = g1g\inv = 1\]
    so that $|x| \leq m$. Hence, $|x| = |g\inv x g|$. We may use similar reasoning to show that if either $x$ or $g\inv x g$ has finite order, then so does the other. Hence, they both have finite or infinite order.

    To deduce that $|ab| = |ba|$ for all $a, b \in G$, let $x = ba$ and $g = b$. Using the fact that $|x| = |g\inv x g|$, we have
    \[|ba| = |b\inv (ba) b| = |ab|\]
    Hence, the result follows.
\end{sol}

\begin{exercise}
    Suppose $x \in G$ and $|x| = n < \infty$. If $n = st$ for some positive integers $s$ and $t$, prove that $|x^s| = t$.
\end{exercise}

\begin{sol}
    Let $|x^s| = r \in \zp$. Then
    \[1 = (x^s)^r = x^{sr}\]
    so that $r \leq t$. Moreover,
    \[(x^s)^r = x^{sr} = 1\]
    implies that $|x| = st \leq sr$, hence $t \leq r$. Thus, $|x^s| = r = t$.
\end{sol}

\begin{exercise}
    If $a$ and $b$ are commuting elements of $G$, prove that $(ab)^n = a^n b^n$ for all $n \in \z$. [Do this by induction for positive $n$ first.]
\end{exercise}

\begin{sol}
    We proceed by induction. For $n = 1$, the result is clear. Suppose it holds for some $n \in \zp$. Then for $n + 1$, we have
    \[(ab)^{n + 1} = (ab)^n (ab) = (a^n b^n)(ab) = a^{n + 1} b^{n + 1}\]
    so the result follows by induction for all positive $n$. For $n = 0$, we have $(ab)^0 = 1 = 1 \cdot 1 = a^0 b^0$. Finally, for negative $n$, we have
    \[(ab)^n = ((ab)^{-n})\inv = (a^{-n} b^{-n})\inv = (b^{-n} a^{-n})\inv = a^n b^n\]
    since $a$ and $b$ commute. Hence, the result holds for all $n \in \z$.
\end{sol}

\begin{exercise}
    Prove that if $x^2 = 1$ for all $x \in G$, then $G$ is abelian.
\end{exercise}

\begin{sol}
    Since $x^2 = 1$, then $x = x\inv$ for all $x \in G$. Let $a, b \in G$. Then
    \[ab = (ab)\inv = b\inv a\inv = ba\]
    so that $G$ is abelian.
\end{sol}

\begin{specialexercise} \label{ex1.1.26}
    Assume $H$ is a nonempty subset of $(G, \star)$ which is closed under the binary operation on $G$ and is closed under inverses, i.e., for all $h, k \in H$, $hk, \, h^{-1} \in H$. Prove that $H$ is a group under the operation $\star$ restricted to $H$ (such a subset $H$ is called a \textit{subgroup} of $G$).
\end{specialexercise}

\begin{sol}
    Let $H$ be a nonempty subset of $(G, \star)$ which is closed under the binary operation on $G$ and is closed under inverses. Since $H$ is nonempty, there exists some $h \in H$. Since $H$ is closed under inverses, then $h\inv \in H$. Since $H$ is closed under the binary operation on $G$, then $hh\inv = 1 \in H$ so that $H$ has an identity. Moreover, for any $k \in H$, its inverse $k\inv \in H$ since $H$ is closed under inverses. Since $\star$ is associative on $G$, then it is also associative on $H$. Hence, $H$ is a group under the operation $\star$ restricted to $H$.
\end{sol}

\begin{specialexercise} \label{ex1.1.27}
    Prove that if $x$ is an element of the group $G$, then $\{x^n \mid n \in \z \}$ is a subgroup. (cf. the preceding exercise) of $G$ (called the \textit{cyclic subgroup} of $G$ generated by $x$).
\end{specialexercise}

\begin{sol}
    Let $H = \{x^n \mid n \in \z \}$. Since $x^0 = 1 \in H$, then $H$ is nonempty. Moreover, for any $x^a, x^b \in H$ where $a, b \in \z$, we have
    \[x^a x^b = x^{a + b} \in H\]
    by \hyperref[ex1.1.19]{Exercise 1.1.19} so that $H$ is closed under the binary operation on $G$. Additionally, we have
    \[(x^a)\inv = x^{-a} \in H\]
    so that $H$ is closed under inverses. Hence, by the preceding exercise, $H$ is a subgroup of $G$.
\end{sol}

\newpage

\begin{exercise}
    Let $(A, \star)$ and $(B, \diamond)$ be groups and let $A \times B$ be their direct product (as defined in Example 6). Verify all the group axioms for $A \times B$:
    \begin{subproblems}
        \item[(a)] Prove that the associative law holds:
        $$\text{for all } (a_1, b_1), (a_2, b_2), (a_3, b_3) \in A \times B,
        [(a_1, b_1)(a_2, b_2)](a_3, b_3) = (a_1, b_1)[(a_2, b_2)(a_3, b_3)]$$
        \item Prove that $(1, 1)$ is the identity of $A \times B$, and
        \item Prove that the inverse of $(a, b)$ is $(a^{-1}, b^{-1})$.
    \end{subproblems}
\end{exercise}

\begin{solalph}
    \item Let $(a_1, b_1), (a_2, b_2), (a_3, b_3) \in A \times B$. Then
    \begin{align*}
        [(a_1, b_1)(a_2, b_2)](a_3, b_3) & = (a_1a_2, b_1b_2)(a_3, b_3) \\
        & = ((a_1a_2)a_3, (b_1b_2)b_3) \\
        & = (a_1(a_2a_3), b_1(b_2b_3)) \\
        & = (a_1,b_1)(a_2a_3, b_2b_3) \\
        & = (a_1, b_1)[(a_2, b_2)(a_3, b_3)]
    \end{align*}
    \item For $a \in A, b \in B$, we have $(a, b)(1, 1) = (a \star 1, b \diamond 1) = a, b)$.
    \item We have $(a, b)(a\inv, b\inv) = (a \star a\inv, b \diamond b\inv) = (1, 1)$.
\end{solalph}

\begin{exercise} \label{ex1.1.29}
    Prove that $A \times B$ is an abelian group if and only if both $A$ and $B$ are abelian.
\end{exercise}

\begin{sol}
    Let $a, a' \in A$ and $b, b' \in B$. Suppose $A \times B$ is abelian. Then
    \[(aa', bb') = (a, b)(a', b') = (a', b')(a, b) = (a'a, b'b)\]
    so that $A$ and $B$ are abelian. If $A$ and $B$ are both abelian, then
    \[(a, b)(a', b) = (aa', bb') = (a'a, b'b) = (a', b')(a, b)\]
    hence $A \times B$ is abelian.
\end{sol}

\begin{exercise}
    Prove that the elements $(a, 1)$ and $(1, b)$ of $A \times B$ commute and deduce that the order of $(a, b)$ is the least common multiple of $|a|$ and $|b|$.
\end{exercise}

\begin{sol}
    Let $a \in A$ and $b \in B$. Then
    \[(a, 1)(1, b) = (a \cdot 1, 1 \cdot b) = (a, b) = (1 \cdot a, b \cdot 1) = (1, b)(a, 1)\]
    so that $(a, 1)$ and $(1, b)$ commute. Let $\ell = \lcm(|a|, |b|)$. Then
    \[(a, b)^{\ell} = (a^{\ell}, b^{\ell}) = (1, 1)\]
    so that $|(a, b)| \leq \ell$. Moreover, if $(a, b)^k = (1, 1)$ for some $k \in \zp$, then $a^k = 1$ and $b^k = 1$ so that both $|a|$ and $|b|$ divide $k$. Hence, $\ell$ divides $k$ so that $\ell \leq k$. Thus, $|(a, b)| = \ell$.
\end{sol}

\newpage

\begin{exercise} \label{ex1.1.31}
    Prove that any finite group $G$ of even order contains an element of order 2. [Let $t(G)$ be the set $\{g \in G \mid g \neq g^{-1}\}$. Show that $t(G)$ has an even number of elements and every nonidentity element of $G - t(G)$ has order 2.]
\end{exercise}

\begin{sol}
    Let $g \in G$. If $g \neq g\inv$, then $g \in t(G)$. Moreover, $g\inv \neq (g\inv)\inv = g$ so that $g\inv \in t(G)$ as well. It follows that the elements of $t(G)$ may be paired off as $(g, g\inv)$ where $g \neq g\inv$, hence $t(G)$ has an even number of elements. Since $G$ has even order, then $G - t(G)$ also has an even number of elements. Note that $1 \notin t(G)$ since $1 = 1\inv$. Thus, $G - t(G)$ contains at least one nonidentity element $x$. Since $x \notin t(G)$, then $x = x\inv$ so that $x^2 = 1$ and $|x| = 2$.
\end{sol}

\begin{exercise} \label{ex1.1.32}
    If $x$ is an element of finite order $n$ in $G$, prove that the elements $1, x, x^2, \dots, x^{n-1}$ are all distinct. Deduce that $|x| \leq |G|$.
\end{exercise}

\begin{sol}
    Let $a, b \in \zp$ be such that $0 \leq a < b < n$. Suppose $x^a = x^b$. Then
    \[1 = x^b x^{-a} = x^{b - a}\]
    so that $|x| \leq b - a < n$, which is a contradiction. Hence, the elements $1, x, x^2, \dots, x^{n-1}$ are all distinct. Since these elements are all in $G$, then $|x| = n \leq |G|$.
\end{sol}

\begin{exercise} \label{ex1.1.33}
    Let $x$ be an element of finite order $n$ in $G$.
    \begin{subproblems}
        \item Prove that if $n$ is odd then $x^i \neq x^{-i}$ for all $i = 1, 2, \dots, n-1$.
        \item Prove that if $n = 2k$ and $1 \leq i < n$ then $x^i = x^{-i}$ if and only if $i = k$.
    \end{subproblems}
\end{exercise}

\begin{solalph}
    \item Assume, by way of contradiction, that $x^i = x^{-i}$ for some $i$ where $1 \leq i < n$. Then
    \[1 = x^i x^i = x^{2i}\]
    Note that $2i$ is even, so $2i \neq n$ since $n$ is odd. Noting that $2i < n$, then the only possibility for $x^{2i} = 1$ is if $2i = 0$, which is impossible since $i \geq 1$. This is a contradiction, hence $x^i \neq x^{-i}$ for all $i = 1, 2, \dots, n-1$.
    \item
    \begin{ifandonlyif}
        \item [\rightimp] Suppose $x^i = x^{-i}$ for some $i$ where $1 \leq i < n$. Then
        \[1 = x^i x^i = x^{2i}\]
        Since $x^n = x^{2k} = 1$, then the only $i$ such that $1 \leq i < 2k$ and $x^{2i} = 1$ is $i = k$.
        \item [\leftimp] Suppose $i = k$. Then
        \[x^i = x^k = x^{2k - k} = x^{n - k} = x^{-k} = x^{-i} \qh\]
    \end{ifandonlyif}
\end{solalph}

\begin{exercise}
    If $x$ is an element of infinite order in $G$, prove that the elements $x^n, n \in \z$ are all distinct.
\end{exercise}

\begin{sol}
    Assume, by way of contradiction, that $x^a = x^b$ for some $a, b \in \z$ where $a < b$. Then
    \[1 = x^b x^{-a} = x^{b - a}\]
    so that $|x| \leq b - a$, contradicting that $x$ has infinite order. Hence, the elements $x^n, n \in \z$ are all distinct.
\end{sol}

\begin{exercise} \label{ex1.1.35}
    If $x$ is an element of finite order $n$ in $G$, use the Division Algorithm to show that any integral power of $x$ equals one of the elements in the set $\{1, x, x^2, \dots, x^{n-1}\}$ (so these are all the distinct elements of the cyclic subgroup (cf. \hyperref[ex1.1.27]{Exercise 1.1.27}) of $G$ generated by $x$).
\end{exercise}

\begin{sol}
    Let $\abs x = n \in \zp$ and let $m \in \z$. By the Division Algorithm, there exist unique integers $q$ and $r$ such that
    \[m = qn + r\]
    where $0 \leq r < n$. Then
    \[x^m = x^{qn + r} = x^{qn} x^r = (x^n)^q x^r = 1^q x^r = x^r\]
    so that any integral power of $x$ equals one of the elements in the set $\{1, x, x^2, \dots, x^{n-1}\}$.
\end{sol}

\begin{specialexercise} \label{ex1.1.36} 
    Assume $G = \{1, a, b, c\}$ is a group of order 4 with identity 1. Assume also that $G$ has no elements of order 4 (so by \hyperref[ex1.1.32]{Exercise 1.1.32}, every element has order $\leq 3$). Use the cancellation laws to show that there is a unique group table for $G$. Deduce that $G$ is abelian.
\end{specialexercise}

\begin{sol}
    Since $G$ has even order, it has an element of order 2 by \hyperref[ex1.1.31]{Exercise 1.1.31}. Without loss of generality, let $a$ be that element so that $a^2 = 1$. Since there are no elements of order 4, then the remaining elements $b$ and $c$ may have order 2 or 3.

    We now observe that $ab \neq 1$ since that would imply that $b = a\inv = a$ contradicting that $a$ and $b$ are distinct elements. Similarly, $ab \neq a$ since that would imply that $b = 1$, and $ab \neq b$ since that would imply that $a = 1$. Thus, $ab = ba = c$. Similarly, $ac \neq 1, a, c$ so that $ac = ca = b$, hence $b^2 = (ca)(ac) = c^2$.

    We now consider the possible orders of $b$ and $c$. If $b^2 \neq 1$, then $\abs b = 3$. But then $b^3 = b b^2 = b c = a$, contradicting that $\abs b = 3$. Thus, $b^2 = 1$. Similarly, $c^2 = 1$. The group table for $G$ is therefore as follows:
    \[
    \begin{array}{c|cccc}
        \star & 1 & a & b & c \\
        \hline 
        1 & 1 & a & b & c \\
        a & a & 1 & c & b \\
        b & b & c & 1 & a \\
        c & c & b & a & 1
    \end{array}\]
    from which we may deduce that $G$ is abelian.
\end{sol}

\newpage

\subsection{Dihedral Groups}

In these exercises, $D_{2n}$ has the usual presentation 
$D_{2n} = \gen{r, s \mid r^n = s^2 = 1, \ rs = sr\inv}$.

\begin{exercise} \label{ex1.2.1}
    Compute the order of each of the elements in the following groups:
    \begin{subproblems}
        \item $D_6$ \quad (b) $D_8$ \quad (c) $D_{10}$
    \end{subproblems}
\end{exercise}

\begin{sol}
    For any reflection $sr^k \in D_{2n}$, it is clear that $sr^k \neq 1$. Moreover, $(sr^k)^2 = s(r^ks)r^k = s(sr^{-k})r^k = r^{-k}r^k = 1$ so that every reflection has order 2. We now compute the orders of the rotations in each group.
    \begin{subproblems}
        \item $D_6 = \{1, r, r^2, s, sr, sr^2\}$. Then $|1| = 1$ and $|r| = |r^2| = 2$.
        \item $D_8 = \{1, r, r^2, r^3, s, sr, sr^2, sr^3\}$. Then $|1| = 1, |r^2| = 2$ and $|r| = |r^3| = 4$.
        \item $D_{10} = \{1, r, r^2, r^3, r^4, s, sr, sr^2, sr^3, sr^4\}$. Then $|1| = 1$ and $|r| = |r^2| = |r^3| = |r^4| = 5$. \qh
    \end{subproblems}
\end{sol}

\begin{exercise}
    Use the generators and relations above to show that if $x$ is any element of $D_{2n}$ which is not a power of $r$, then $rx = xr\inv$.
\end{exercise}

\begin{sol}
    If $x$ is not a power of $r$, then $x$ is a reflection, i.e., $x = sr^k$ for some $0 \leq k < n$. Then
    \[rx = r(sr^k) = (rs)r^k = (sr^{-1})r^k = s(r^{-1}r^k) = s r^{k - 1} = (sr^k) r^{-1} = x r^{-1} \qh\]
\end{sol}

\begin{exercise}
    Use the generators and relations above to show that every element of $D_{2n}$ which is not a power of $r$ has order $2$. Deduce that $D_{2n}$ is generated by the two elements $s$ and $sr$, both of which have order $2$. [cf. \hyperref[ex1.1.33]{Exercise 1.1.33}.]
\end{exercise}
 
\begin{sol}
    Every element of $D_{2n}$ which is not a power of $r$ is a reflection. For this solution, see \hyperref[ex1.2.1]{Exercise 1.2.1}.

    To see that $s$ and $sr$ generate $D_{2n}$, note that for any rotation $r^k \in D_{2n}$ and any reflection $sr^k \in D_{2n}$ where $0 \leq k < n$, then
    \[r^k = (s(sr))^k \longand sr^k = s(s(sr))^k.\]
    Hence, $D_{2n}$ is generated by $s$ and $sr$.
\end{sol}

\begin{exercise}
    If $n = 2k$ is even and $n \ge 4$, show that $z = r^k$ is an element of order $2$ which commutes with all elements of $D_{2n}$. Show also that $z$ is the only nonidentity element of $D_{2n}$ which commutes with all elements of $D_{2n}$. [cf. \hyperref[ex1.1.33]{Exercise 1.1.33}.]
\end{exercise}
   
\begin{sol}
    Since $k < n$, then $r^k \neq 1$. Moreover, $(r^k)^2 = r^{2k} = r^n = 1$ so that $|r^k| = 2$. To see that $r^k$ commutes with all elements of $D_{2n}$, note that every element of $D_{2n}$ is of the form $s^ir^j$ where $i \in \{0, 1\}$ and $0 \leq j < n$. Then
    \[r^k(s^ir^j) = s^ir^{-k}r^j = s^ir^kr^j = (s^ir^j)r^k.\]
    Now suppose $w \in D_{2n}$ commutes with all elements of $D_{2n}$. If $w$ is a rotation, then $w = r^m$ for some $0 \leq m < n$. Since $w$ commutes with $s$, then $sw = ws$. But $sw = w\inv s$ as well, so that $w = w\inv$. By \hyperref[ex1.1.33]{Exercise 1.1.33}, then $m = k$ since $n$ is even.

    Suppose $w$ is a reflection, so that $w = sr^m$ for some $0 \leq m < n$. Since $w$ commutes with $r$, then $rw = wr$. But $rw = w r\inv$ as well, so that $r = r\inv$. Then $r^2 = 1$, contradicting that $|r| = n \geq 4$. Hence, $w$ must be a rotation, and we have shown that $w = r^k$.
\end{sol}

\begin{exercise}
    If $n$ is odd and $n \geq 3$, show that the identity is the only element of $D_{2n}$ which commutes with all elements of $D_{2n}$. [cf. \hyperref[ex1.1.33]{Exercise 1.1.33}]
\end{exercise}

\begin{sol}
    We use a similar argument as the previous exercise. If there was such a nonidentity $w \in D_{2n}$, we may use the odd case in \hyperref[ex1.1.33]{Exercise 1.1.33} if we assume $w$ is a rotation. We would get $r^k = r^{-k}$, which is never true since $n$ is odd. If $w$ is a reflection, then we would have the same contradiction as before. Hence, the identity is the only element of $D_{2n}$ which commutes with all elements of $D_{2n}$.
\end{sol}

\begin{exercise}
    Let $x$ and $y$ be elements of order $2$ in any group $G$. Prove that if $t = xy$, then $tx = xt\inv$ (so that if $n = |xy| < \infty$ then $x, y$ satisfy the same relations in $G$ as $s, r$ do in $D_{2n}$).
\end{exercise}

\begin{sol}
    Since $|x| = |y| = 2$, then $x = x\inv$ and $y = y\inv$. Then
    \[tx = (xy)x = x(yx) = x(xy)\inv = x t\inv \qh\]
\end{sol}

\begin{exercise}
    Show that $\gen{a, b \mid a^2 = b^2 = (ab)^n = 1}$ gives a presentation for $D_{2n}$ in terms of the two generators $a$ and $b$ of order $2$ computed in Exercise 3 above. [Show that the relations for $r$ and $s$ follow from the relations for $a$ and $b$ and conversely, the relations for $a$ and $b$ follow from those for $r$ and $s$.]
\end{exercise}

\begin{sol}
    Suppose $a^2 = b^2 = (ab)^n = 1$. The natural choices for $r$ and $s$ are $r = ab$ and $s = a$. Then
    \[r^n = (ab)^n = 1, \quad s^2 = a^2 = 1, \quad rs = (ab)a = a(ba) = a(b\inv a\inv) = a(ab)\inv = sr\inv \]
    Conversely, suppose $r^n = s^2 = 1$ and $rs = sr\inv$. The natural choices for $a$ and $b$ are $a = s$ and $b = sr$. Then
    \[a^2 = s^2 = 1, \quad b^2 = (sr)(sr) = s(r s)r = s(sr\inv)r = (s s) r\inv r = 1, \quad (ab)^n = (s(sr))^n = r^n = 1 \qh\]
\end{sol}

\begin{exercise}
    Find the order of the cyclic subgroup of $D_{2n}$ generated by $r$ (cf. \hyperref[ex1.1.27]{Exercise 1.1.27}).
\end{exercise}

\begin{sol}
    Let $H = \gen r$ be the cyclic subgroup of $D_{2n}$ generated by $r$. Since $\abs r = n$, then by \hyperref[ex1.1.35]{Exercise 1.1.35}, the distinct elements of $H$ are $1, r, r^2, \dots, r^{n-1}$. It follows that $|H| = n$.
\end{sol}

In each of Exercises 9 to 13 you can find the order of the group of rigid motions in $\r^3$ (also called the group of rotations) of the given Platonic solid by following the proof for the order of $D_{2n}$: find the number of positions to which an adjacent pair of vertices can be sent. Alternatively, you can find the number of places to which a given face may be sent and, once a face is fixed, the number of positions to which a vertex on that face may be sent. 

\begin{exercise} \label{ex1.2.9} 
    Let $G$ be the group of rigid motions in $\r^3$ of a tetrahedron. Show that $|G| = 12$.
\end{exercise}

\begin{sol}
    Label the vertices of a tetrahedron 1 through 4 so that vertex 1 has 4 choices. Then vertex 2 has 3 remaining choices, and vertex 3 and 4 are determined after. It follows that $4(3) = 12$ symmetries.
\end{sol}

\begin{exercise} \label{ex1.2.10}
    Let $G$ be the group of rigid motions in $\r^3$ of a cube. Show that $|G| = 24$.
\end{exercise}

\begin{sol}
    8 choices for vertex 1 and 3 adjacent vertices for vertex 2; $8(3) = 24$ symmetries.
\end{sol}

\begin{exercise} \label{ex1.2.11}
    Let $G$ be the group of rigid motions in $\r^3$ of an octahedron. Show that $|G| = 24$.
\end{exercise}

\begin{sol}
    6 choices for vertex 1 and 4 adjacent vertices for vertex 2; $6(4) = 24$ symmetries.
\end{sol}

\begin{exercise}
    Let $G$ be the group of rigid motions in $\r^3$ of a dodecahedron. Show that $|G| = 60$.
\end{exercise}

\begin{sol}
    20 choices for vertex 1 and 3 adjacent vertices for vertex 2; $20(3) = 60$ symmetries.
\end{sol}

\begin{exercise}
    Let $G$ be the group of rigid motions in $\r^3$ of an icosahedron. Show that $|G| = 60$.
\end{exercise}

\begin{sol}
    12 choices for vertex 1 and 5 adjacent vertices for vertex 2; $12(5) = 60$ symmetries.
\end{sol}

\begin{exercise}
    Find a set of generators for $\z$.
\end{exercise}

\begin{sol}
    Any integer is of the form $n1$, so $\z = \gen 1$.
\end{sol}

\begin{exercise}
    Find a set of generators and relations for $\intmod$.
\end{exercise}

\begin{sol}
    Each element of $\intmod$ is of the form $k\bar 1$ for $0 \leq k < n$. Then $\intmod = \gen{\bar 1 \mid n\bar 1 = \bar 0}$.
\end{sol}

\begin{exercise}
    Show that the group $\langle x_1, y_1 \mid x_1^2 = y_1^2 = (x_1 y_1)^2 = 1 \rangle$ is the dihedral group $D_4$ (where $x_1$ may be replaced by the letter $r$ and $y_1$ by $s$). (Show that the last relation is the same as $x_1 y_1 = y_1 x_1^{-1}$.)
\end{exercise}

\begin{sol}
    Let $D_4$ have the usual presentation $\gen{r, s \mid r^2 = s^2 = 1, rs = sr\inv}$. Since $r^2 = 1$, then $r = r\inv$, hence $rs = sr\inv$ is equivalent to $rs = sr$. Let $x_1 = r$ and $y_1 = s$. Then $x_1^2 = r^2 = 1$ and $y_1^2 = s^2 = 1$. Suppose $rs = sr$. Then
    \[1 = rs sr = (rs)^2 = x_1 y_1 x_1 y_1 = (x_1 y_1)^2\]
    Conversely, suppose $(x_1 y_1)^2 = 1$. Then
    \[rs = x_1 y_1 = (x_1 y_1)^{-1} = y_1^{-1} x_1^{-1} = y_1 x_1 = sr\]
    Thus, the two presentations are equivalent.
\end{sol}

\newpage

\begin{exercise}
    Let $X_{2n}$ be the group whose presentation is displayed in (1.2).
    \[X_{2n} = \gen{x, y \mid x^n = y^2 = 1, xy = yx^2}\]
    \begin{subproblems}
        \item Show that if $n = 3k$, then $X_{2n}$ has order $6$, and it has the same generators and relations as $D_6$ when $x$ is replaced by $r$ and $y$ by $s$.
        \item Show that if $(3, n) = 1$, then $x$ satisfies the additional relation: $x = 1$. In this case deduce that $X_{2n}$ has order $2$. [use the facts that $x^n = 1$ and $x^3 = 1$.]
    \end{subproblems}
\end{exercise}

\begin{solalph}
    \item If $n = 3k$, then
    \[X_{2n} = X_{6k} = \gen{x, y \mid x^{3k} = y^2 = 1, xy = yx^2}\]
    As shown in the text, 
    \[x = xy^2 = yx^2y = yxyx^2 = y^2x^4 = x^4\]
    so that $x^3 = 1$. To show that $X_{6k}$ has the same generators and relations as $D_6$, assume the relations of $X_{6k}$ hold. Set $r = x$ and $s = y$. We then have $r^3 = x^3 = 1$, $s^2 = y^2 = 1$, and
    \[rs = xy = yx^2 = sr^2 = sr\inv\]
    Now suppose the relations of $D_6$ hold. Then
    \[x^n = r^{3k} = (r^3)^k = 1^k = 1, \quad y^2 = s^2 = 1\]
    and
    \[xy = rs = sr\inv = sr^2 = yx^2\]
    Thus, the two presentations are equivalent, and $|X_{6k}| = |D_6| = 6$.
    \item Since $(3, n) = 1$, there exist integers $a$ and $b$ such that $3a + nb = 1$. From the relations of $X_{2n}$, we have $x^3 = 1$ and $x^n = 1$. Then
    \[x = x^{3a + nb} = (x^3)^a (x^n)^b = 1^a 1^b = 1\]
    so that $X_{2n} = {1, y}$ where $y^2 = 1$. It follows that $|X_{2n}| = 2$.
\end{solalph}

\begin{exercise}
    Let $Y$ be the group whose presentation is displayed in (1.3).
    \begin{subproblems}
        \item Show that $v^2 = v^{-1}$. [Use the relation $v^3 = 1$.]
        \item Show that $v$ commutes with $u^3$. [Show that $v^2 u^3v = u^3$ by writing the left-hand side as $(v^2u^2)(uv)$ and using the relations to reduce this to the right-hand side. Then use part (a).]
        \item Show that $v$ commutes with $u$. [Show that $u^9 = u$ and then use part (b).]
        \item Show that $uv = 1$. [Use part (c) and the last relation.]
        \item Show that $u = 1$. Deduce that $v = 1$, and conclude that $Y = 1$. [Use part (d) and the equation $u^4 v^3 = 1$.]
    \end{subproblems}
\end{exercise}

\begin{solalph}
    \item $v^3 = 1 \implies v^2 = v\inv$.
    \item From $uv = v^2u^2$, we have $vuv = u^2$ so that $vu = u^2v\inv = u^2v^2$. It follows that $v^2u^3v = (v^2u^2)(uv) = (uv)(uv) = u(vu)v = u(u^2v^2)v = u^3$.
    \item It follows that $u^9 = (u^4)^2u = u$. Then $uv = (u^3)^3v = v(u^3)^3 = vu$.
    \item $uv = v^2u^2 = u^2v^2$. Then $1 = uv$.
    \item $u^4v^3 = u(uv)^3 = u1^3 = 1$. Then $1v = 1$, and $u = v = 1$, so $Y = 1$.
\end{solalph}

\newpage

\subsection{Symmetric Groups}

\begin{exercise} \label{ex1.3.1}
    Let $\sigma$ be the permutation
    \[1 \mapsto 3, \quad 2 \mapsto 4, \quad 3 \mapsto 5, \quad 4 \mapsto 2, \quad 5 \mapsto 1\]
    and let $\tau$ be the permutation
    \[1 \mapsto 5, \quad 2 \mapsto 3, \quad 3 \mapsto 2, \quad 4 \mapsto 4, \quad 5 \mapsto 1\]
    Find the cycle decompositions of each of the following permutations:
    $\sigma, \tau, \sigma^2, \sigma\tau, \tau\sigma,$ and $\tau^2\sigma.$
\end{exercise}

\begin{sol}
    \begin{align*}
        \sigma & = (1\ 3\ 5)(2\ 4) & \tau & = (1\ 5)(2\ 3) \\
        \sigma^2 & = (1\ 5\ 3) & \sigma\tau & = (2\ 5\ 3\ 4)\\
        \tau\sigma & = (1\ 2\ 4\ 3) & \tau^2\sigma & = (1\ 3\ 5)(2\ 4) \qh
    \end{align*}
\end{sol}

\begin{exercise} \label{ex1.3.2} 
    Let $\sigma$ be the permutation
    \begin{align*}
        1 &\mapsto 13 & 2 &\mapsto 2 & 3 & \mapsto 15 & 4 & \mapsto 14 & 5 & \mapsto 10  \\
        6 &\mapsto 7 & 7 &\mapsto 12 & 8 & \mapsto 9 & 9 & \mapsto 3 & 10 & \mapsto 1 \\
        11 &\mapsto 7 & 12 &\mapsto 9 & 13 & \mapsto 5 & 14 & \mapsto 11 & 15 & \mapsto 8
    \end{align*}
    and let $\tau$ be the permutation
    \begin{align*}
        1 &\mapsto 14 & 2 &\mapsto 9 & 3 &\mapsto 10 & 4 &\mapsto 2 & 5 &\mapsto 12 \\
        6 &\mapsto 6 & 7 &\mapsto 5 & 8 &\mapsto 11 & 9 &\mapsto 15 & 10 &\mapsto 3 \\
        11 &\mapsto 8 & 12 &\mapsto 7 & 13 &\mapsto 4 & 14 &\mapsto 1 & 15 &\mapsto 13
    \end{align*}
    Find the cycle decompositions of the following permutations: $\sigma, \tau, \sigma^2, \sigma\tau, \tau\sigma$, and $\tau^2\sigma.$
\end{exercise}

\begin{sol}
    \begin{align*}
        \sigma & = (1\ 13\ 5\ 1)(3\ 15\ 8)(4\ 14\ 11\ 7\ 12\ 9) & \tau & = (1\ 14)(2\ 9\ 15\ 13\ 4)(3\ 10)(5\ 12\ 7)(8\ 11)\\
        \sigma^2 & = (1\ 5)(3\ 8\ 15)(4\ 11\ 12)(7\ 9\ 14)(10\ 13) & \sigma\tau & = (1\ 11\ 3)(2\ 4)(5\ 9\ 8\ 7\ 10\ 15)(8\ 10\ 14)\\
        \tau\sigma & = (1\ 4)(2\ 9)(3\ 13\ 12\ 15\ 11\ 5)(8\ 10\ 14) & \tau^2\sigma & = (1\ 2\ 15\ 8\ 3\ 4\ 14\ 11\ 12\ 13\ 7\ 5\ 10) \qh
    \end{align*}
\end{sol}

\begin{exercise}
    For each of the permutations whose cycle decompositions were computed in the preceding two exercises, compute its order.
\end{exercise}

\begin{sol}
    Note that the order of a cycle decomposition is the $\lcm$ of the orders of each of the cycles, and that the order of an $m$-cycle is $m$. Then the orders are the following (the first number will be for Exercise 1, and the second for Exercise 2):
    \begin{align*}
        \abs\sigma & = 6, 12 & \abs\tau & = 2, 30 \\
        \abs{\sigma^2} & = 3, 6 & \abs{\sigma\tau} & = 4, 6 \\
        \abs{\tau\sigma} & = 4, 6 & \abs{\tau^2\sigma} & = 6, 13 \qh
    \end{align*}
\end{sol}

\begin{exercise} \label{ex1.3.4}
    Compute the order of each of the elements in the following groups:
    \begin{subproblems}
        \item $S_3$
        \item $S_4$
    \end{subproblems}
\end{exercise}

\begin{solalph}
    \item
    $$\begin{array}{c|c}
        \text{Permutation} & \text{Order} \\
        \hline
        1 & 1 \\
        (12) & 2 \\
        (13) & 2 \\
        (23) & 2 \\
        (123) & 3 \\
        (132) & 3
    \end{array}$$
    \item 
    $$\begin{array}{c|c}
            \text{Permutation} & \text{Order} \\
            \hline
            1 & 1 \\
            (12) & 2 \\
            (13) & 2 \\
            (14) & 2 \\
            (23) & 2 \\
            (24) & 2
    \end{array}\enspace
    \begin{array}{c|c}
        \text{Permutation} & \text{Order} \\
        \hline
        (34) & 2 \\
        (123) & 3 \\
        (124) & 3 \\
        (132) & 3 \\
        (134) & 3 \\
        (142) & 3
    \end{array}\enspace
    \begin{array}{c|c}
        \text{Permutation} & \text{Order} \\
        \hline
        (143) & 3 \\
        (234) & 3 \\
        (243) & 3 \\
        (1234) & 4 \\
        (1243) & 4 \\
        (1324) & 4
    \end{array}\enspace
    \begin{array}{c|c}
        \text{Permutation} & \text{Order} \\
        \hline
        (1342) & 4 \\
        (1423) & 4 \\
        (1432) & 4 \\
        (12)(34) & 2 \\
        (13)(24) & 2 \\
        (14)(23) & 2
    \end{array}$$
\end{solalph}

\begin{exercise}
    Find the order of $(1\ 12\ 8\ 10\ 4)(2\ 13)(5\ 11\ 7)(6\ 9)$.
\end{exercise}

\begin{sol}
    There are cycles of lengths 5, 2, 3, and 2. Then the order is $\lcm(5, 2, 3, 2) = 30$.
\end{sol}

\begin{exercise}
    \label{ex1.3.6} Write out the cycle decomposition of each element of order 4 in $S_4$.
\end{exercise}

\begin{sol}
    See \hyperref[ex1.3.4]{Exercise 1.3.4}.
\end{sol}

\begin{exercise}
    \label{ex1.3.7} Write out the cycle decomposition of each element of order 2 in $S_4$.
\end{exercise}

\begin{sol}
    See \hyperref[ex1.3.4]{Exercise 1.3.4}.
\end{sol}

\begin{exercise}
    Prove that if $\Omega = \{1,2,3,\dots\}$, then $S_\Omega$ is an infinite group (do not say $\infty! = \infty$).
\end{exercise}

\begin{sol}
    For each $n \in \zp$, consider the permutation $\sigma_n$ defined by
    \[\sigma_n = (2n - 1\ 2n)\]
    Note that $\sigma_n \in S_\Omega$. Moreover, for $m, n \in \zp$ with $m \neq n$, then $\sigma_m \neq \sigma_n$, since $\sigma_m$ moves $2m - 1$ while $\sigma_n$ does not, hence each $\sigma_n$ is distinct. Then the set
    \[A = \{\sigma_n \mid n \in \zp\} \subseteq S_\Omega\]
    is infinite, so $S_\Omega$ is infinite as well.
\end{sol}

\begin{exercise}
    \begin{subproblems}
        \item Let $\sigma$ be the 12-cycle $(1\ 2\ 3\ 4\ 5\ 6\ 7\ 8\ 9\ 10\ 11\ 12)$. For which positive integers $i$ is $\sigma^i$ also a 12-cycle?
        \item Let $\tau$ be the 8-cycle $(1\ 2\ 3\ 4\ 5\ 6\ 7\ 8)$. For which positive integers $i$ is $\tau^i$ also an 8-cycle?
        \item Let $\omega$ be the 14-cycle $(1\ 2\ 3\ 4\ 5\ 6\ 7\ 8\ 9\ 10\ 11\ 12\ 13\ 14)$. For which positive integers $i$ is $\omega^i$ also a 14-cycle?
    \end{subproblems}
\end{exercise}

\begin{sol}
    We will compute the explicit powers for the first problem, notice a pattern with the integers that produce a 12-cycle, and apply that pattern to the other problems.
    \begin{subproblems}
        \item For each integer between 1 and 11, we compute the cycle:
        \begin{align*}
            \sigma^1 & = \sigma & \sigma^2 & = (1\ 3\ 5\ 7\ 9\ 11)(2\ 4\ 6\ 8\ 10\ 12) \\
            \sigma^3 & = (1\ 4\ 7\ 10)(2\ 5\ 8\ 11)(3\ 6\ 9\ 12)& \sigma^4 & = (1\ 5\ 9)(2\ 6\ 10)(3\ 7\ 11)(4\ 8\ 12) \\
            \sigma^5 & = \sigma & \sigma^6 & = (1\ 7)(2\ 8)(3\ 9)(4\ 10)(5\ 11)(6\ 12) \\
            \sigma^7 & = \sigma & \sigma^8 & = \sigma^4 \\
            \sigma^9 & = \sigma^3 & \sigma^{10} & = \sigma^2 \\
            \sigma^{11} & = \sigma & 
        \end{align*}
        From direct calculations, it seems that the integers $i$ such that $(12, i) = 1$ produce a 12-cycle, while $i$ such that $(12, i) = k$ produces $k$ $12/k$-cycles. In this particular case, the set of integers that produce a 12-cycle is the set $\{x + 12k \mid x \in \{1, 5, 7, 11\}, k \in \z\}$.
        \item $\{x + 8k \mid x \in \{1, 3, 5, 7\}, k \in \z\}$.
        \item $\{x + 14k \mid x \in \{1, 3, 5, 9, 11, 13\}, k \in \z\}$. \qh
    \end{subproblems}
\end{sol}

\begin{exercise} \label{ex1.3.10}
    Prove that if $\sigma$ is the $m$-cycle $(a_1\ a_2\ \dots\ a_m)$, then for all $i \in \{1,2,\dots,m\}$, $\sigma^i(a_k) = a_{k+i}$, where $k+i$ is replaced by its least residue mod $m$ when $k+i > m$. Deduce that $|\sigma| = m$.
\end{exercise}

\begin{sol}
    We proceed by induction on the first part. For $i = 1$, then $\sigma^1(a_k) = \sigma(a_k) = a_{k+1}$, where $k + 1$ is replaced by its least residue mod $m$ when $k + 1 > m$. Suppose the statement is true for some $i$. Then
    \[\sigma^{i + 1}(a_k) = \sigma(\sigma^i(a_k)) = \sigma(a_{k + i}) = a_{k + i + 1},\]
    where $k + i + 1$ is replaced by its least residue mod $m$ when $k + i + 1 > m$. By induction, the statement is true for all $i \in \zp$.

    To determine the order of $\sigma$, note that there are $m$ distinct elements $a_1, a_2, \ldots, a_m$ in the cycle. Then for any $i$ where $1 \leq i < m$, then $\sigma^i(a_1) = a_{1 + i} \neq a_1$. For $\sigma^m$, then $\sigma^m(a_1) = a_{1 + m} = a_1$. In particular, this holds for all $a_k$. Then $\sigma^m = \id$, and $|\sigma| = m$.
\end{sol}

\newpage

\begin{specialexercise}
    Let $\sigma$ be the $m$-cycle $(1\ 2\ \dots\ m)$. Show that $\sigma^i$ is also an $m$-cycle if and only if $i$ is relatively prime to $m$.
\end{specialexercise}

\begin{solitem}
    \item [\rightimp] Suppose $\sigma^i$ is an $m$-cycle, and assume, by way of contradiction, that $(m, i) = d > 1$. Then there exist $a, b \in \zp$ such that $ad = i$ and $bd = m$. Since $|\sigma^i| = m$ because it is an $m$-cycle, then
    \[(\sigma^i)^b = (\sigma^{ad})^b = (\sigma^{bd})^a = (\sigma^m)^a = \id^a = \id.\]
    However, $bd = m$ with $d > 1$ implies that $b < m$, so $|\sigma^i| \leq b < m$, contradicting that $|\sigma^i| = m$. It follows that $(m, i) = 1$.
    \item [\leftimp] Suppose that $(m, i) = 1$. Denote $x' \equiv x \bmod m$. We claim that the integers
    \[(1 + i)', (1 + 2i)', \ldots, (1 + (m - 1)i)'\]
    are all distinct. Suppose $(1 + xi)' = (1 + yi)'$ for some $0 \leq x, y \leq m - 1$ with $x \neq y$. Then $i(x - y) \equiv 0 \bmod m$. Since $(m, i) = 1$, then $m \mid (x - y)$. However, since $1 - m \leq x - y \leq m - 1$, then the only choice for $x - y$ is 0, or that $x = y$. Then we have exactly $m$ distinct integers so that
    \[\sigma^i = (1\ (1 + i)'\ (1 + 2i)'\ \ldots\ (1 + (m - 1)i)').\]
    Thus, $\sigma^i$ is an $m$-cycle.
\end{solitem}

\begin{exercise} \label{ex1.3.12}
    \begin{subproblems}
        \item If $\tau = (1\ 2)(3\ 4)(5\ 6)(7\ 8)(9\ 10)$, determine whether there is an $n$-cycle ($n \ge 10$) with $\tau = \sigma^k$ for some integer $k$.
        \item If $\tau = (1\ 2)(3\ 4\ 5)$, determine whether there is an $n$-cycle ($n \ge 5$) with $\tau = \sigma^k$ for some integer $k$.
    \end{subproblems}
\end{exercise}

\begin{solalph}
    \item We first observe that for any $n$-cycle $\sigma$, then $\sigma^k$ sends any element $a_i$ to $a_{i + k}$, where $i + k$ is replaced by its least residue mod $n$ when $i + k > n$. In this specific problem, $\tau$ consists of 2-cycles, which means that $\sigma^k$ must send each element to itself after two applications. Then $2k \equiv 0 \bmod n$, but $k \not\equiv 0 \bmod n$ since $\sigma^k \neq 1$. It follows that $n \mid 2k$ but $n \nmid k$, so it must be that $n = 2k$. We see that $\sigma^k$ will be a product of $k$ 2-cycles. However, $\tau$ is a product of 5 2-cycles, so $k = 5$ and $n = 10$. Thus, there exists an $n$-cycle $\sigma$ such that $\sigma^5 = \tau$ when $n = 10$, and $\sigma = (1\ 6\ 2\ 7\ 3\ 8\ 4\ 9\ 5\ 10)$.
    \item We first prove that $(ac, bc) = c(a, b)$. Let $d = (a, b)$ and $d' = (ac, bc)$. Then there exist $x, y \in \z$ such that $ax + by = d$. Then $acx + bcy = dc$. Since $d'$ is the smallest integer that is written of the form $acx' + bcy'$ for $x', y' \in \z$ (if there were any greater, then $d' \neq (ac, bc)$), then $d' \leq dc$. Moreover, $d \mid a$ and $d \mid b$ implies $cd \mid ac$ and $cd \mid bc$. Then $cd \mid d'$, or $cd \leq d'$. Then $dc = d'$, or $c(a, b) = (ac, dc)$.
    
    Suppose there existed some $n$-cycle $\sigma$ and $k$ such that $\sigma^k = \tau$. Then $\sigma^{2k}(1) = 1$, or $2k \equiv 0 \bmod n$. Moreover, $\sigma^{3k}(3) = 3$, or that $3k \equiv 0 \bmod n$. Then $n \mid 2k$ and $n \mid 3k$ so that $n \mid (2k,3k)$. By the above, then $(2k, 3k) = k(2, 3) = k$, and $n \mid k$. But then $\sigma^k = (\sigma^n)^q = 1$, contradicting that $\sigma^k = \tau$.
\end{solalph}

\newpage

\begin{exercise}
    Show that an element has order 2 in $S_n$ if and only if its cycle decomposition is a product of commuting 2-cycles.
\end{exercise}

\begin{solitem}
    \item [\rightimp] Let $\sigma \in S_n$ such that $\abs\sigma = 2$. Perform cycle decomposition on $\sigma$ to obtain a product of disjoint cycles. Let $(a_1\ a_2\ \ldots\ a_m)$ be one of the cycles. Since $\sigma \neq 1$, then $\sigma(a_1) = a_2$ and $\sigma^2(a_1) = a_3$. However, $\sigma^2 = 1$, so $\sigma^2(a_1) = a_1$ so that $a_1 = a_3$. Then $m \leq 2$, and the cycle is $(a_1\ a_2)$ so that any cycle in the cycle decomposition of $\sigma$ is of length 2.
    \item [\leftimp] Suppose $\sigma$'s cycle decomposition is a product of commuting 2-cycles $(a_1\ b_1)(a_2\ b_2) \ldots (a_k\ b_k)$. Since each 2-cycle has order 2, and the 2-cycles commute, then $\sigma^2 = \id$. Because $\sigma \neq \id$, then $\abs\sigma = 2$.
\end{solitem}

\begin{exercise}
    Let $p$ be a prime. Show that an element has order $p$ in $S_n$ if and only if its cycle decomposition is a product of commuting $p$-cycles. Show by an explicit example that this need not be the case if $p$ is not prime.
\end{exercise}

\begin{sol} \leavevmode
    \begin{ifandonlyif}
        \item [\rightimp] Let $\sigma \in S_n$ such that $\abs\sigma = p$. Perform cycle decomposition on $\sigma$ to obtain a product of disjoint cycles. Let $(a_1\ a_2\ \ldots\ a_m)$ be one of the cycles. Recall from \hyperref[ex1.3.12]{Exercise 1.3.12} (a) that for any $i$, $\sigma^i(a_1) = a_{1 + i}$, where $1 + i$ is replaced by its least residue mod $m$ when $1 + i > m$. Since $\sigma^p = 1$, then $\sigma^p(a_1) = a_1$, so that $a_{1 + p} = a_1$. It follows that $m \mid p$. Since $p$ is prime, then $m = 1$ or $m = p$. However, if $m = 1$, then the cycle is trivial and can be removed from the cycle decomposition. Then every cycle in the cycle decomposition of $\sigma$ has length $p$.
        \item [\leftimp] Suppose $\sigma$'s cycle decomposition is a product of commuting $p$-cycles $\pi_1, \pi_2, \ldots, \pi_k$. For any $1 \leq t < p$, then $\pi_i^t \neq 1$ for all $i$. Then $\sigma^t = \pi_1^t \pi_2^t \ldots \pi_k^t \neq 1$. However, $\sigma^p = (\pi_1^p)(\pi_2^p) \ldots (\pi_k^p) = 1$. Then $\abs\sigma = p$.
    \end{ifandonlyif}
    For a counterexample when $p$ is not prime, consider $\sigma = (1\ 2\ 3)(4\ 5)$ in $S_5$. Then $\abs\sigma = 6$, but the cycle decomposition consists of a 3-cycle and a 2-cycle.
\end{sol}

\begin{exercise}
    Prove that the order of an element in $S_n$ equals the least common multiple of the lengths of the cycles in its cycle decomposition.
\end{exercise}

\begin{sol}
    Let $\sigma \in S_n$ have the cycle decomposition
    \[\sigma = \pi_1\pi_2 \ldots \pi_k\]
    where each $\pi_i$ is a disjoint cycle of length $m_i$. Note that since the cycles are disjoint, they commute. Then
    \[\sigma^t = \pi_1^t \pi_2^t \ldots \pi_k^t\]
    Suppose $|\sigma| = m$. Then $\sigma^m = \id$. Let $x$ be in some cycle $\pi_i$. Note that for all other $\pi_j$ where $j \neq i$, then $\pi_j$ fixes $x$. Then 
    \[\sigma^m(x) = \pi_i^m(x) = x\]
    so that $\pi_i^m = \id$. It follows that $m_i \mid m$ for all $i$, so $\lcm(m_1, m_2, \ldots, m_k) \mid m$. Morover, let $l = \lcm(m_1, m_2, \ldots, m_k)$. Then
    \[\sigma^l = \pi_1^l \pi_2^l \ldots \pi_k^l = \id\]
    so that $m \mid l$. It follows that $m = l$.
\end{sol}

\begin{specialexercise}
    Show that if $n \ge m$ then the number of $m$-cycles in $S_n$ is given by
    \[\frac{n(n-1)(n-2)\dots(n-m+1)}{m}.\]
    [Count the number of ways of forming an $m$-cycle and divide by the number of representations of a particular $m$-cycle.]
\end{specialexercise}

\begin{sol}
    Note that in an $m$-cycle, there are $m$ integers to write. The first integer has $n$ choices, the second $n - 1$ choices. Continuing on, then the $m$-th integer has $n - m + 1$ choices. However, each choice of $m$ integers has $m$ equivalent representations (such as $(1\ 2\ 3\ 4)$ being equivalent to $(2\ 3\ 4\ 1)$ by shifting each integer to the left one place). Then take the product $n(n - 1)(n - 2) \ldots (n - m + 1)$ and divide by $m$.
\end{sol}

\begin{exercise}
    Show that if $n \geq 4$ then the number of permutations in $S_n$ which are the product of two disjoint 2-cycles is $n(n - 1)(n - 2)(n - 3)/8$.
\end{exercise}

\begin{sol}
    The first 2-cycle has $n(n - 1)/2$ choices, and the second 2-cycle has $(n - 2)(n - 3)/2$ choices. Moreover, there are 2 representations for the same set of 2-cycles, so multiply the above quantities and divide by 2 to obtain $n(n - 1)(n - 2)(n - 3)/8$.
\end{sol}

\begin{exercise}
    Find all numbers $n$ such that $S_5$ contains an element of order $n$.
\end{exercise}

\begin{sol}
    $S_5$ contains elements of order 1 through 5. Moreover, we can have a 2 and 3-cycle together whose lcm is 6. Then $n = 1, 2, 3, 4, 5, 6$.
\end{sol}

\begin{exercise}
    Find all numbers $n$ such that $S_7$ contains a number an element of order $n$.
\end{exercise}

\begin{sol}
    We have $n$ from 1 to 7. We can have 2 and 5-cycles and 3 and 4-cycles. Then $n = 10$ and 12.
\end{sol}

\begin{exercise} \label{ex1.3.20}
    Find a set of generators and relations for $S_3$.
\end{exercise}

\begin{sol}
    Recall that $S_3 = \{1, (1\ 2), (1\ 3), (2\ 3), (1\ 2\ 3), (1\ 3\ 2)\}$. Moreover, no element in $S_3$ has order 6 so that no element generates $S_3$. It must be that there are at least 2 generators. Put $\alpha = (1\ 2)$ and $\beta = (1\ 3)$. Then $\alpha^2 = \beta^2 = 1$. Then $|\alpha| = |\beta| = 2$, and $\alpha\beta = (1\ 3\ 2), \beta\alpha = (1\ 2\ 3)$, and $\alpha\beta\alpha = (2\ 3)$. Since $\alpha^2 = \beta^2 = 1$ is not enough to determine the order of $\alpha\beta$ (because $\alpha\beta \neq \beta\alpha$), then we must add $(\alpha\beta)^2 = 1$. Then we have the presentation $S_3 = \gen{\alpha, \beta \mid \alpha^2 = \beta^2 = (\alpha\beta)^3 = 1}$.
\end{sol}

\newpage

\subsection{Matrix Groups}

Let $F$ be a field and let $n \in \zp$.

\begin{exercise}
    Prove that $|\gl_2(\f_2)| = 6$.
\end{exercise}

\begin{sol}
    Recall that $\f_2 = \{0, 1\}$. Moreover, consider
    \[
    \begin{pmatrix}
        a & b \\
        c & d
    \end{pmatrix} \in \gl_2(\f_2)
    \]
    where $a, b, c, d \in \f_2$. The determinant is $ad - bc$, which is never 0 whenever $a, d$ are nonzero or when $b, c$ are nonzero, but not both. Then
    \[\gl_2(\f_2) = 
    \begin{pmatrix}
        1 & 0 \\
        0 & 1
    \end{pmatrix}, 
    \begin{pmatrix}
        1 & 1 \\
        0 & 1
    \end{pmatrix},
    \begin{pmatrix}
        1 & 0 \\
        1 & 1
    \end{pmatrix},
    \begin{pmatrix}
        0 & 1 \\
        1 & 0
    \end{pmatrix},
    \begin{pmatrix}
        1 & 1 \\
        1 & 0
    \end{pmatrix},
    \begin{pmatrix}
        0 & 1 \\
        1 & 1
    \end{pmatrix} \qh
    \]
\end{sol}

\begin{exercise}
    Write out all the elements of $\gl_2(\f_2)$ and compute the order of each element.
\end{exercise}

\begin{sol}
    By direct calcuation, we obtain the following:
    \[
    \begin{array}{c|c|c|c|c|c|c}
        A & 
        \begin{pmatrix}
            1 & 0 \\
            0 & 1
        \end{pmatrix} & 
        \begin{pmatrix}
            1 & 1 \\
            0 & 1
        \end{pmatrix} &
        \begin{pmatrix}
            1 & 0 \\
            1 & 1
        \end{pmatrix} &
        \begin{pmatrix}
            0 & 1 \\
            1 & 0
        \end{pmatrix} &
        \begin{pmatrix}
            1 & 1 \\
            1 & 0
        \end{pmatrix} &
        \begin{pmatrix}
            0 & 1 \\
            1 & 1
        \end{pmatrix} \\
        \hline
        |A| & 1 & 2 & 2 & 2 & 3 & 3
    \end{array} \qh
    \]
\end{sol}

\begin{exercise} \label{ex1.4.3}
    Show that $\gl_2(\f_2)$ is non-abelian.
\end{exercise}

\begin{sol}
    Consider the matrices of $\gl_2(\f_2)$:
    \[
    \begin{pmatrix}
        1 & 1 \\
        0 & 1
    \end{pmatrix}, 
    \begin{pmatrix}
        1 & 1 \\
        1 & 0
    \end{pmatrix}
    \]
    Then we have the following products:
    \[
    \begin{pmatrix}
        1 & 1 \\
        0 & 1
    \end{pmatrix}
    \begin{pmatrix}
        1 & 1 \\
        1 & 0
    \end{pmatrix} = 
    \begin{pmatrix}
        0 & 1 \\
        1 & 0
    \end{pmatrix}, \quad
    \begin{pmatrix}
        1 & 1 \\
        1 & 0
    \end{pmatrix}
    \begin{pmatrix}
        1 & 1 \\
        0 & 1
    \end{pmatrix} = 
    \begin{pmatrix}
        1 & 0 \\
        1 & 1
    \end{pmatrix}
    \]
    Since the two products are not equal, then $\gl_2(\f_2)$ is non-abelian.
\end{sol}

\begin{exercise}
    Show that if $n$ is not prime then $\intmod$ is not a field.
\end{exercise}

\begin{sol}
    Since $n$ is not prime, it must be composite. Then there exist $a, b \in \zp$ such that $1 < a, b < n$ and $n = ab$. Moreover, $(a, n) \neq 1$ since $a \mid n$. Then $a$ has no multiplicative inverse in $\intmod$, so $\intmod$ is not a field.
\end{sol}

\newpage

\begin{exercise}
    Show that $\gl_n(F)$ is a finite group if and only if $F$ has a finite number of elements.
\end{exercise}

\begin{solitem}
    \item [\rightimp] We proceed by contrapositive. Suppose $F$ has an infinite number of elements. Consider the set of matrices
    \[S = \set{\alpha I \mid \alpha \in F\unt}\]
    where $I$ is the identity matrix. It is clear that $S \subseteq \gl_n(F)$ since $\det(\alpha I) = \alpha^n \neq 0$ for all $\alpha \in F\unt$. Moreover, for $\alpha, \beta \in F\unt$ with $\alpha \neq \beta$, then $\alpha I \neq \beta I$, so that $S$ is infinite. It follows that $\gl_n(F)$ is infinite.
    \item [\leftimp] Suppose $F$ has a finite number of elements, say $q$. Then each entry of a matrix in $\gl_n(F)$ has $q$ choices, and there are $n^2$ entries in total. Then the total number of matrices with entries from $F$ is $q^{n^2}$. Since $\gl_n(F)$ is a subset of these matrices, then $\gl_n(F)$ is finite.
\end{solitem}

\begin{exercise}
    If $|F| = q$ is finite prove that $|\gl_n(F)| < q^{n^2}$.
\end{exercise}

\begin{sol}
    See the right-to-left implication in the previous exercise.
\end{sol}

\begin{exercise}
    
Let $p$ be a prime. Prove that the order of $\gl_2(\f_p)$ is $p^4 - p^3 - p^2 + p$ (do not just quote the order formula in this section). [Subtract the number of $2 \times 2$ matrices which are \textit{not} invertible from the total number of $2 \times 2$ matrices over $\f_p$. You may use the fact that a $2 \times 2$ matrix is not invertible if and only if one row is a multiple of the other.]
\end{exercise}

\begin{sol}
    Observe that the total number of $2 \times 2$ matrices over $\f_p$ is $p^4$ since each of the 4 entries has $p$ choices. Next, we count the number of non-invertible matrices. Using the given fact that a $2 \times 2$ matrix is not invertible if and only if one row is a multiple of the other, we may view a $2 \times 2$ matrix as the matrix $\begin{pmatrix} r_1 \\ r_2 \end{pmatrix}^T$ where $r_1$ and $r_2$ are the first and second rows respectively. Thus, this matrix is not invertible if and only if $r_2 = \alpha r_1$ for some $\alpha \in \f_p$. We have two cases:
    \begin{itemize}
        \item If $r_1 = \b 0_{1 \times 2}$, then $r_2$ has $p^2$ choices as it has 2 entries with $p$ choices each.
        \item If $r_1 \neq \b 0_{1 \times 2}$, then $r_1$ has $p^2 - 1$ choices (all possible rows except the zero row). For each such $r_1$, there are $p$ choices for $\alpha$, so $r_2$ has $p$ choices. Then there are $(p^2 - 1)p = p^3 - p$ such matrices.
    \end{itemize}
    Then the total number of non-invertible matrices is $p^2 + p^3 - p = p^3 + p^2 - p$. Subtracting this from the total number of matrices, we obtain $|\gl_2(\f_p)| = p^4 - (p^3 + p^2 - p) = p^4 - p^3 - p^2 + p$.
\end{sol}

\begin{exercise}
    Show that $\gl_n(F)$ is non-abelian for any $n \geq 2$ and any $F$.
\end{exercise}

\begin{sol}
    Note that for any field $F$, the field $\f_2 = \set{0, 1}$ is a subset of $F$. We proceed with the proof over $\f_2$, which will imply the result for any field $F$.

    We proceed by induction on $n$. For $n = 2$, see \hyperref[ex1.4.3]{Exercise 1.4.3}. Suppose that $\gl_n(\f_2)$ is non-abelian for some $n \geq 2$. Let $A, B \in \gl_{n + 1}(\f_2)$, and let $A_0, B_0 \in \gl_n(\f_2)$ be the top-left $n \times n$ block matrices of $A$ and $B$ respectively. Then
    \[
    AB = 
    \begin{pmatrix}
        A_0 & 0 \\
        0 & 1
    \end{pmatrix}
    \begin{pmatrix}
        B_0 & 0 \\
        0 & 1
    \end{pmatrix} = 
    \begin{pmatrix}
        A_0B_0 & 0 \\
        0 & 1
    \end{pmatrix} \neq 
    \begin{pmatrix}
        B_0A_0 & 0 \\
        0 & 1
    \end{pmatrix}
    \begin{pmatrix}
        B_0 & 0 \\
        0 & 1
    \end{pmatrix}
    \begin{pmatrix}
        A_0 & 0 \\
        0 & 1
    \end{pmatrix} = BA\]
    since $A_0B_0 \neq B_0A_0$ by the inductive hypothesis. Thus, $\gl_{n + 1}(\f_2)$ is non-abelian. By induction, $\gl_n(\f_2)$ is non-abelian for all $n \geq 2$, which implies that $\gl_n(F)$ is non-abelian for any field $F$.
\end{sol}

\begin{exercise}
    Prove that the binary operation of matrix multiplication of $2 \times 2$ matrices with real number entries is associative.
\end{exercise}

\begin{sol}
    \begin{align*}
        \left[
        \begin{pmatrix}
            a & b \\
            c & d
        \end{pmatrix}
        \begin{pmatrix}
            a' & b' \\
            c' & d'
        \end{pmatrix}
        \right]
        \begin{pmatrix}
            a'' & b'' \\
            c'' & d''
        \end{pmatrix} & = 
        \begin{pmatrix}
            aa' + bc' & ab' +bd' \\
            ca' + dc' & cb' + dd'
        \end{pmatrix}
        \begin{pmatrix}
            a'' & b'' \\
            c'' & d''
        \end{pmatrix} \\
        & = 
        \begin{pmatrix}
            (aa'+ bc')a'' + (ab'+bd')c'' & (aa' + bc')b'' + (ab' + bd')d'' \\
            (ca' + dc')a'' + (cb' + dd')c' & (ca' + dc')b'' + (cb' + dd')d''
        \end{pmatrix} \\
        & = 
        \begin{pmatrix}
            aa'a'' + bca'' + ab'c'' + bd'c'' & aa'b'' + bc'b'' + ab'd'' + bd'd'' \\
            ca'a'' + dc'a'' + cb'c' + dd'c' & ca'b'' + dc'b'' + cb'd'' + dd'd''
        \end{pmatrix} \\
        & = 
        \begin{pmatrix}
            a(a'a'' + b'c'') + b(c'a'' + d'c'') & a(a'b'' + b'd'') + b(c'b'' + d'd'') \\
            c(a'a'' + b'c'') + d(c'a'' + d'c'') & c(a'b'' + b'd'') + d(c'b'' + d'd'')
        \end{pmatrix} \\
        & = 
        \begin{pmatrix}
            a & b \\
            c & d
        \end{pmatrix}
        \begin{pmatrix}
            a'a''  + b'c'' & a'b'' + b'd'' \\
            c'a'' + d'c'' & c'b'' + d'd''
        \end{pmatrix} \\
        & = 
        \begin{pmatrix}
            a & b \\
            c & d
        \end{pmatrix}
        \left[
        \begin{pmatrix}
            a' & b' \\
            c' & d'
        \end{pmatrix}
        \begin{pmatrix}
            a'' & b'' \\
            c'' & d''
        \end{pmatrix}
        \right] \qh
    \end{align*}
\end{sol}

\begin{exercise}
    Let
    $$G = \left\{\left.
    \begin{pmatrix}
        a & b \\
        0 & c
    \end{pmatrix}\,\right|\, a, b, c \in \r, a \neq 0, c \neq 0\right\}$$
    \begin{subproblems}
        \item Compute the product of $
        \begin{pmatrix}
            a_1 & b_1 \\
            0 & c_1
        \end{pmatrix}$ and $
        \begin{pmatrix}
            a_2 & b_2 \\
            0 & c_2
        \end{pmatrix}$ to show that $G$ is closed under matrix multiplication.
        \item Find the matrix inverse of $
        \begin{pmatrix}
            a & b \\
            0 & c
        \end{pmatrix}$ and deduce that $G$ is closed under inverses.
        \item Deduce that $G$ is a subgroup of $\gl_2(\r)$ (cf. \hyperref[ex1.1.26]{Exercise 1.1.26}).
        \item Prove that the set of elements of $G$ whose two diagonal entries are equal (i.e., $a = c$) is also a subgroup of $\gl_2(\r)$.
    \end{subproblems}
\end{exercise}

\begin{solalph}
    \item 
    $$\begin{pmatrix}
        a_1 & b_1 \\
        0 & c_1
    \end{pmatrix}
    \begin{pmatrix}
        a_2 & b_2 \\
        0 & c_2
    \end{pmatrix} = 
    \begin{pmatrix}
        a_1a_2 & a_1b_2 + b_1c_2 \\
        0 & c_1c_2
    \end{pmatrix}$$
    Since $a_i \neq 0$ and $c_i \neq 0$, then $a_1a_2 \neq 0$ and $c_1c_2 \neq 0$ so that $G$ is closed under matrix multiplication.
    \item The determinant is $ac$. We then have the inverse
    $$
    \begin{pmatrix}
        a & b \\
        0 & c
    \end{pmatrix}\inv = 
    \frac{1}{ac}
    \begin{pmatrix}
        c & -b \\
        0 & a
    \end{pmatrix} = 
    \begin{pmatrix}
        1/a & -b/ac \\
        0 & 1/c
    \end{pmatrix}$$
    \item Since $G$ is closed under inverses and matrix multiplication, the result follows.
    \item Let $H = \{A \in G \mid a = c\}$. Then
    $$
    \begin{pmatrix}
        a_1 & b_1 \\
        0 & a_1
    \end{pmatrix}
    \begin{pmatrix}
        a_2 & b_2 \\
        0 & a_2
    \end{pmatrix} = 
    \begin{pmatrix}
        a_1a_2 & a_1b_2 + b_1a_2 \\
        0 & a_1a_2
    \end{pmatrix} \in H$$
    and
    $$
    \begin{pmatrix}
        a & b \\
        0 & a
    \end{pmatrix}\inv = 
    \frac{1}{a^2}
    \begin{pmatrix}
        a & -b \\
        0 & a 
    \end{pmatrix} = 
    \begin{pmatrix}
        1/a & -b/a^2 \\
        0 & 1/a
    \end{pmatrix} \in H$$
    Then $H$ is closed under inverses and matrix multiplication, hence it is a subgroup of $\gl_2(\r)$.
\end{solalph}

\begin{exercise} \label{ex1.4.11}
    Let
    \[H(F) = \left\{\left.
    \begin{pmatrix}
        1 & a & b \\
        0 & 1 & c \\
        0 & 0 & 1
    \end{pmatrix}\,\right|\, a, b, c \in F\right\}\]
    called the \textit{Heisenberg group} over $F$. Let
    \[X = 
    \begin{pmatrix}
        1 & a & b \\
        0 & 1 & c \\
        0 & 0 & 1
    \end{pmatrix} \text{ and }Y = 
    \begin{pmatrix}
        1 & d & e \\
        0 & 1 & f \\
        0 & 0 & 1
    \end{pmatrix}\]
    be elements of $H(F)$.
    \begin{subproblems}
        \item Compute the matrix product $XY$ and deduce that $H(F)$ is closed under matrix multiplication. Exhibit explicit matrices such that $XY \neq YX$ (so that $H(F)$ is always non-abelian).
        \item Find an explicit formula for the matrix inverse $X\inv$ and deduce that $H(F)$ is closed under inverses.
        \item Prove the associative law for $H(F)$ and deduce that $H(F)$ is a group of order $|F|^3$. (Do not assume that matrix multiplication is associative.)
        \item Find the order of each element of the finite group $H(\intmod[2])$.
        \item Prove that every nonidentity of the group $H(\r)$ has infinite order.
    \end{subproblems}
\end{exercise}


\begin{solalph}
    \item Calculating the product, we have
    \[XY = 
    \begin{pmatrix}
        1 & d + a & e + af + b \\
        0 & 1 & f + c \\
        0 & 0 & 1
    \end{pmatrix} \in H(F)\]
    Moreover, we exhibit matrices $A, B$ such that $AB \neq BA$:
    \[AB =
    \begin{pmatrix}
        1 & 1 & 0 \\
        0 & 1 & 0 \\
        0 & 0 & 1
    \end{pmatrix}
    \begin{pmatrix}
        1 & 0 & 0 \\
        0 & 1 & 1 \\
        0 & 0 & 1
    \end{pmatrix} = 
    \begin{pmatrix}
        1 & 1 & 1 \\ 
        0 & 1 & 1 \\
        0 & 0 & 1
    \end{pmatrix} \neq
    \begin{pmatrix}
        1 & 1 & 0 \\
        0 & 1 & 1 \\
        0 & 0 & 1
    \end{pmatrix} = 
    \begin{pmatrix}
        1 & 0 & 0 \\
        0 & 1 & 1 \\
        0 & 0 & 1
    \end{pmatrix}
    \begin{pmatrix}
        1 & 1 & 0 \\
        0 & 1 & 0 \\
        0 & 0 & 1
    \end{pmatrix}
    = BA\]
    \item Using augmented matrices, we have
    \[\left(\begin{array}{ccc|ccc}
        1 & a & b & 1 & 0 & 0 \\
        0 & 1 & c & 0 & 1 & 0 \\
        0 & 0 & 1 & 0 & 0 & 1
    \end{array}\right) \implies
    \left(\begin{array}{ccc|ccc}
        1 & 0 & 0 & 1 & -a & ac - b \\
        0 & 1 & 0 & 0 & 1 & -c \\
        0 & 0 & 1 & 0 & 0 & 1
    \end{array}\right)\]
    Calling the right matrix $Z$, it is easy to see that $XZ = ZX = I_3$, hence $Z  = X\inv$.
    \item It is clear that each of $a, b, c$ has $|F|$ choices, so that there are $|F|^3$ elements in $H(F)$. Moreover:
    \begin{align*}
        \left[\begin{pmatrix}
            1 & a & b \\
            0 & 1 & c \\
            0 & 0 & 1
        \end{pmatrix}
        \begin{pmatrix}
            1 & d & e \\
            0 & 1 & f \\
            0 & 0 & 1
        \end{pmatrix}\right]
        \begin{pmatrix}
            1 & g & h \\
            0 & 1 & i \\
            0 & 0 & 1
        \end{pmatrix} & = 
        \begin{pmatrix}
            1 & a + d & af + b + e \\
            0 & 1 & c + f \\
            0 & 0 & 1
        \end{pmatrix}
        \begin{pmatrix}
            1 & g & h \\
            0 & 1 & i \\
            0 & 0 & 1
        \end{pmatrix} \\
        & = 
        \begin{pmatrix}
            1 & a + d + g & ai + ad + af + b + e + h \\
            0 & 1 & c + f + i \\
            0 & 0 & 1
        \end{pmatrix} \\
        & = 
        \begin{pmatrix}
            1 & a & b \\
            0 & 1 & c \\
            0 & 0 & 1
        \end{pmatrix}
        \begin{pmatrix}
            1 & d + g & di + e + h \\
            0 & 1 & f + i \\
            0 & 0 & 1
        \end{pmatrix} \\
        & = 
        \begin{pmatrix}
            1 & a & b \\
            0 & 1 & c \\
            0 & 0 & 1
        \end{pmatrix}
        \left[\begin{pmatrix}
            1 & d & e \\
            0 & 1 & f \\
            0 & 0 & 1
        \end{pmatrix}
        \begin{pmatrix}
            1 & g & h \\
            0 & 1 & i \\
            0 & 0 & 1
        \end{pmatrix}\right]
    \end{align*}
    \item Denote a matrix in $H(\intmod[2])$ as $(a, b, c)$, where $a, b, c \in \intmod[2]$. We may view matrix multiplication in $H(\intmod[2])$ as
    \[(a, b, c)(d, e, f) = (a + d, af + b + e, c + f)\]
    Moreover, we identify the identity matrix as $(0, 0, 0)$. Lastly, the elements of $H(\intmod[2])$ are
    \[
    (0, 0, 0), (1, 0, 0), (0, 1, 0), (0, 0, 1), (1, 1, 0), (1, 0, 1), (0, 1, 1), (1, 1, 1)\]
    Noting that $(a, b, c)^2 = (2a, ac + 2b, 2c) = (0, ac, 0)$ in $\intmod[2]$, we may consider two cases instead of computing the order of each element directly:
    \begin{itemize}
        \item If $ac = 0$, then at least one of $a$ or $c$ is 0, while $b$ varies. These are the matrices
        \[(0, 0, 0), (1, 0, 0), (0, 1, 0), (0, 0, 1), (1, 1, 0), (1, 0, 1), (0, 1, 1)\]
        Except for the identity matrix which has order 1, squaring any of these matrices results in $(0, 0, 0)$ so that they all have order 2.
        \item If $ac = 1$, then both $a$ and $c$ are 1, while $b$ varies. These are the matrices
        \[(1, 0, 1), (1, 1, 1)\]
        The square of either matrix is $(0, 1, 0)$, which has order 2. Then the matrices have order 4.
    \end{itemize}
    \item Utilizing the same formulation as in part (d), let $A = (a, b, c) \in H(\r)$. We prove that
    $$(a, b, c)^n = (na, n(n - 1)ac/2 + nb, nc)$$
    To that end, it is clear that $n = 1$ holds true. Suppose it holds for some $n$. Then for $n + 1$, we have
    \begin{align*}
        (a, b, c)^{n + 1} & = (a, b, c)^n(a, b, c) \\
        & = (na, n(n - 1)ac/2 + nb, nc)(a, b, c) \\
        & = (na + a, (na)c + n(n - 1)ac/2 + nb + b, nc + c) \\
        & = ((n + 1)a, n(n + 1)ac/2 + (n + 1)b, (n + 1)c)
    \end{align*}
    so that the relationship is true by induction. Since $A \neq I$, then at least one of $a, b, c$ must be nonzero. Then for any $n \in \zp$, none of the integers $na, n(n - 1)ac/2 + nb,$ or $nc$ are 0. Then no power of $(a, b, c)$ results in $(0, 0, 0)$, hence no nonidentity element of $H(\r)$ has any finite order. Therefore, every nonidentity element of $H(\r)$ has infinite order.
\end{solalph}

\newpage

\subsection{The Quaternion Group}

\begin{exercise}
    Compute the order of each of the elements in $Q_8$.
\end{exercise}

\begin{sol}
    \[\begin{array}{c|c|c|c|c|c|c|c|c}
        a & 1 & -1 & i & -i & j & -j & k & -k \\
        \hline
        |a| & 1 & 2 & 4 & 4 & 4 & 4 & 4 & 4
    \end{array}\]
\end{sol}

\begin{exercise}
    Write out the group tables for $S_3, D_8$, and $Q_8$.
\end{exercise}

\begin{sol}
    $S_3$:
    \[\begin{array}{c|c|c|c|c|c|c}
        \circ & 1 & (1\ 2) & (1\ 3) & (2\ 3) & (1\ 2\ 3) & (1\ 3\ 2) \\
        \hline
        1 & 1 & (1\ 2) & (1\ 3) & (2\ 3) & (1\ 2\ 3) & (1\ 3\ 2) \\
        (1\ 2) & (1\ 2) & 1 & (1\ 2\ 3) & (1\ 3\ 2) & (1\ 3) & (2\ 3) \\
        (1\ 3) & (1\ 3) & (1\ 3\ 2) & 1 & (1\ 2\ 3) & (2\ 3) & (1\ 2) \\
        (2\ 3) & (2\ 3) & (1\ 2\ 3) & (1\ 3\ 2) & 1 & (1\ 2) & (1\ 3) \\ 
        (1\ 2\ 3) & (1\ 2\ 3) & (2\ 3) & (1\ 2) & (1\ 3) & (1\ 3\ 2) & 1 \\
        (1\ 3\ 2) & (1\ 3\ 2) & (1\ 3) & (2\ 3) & (1\ 2) & 1 & (1\ 2\ 3)
    \end{array}\]
    $D_8$:
    \[\begin{array}{c|c|c|c|c|c|c|c|c}
        \circ & 1 & r & r^2 & r^3 & s & sr & sr^2 & sr^3 \\
        \hline
        1 & 1 & r & r^2 & r^3 & s & sr & sr^2 & sr^3 \\
        r & r & r^2 & r^3 & 1 & sr^3 & s & sr & sr^2 \\
        r^2 & r^2 & r^3 & 1 & r & sr^2 & sr^3 & s & sr \\
        r^3 & r^3 & 1 & r & r^2 & sr & sr^2 & sr^3 & s \\
        s & s & sr & sr^2 & sr^3 & 1 & r & r^2 & r^3 \\
        sr & sr & sr^2 & sr^3 & s & r^3 & 1 & r & r^2 \\
        sr^2 & sr^2 & sr^3 & s & sr & r^2 & r^3 & 1 & r \\
        sr^3 & sr^3 & s & sr & sr^2 & r & r^2 & r^3 & 1
    \end{array} \qh\]
    $Q_8$:
    \[\begin{array}{c|c|c|c|c|c|c|c|c}
        \cdot & 1 & -1 & i & -i & j & -j & k & -k \\
        \hline
        1 & 1 & -1 & i & -i & j & -j & k & -k \\
        -1 & -1 & 1 & -i & i & -j & j & -k & k \\
        i & i & -i & -1 & 1 & k & -k & j & -j \\
        -i & -i & i & 1 & -1 & -k & k & -j & j \\
        j & j & -j & -k & k & -1 & 1 & i & -i \\
        -j & -j & j & k & -k & 1 & -1 & -i & i \\
        k & k & -k & j & -j & -i & i & -1 & 1 \\
        -k & -k & k & -j & j & i & -i & 1 & -1
    \end{array} \qh\]
\end{sol}

\begin{exercise} \label{ex1.5.3}
    Find a set of generators and relations for $Q_8$.
\end{exercise}

\begin{sol}
    An extra relation is necessary to intertwine $i, j, k$ together, so a presentation for $Q_8$ is:
    \[Q_8 = \gen{-1, i, j, k \mid i^2 = j^2 = k^2 = ijk = -1} \qh\]
\end{sol}

\newpage

\subsection{Homomorphisms and Isomorphisms}

\begin{exercise}
    Let $\phi : G \to H$ be a homomorphism.
    \begin{subproblems}
        \item Prove that $\phi(x^n) = \phi(x)^n$ for all $n \in \zp$.
        \item Do part (a) for $n = - 1$ and deduce that $\phi(x^n) = \phi(x)^n$ for all $n \in \z$.
    \end{subproblems}
\end{exercise}

\begin{solalph}
    \item We proceed by induction on $n$. For $n = 1$, the result holds. Suppose it holds for some $n \in \zp$. Then for $n + 1$, we have
    \[\phi(x^{n + 1}) = \phi(x^n x) = \phi(x^n)\phi(x) = \phi(x)^n \phi(x) = \phi(x)^{n + 1}\]
    so that the result holds by induction.
    \item We first show that the result holds for $n = 0$, i.e., identities map to identities. Let $1_G$ and $1_H$ be the identities of $G$ and $H$ respectively. Then
    \[\phi(1_G) = \phi(1_G \cdot 1_G) = \phi(1_G)\phi(1_G)\]
    so that multiplying both sides on the left by $\phi(1_G)\inv$ yields $\phi(1_G) = 1_H$. Let $x \in G$. Then for $n = -1$, we have
    \[1_H = \phi(1_G) = \phi(xx\inv) = \phi(x)\phi(x\inv)\]
    Left multiplying both sides by $\phi(x)\inv$ yields $\phi(x\inv) = \phi(x)\inv$. Finally, for $n < -1$, we have
    \[\phi(x^n) = \phi(x\inv)^{-n} = \phi(x)^{-n} = \phi(x)^n\]
    so that the result holds for all $n \in \z$.
\end{solalph}

\begin{specialexercise} \label{ex1.6.2}
    If $\phi : G \to H$ is an isomorphism, prove that $|\phi(x)| = |x|$ for all $x \in G$. Deduce that any two isomorphic groups have the same number of elements of order $n$ for each $n \in \z^+$. Is the result true if $\phi$ is only assumed to be a homomorphism?
\end{specialexercise}

\begin{sol}
    Let $x \in G$. Let $\abs x = m$ and $\abs{\phi(x)} = n$. Then
    \[1_H = \phi(1_G) = \phi(x^m) = \phi(x)^m\]
    so that $n \leq m$. Similarly,
    \[1_G = \phi\inv(1_H) = \phi\inv(\phi(x)^n) = x^n\]
    so that $m \leq n$. It follows that $m = n$, i.e., $|\phi(x)| = |x|$.
    
    Assume, by way of contradiction and without loss of generality that $x$ has infinite order in $G$ but $\phi(x)$ has finite order in $H$. Then by the above result, $|\phi(x)| = |x|$ implies that $x$ must also have finite order, a contradiction. Thus, both $x$ and $\phi(x)$ both must have either finite or infinite order. Moreover, for any $n \in \z^+$, the number of elements of order $n$ in $G$ must equal the number of elements of order $n$ in $H$ since $\phi$ is bijective and preserves order.

    The result does not always hold if $\phi$ is only a homomorphism. Let $G$ be any group, and let $H$ be the trivial group. Consider the mapping $\phi : G \to H$ defined by $\phi(g) = 1_H$ for all $g \in G$. Clearly, $\phi$ is a homomorphism, but all elements of $H$ have order 1, while $G$ may have elements of various orders.
\end{sol}

\newpage

\begin{exercise}
    If $\phi : G \to H$ is an isomorphism, prove that $G$ is abelian if and only if $H$ is abelian. If $\phi : G \to H$ is a homomorphism, what additional conditions on $\phi$ (if any) are sufficient to ensure that if $G$ is abelian, then so is $H$?
\end{exercise}

\begin{sol}
    Since $\phi$ is bijective, then $\phi\inv$ exists and is also an isomorphism. 
    \begin{ifandonlyif}
        \item [\rightimp] Suppose $G$ is abelian. Then for any $h_1, h_2 \in H$, there exists $g_1, g_2 \in G$ such that $g_1 = \phi\inv(h_1)$ and $g_2 = \phi\inv(h_2)$. Then
        \[h_1h_2 = \phi(g_1)\phi(g_2) = \phi(g_1g_2) = \phi(g_2g_1) = \phi(g_2)\phi(g_1) = h_2h_1\]
        so that $H$ is abelian.
        \item [\leftimp] Suppose $H$ is abelian. Then for any $g_1, g_2 \in G$, there exists $h_1, h_2 \in H$ such that $h_1 = \phi(g_1)$ and $h_2 = \phi(g_2)$. Then
        \[g_1g_2 = \phi\inv(h_1)\phi\inv(h_2) = \phi\inv(h_1h_2) = \phi\inv(h_2h_1) = \phi\inv(h_2)\phi\inv(h_1) = g_2g_1\]
        so that $G$ is abelian.
    \end{ifandonlyif}
    To see what conditions are sufficient for the homomorphism to imply that its codomain is abelian if its domain is abelian, let $\phi : G \to H$ be a homomorphism. Then for $h_1, h_2 \in H$, then see by the previous proof that if there exists $g_1, g_2 \in G$ such that $h_1 = \phi(g_1)$ and $h_2 = \phi(g_2)$, then $H$ is abelian whenever $G$ is abelian. Thus, it suffices to ensure that such $g_1$ and $g_2$ always exist, which is true if $\phi$ is surjective. Hence, we may deduce that to ensure that if $G$ is abelian, then so is $H$, it suffices to require that $\phi$ is surjective.
\end{sol}

\begin{exercise}
    Prove that the multiplicative groups $\r - \{0\}$ and $\mathbb{C} - \{0\}$ are not isomorphic.
\end{exercise}

\begin{sol}
    Note that $i \in C \nz$ has order 4, but there is no element in $\r \nz$ that has order 4.
\end{sol}

\begin{exercise}
    Prove that the additive groups $\r$ and $\q$ are not isomorphic.
\end{exercise}

\begin{sol}
    Since $\r$ is uncountable while $\q$ is countable, there cannot exist a bijection between the two groups, hence they cannot be isomorphic.
\end{sol}

\begin{exercise}
    Prove that the additive groups $\z$ and $\q$ are not isomorphic.
\end{exercise}

\begin{sol}
    For every $n \in \z$, there exists no $x \in \z$ such that $nx = 0$. However, for every $q \in \q$ of the form $1/k$ for some $k \in \z$ with $k \neq 0$, we have $kq = 0$. Thus, there cannot exist a bijection between the two groups that preserves addition, hence they cannot be isomorphic.
\end{sol}

\begin{exercise}
    Prove that $D_8$ and $Q_8$ are not isomorphic.
\end{exercise}

\begin{sol}
    $D_8$ has only 1 element of order 4, while $Q_8$ has 3 elements of order 4.
\end{sol}

\begin{exercise}
    Prove that if $n \neq m$, then $S_n$ and $S_m$ are not isomorphic.
\end{exercise}

\begin{sol}
    If $n \neq m$, then $n! \neq m!$ so that $|S_n| \neq |S_m|$.
\end{sol}

\begin{exercise}
    Prove that $D_{24}$ and $S_4$ are not isomorphic.
\end{exercise}

\begin{sol}
    $D_{24}$ has elements of order 12 (e.g. $r$) but $S_4$ has no elements of order 12, since every element of $S_4$ is either a 2-cycle, 3-cycle, 4-cycle, or a combination of 2-cycles.
\end{sol}

\begin{exercise} \label{ex1.6.10}
    Fill in the details of the proof that the symmetric groups $S_\Delta$ and $S_\Omega$ are isomorphic if $|\Delta| = |\Omega|$ as follows: let $\theta : \Delta \to \Omega$ be a bijection. Define
    \[
    \phi : S_\Delta \to S_\Omega \quad \text{by} \quad \phi(\sigma) = \theta \circ \sigma \circ \theta^{-1} \quad \text{for all } \sigma \in S_\Delta
    \]
    and prove the following:
    \begin{subproblems}
        \item Prove that $\phi$ is well-defined, that is, if $\sigma$ is a permutation of $\Delta$ then $\theta \circ \sigma \circ \theta^{-1}$ is a permutation of $\Omega$.
        \item Prove that $\phi$ is a bijection from $S_\Delta$ onto $S_\Omega$ by finding a two-sided inverse for $\phi$.
        \item Prove that $\phi$ is a homomorphism, that is, $\phi(\sigma \circ \tau) = \phi(\sigma) \circ \phi(\tau)$.
    \end{subproblems}
    Note the similarity to the change of basis or similarity transformations for matrices (we shall see the connections between these later in the text).
\end{exercise}

\begin{solalph}
    \item We show that a composition of bijections is a bijection. Let $A, B, C$ be nonempty sets, and let $f : A \to B$ and $g : B \to C$ be bijections. We show that $g \circ f : A \to C$ is a bijection. For injectivity, let $a_1, a_2 \in A$ such that $(g \circ f)(a_1) = (g \circ f)(a_2)$. Then $g(f(a_1)) = g(f(a_2))$, and since $g$ is injective, then $f(a_1) = f(a_2)$. Since $f$ is also injective, then $a_1 = a_2$. For surjectivity, let $c \in C$. Since $g$ is surjective, there exists some $b \in B$ such that $g(b) = c$. Since $f$ is surjective, there exists some $a \in A$ such that $f(a) = b$. Then $(g \circ f)(a) = g(f(a)) = g(b) = c$. Hence, $g \circ f$ is a bijection.
    
    Now, by assumption, $\theta : \Delta \to \Omega$ is a bijection, and since $\sigma : \Delta \to \Delta$ is also a bijection, then by the above result, $\theta \circ \sigma : \Delta \to \Omega$ is a bijection. Finally, since $\theta\inv : \Omega \to \Delta$ is a bijection, then by the above result again, $\theta \circ \sigma \circ \theta\inv : \Omega \to \Omega$ is a bijection. Hence, $\phi$ is well-defined.
    \item Consider the mapping 
    \[\psi : S_\Omega \to S_\Delta\quad \text{by} \quad \psi(\tau) = \theta\inv \circ \tau \circ \theta \quad \text{for all } \tau \in S_\Omega\]
    By the above discussion, $\psi$ is well-defined. For any $\sigma \in S_\Delta$, we have
    \[\psi(\phi(\sigma)) = \theta\inv \circ (\theta \circ \sigma \circ \theta\inv) \circ \theta = \id_{S_\Delta} \circ \sigma \circ \id_{S_\Delta} = \sigma\]
    and for any $\tau \in S_\Omega$, we have
    \[\phi(\psi(\tau)) = \theta \circ (\theta\inv \circ \tau \circ \theta) \circ \theta\inv = \id_{S_\Omega} \circ \tau \circ \id_{S_\Omega} = \tau\]
    so that $\psi$ is a two-sided inverse for $\phi$. Hence, $\phi$ is a bijection from $S_\Delta$ onto $S_\Omega$.
    \item Let $\sigma, \tau \in S_\Delta$. Then
    \[\phi(\sigma \circ \tau) = \theta \circ (\sigma \circ \tau) \circ \theta\inv = (\theta \circ \sigma) \circ (\tau \circ \theta\inv) = (\theta \circ \sigma \circ \theta\inv) \circ (\theta \circ \tau \circ \theta\inv) = \phi(\sigma) \circ \phi(\tau)\]
    so that $\phi$ is a homomorphism. By the above, $\phi$ is an isomorphism, and hence $S_\Delta \cong S_\Omega$.
\end{solalph}

\newpage

\begin{exercise}
    Let $A$ and $B$ be groups. Prove that $A \times B \cong B \times A$.
\end{exercise}

\begin{sol}
    Consider the mapping
    \[\phi : A \times B \to B \times A \quad \text{by} \quad (a, b) \mapsto (b, a)\]
    for all $a \in A$ and $b \in B$. For any $(a_1, b_1), (a_2, b_2) \in A \times B$, we have
    \[\phi((a_1, b_1)(a_2, b_2)) = \phi((a_1a_2, b_1b_2)) = (b_1b_2, a_1a_2) = (b_1, a_1)(b_2, a_2) = \phi(a_1, b_1)\phi(a_2, b_2)\]
    so that $\phi$ is a homomorphism. Moreover, suppose $(a_1, b_1), (a_2, b_2) \in A \times B$ are such that $\phi(a_1, b_1) = \phi(a_2, b_2)$. Then $(b_1, a_1) = (b_2, a_2)$ so that $a_1 = a_2$ and $b_1 = b_2$. Hence, $\phi$ is injective. Finally, for any $(b, a) \in B \times A$, then $\phi(a, b) = (b, a)$ so that $\phi$ is surjective. Therefore, $\phi$ is an isomorphism and $A \times B \cong B \times A$.
\end{sol}

\begin{exercise}
    Let $A$, $B$, and $C$ be groups and let $G = A \times B$ and $H = B \times C$. Prove that $G \times C \cong A \times H$.
\end{exercise}

\begin{sol}
    Consider the mapping
    \[\phi : G \times C \to A \times H \quad \text{by} \quad ((a, b), c) \mapsto (a, (b, c))\]
    for all $a \in A$, $b \in B$, and $c \in C$. The proof is similar to the previous exercise, hence we omit the details to conclude that $\phi$ is an isomorphism and $G \times C \cong A \times H$.
\end{sol}

\begin{specialexercise} \label{ex1.6.13}
    Let $G$ and $H$ be groups and let $\phi : G \to H$ be a homomorphism. Prove that the image of $\phi$, $\phi(G)$, is a subgroup of $H$ (cf. \hyperref[ex1.1.26]{Exercise 1.1.26}). Prove that if $\phi$ is injective then $G \cong \phi(G)$.
\end{specialexercise}

\begin{sol}
    We first show that $\phi(G) = \{\phi(g) \mid g \in G\}$ is a subgroup of $H$. Note that $1_H = \phi(1_G) \in \phi(G)$ so that it is nonempty. To show closure, let $h_1, h_2 \in \phi(G)$. Then there exists $g_1, g_2 \in G$ such that $h_1 = \phi(g_1)$ and $h_2 = \phi(g_2)$. Then
    \[h_1h_2 = \phi(g_1)\phi(g_2) = \phi(g_1g_2) \in \phi(G)\]
    so that $\phi(G)$ is closed under the group operation of $H$. Finally, for any $h \in \phi(G)$, there exists some $g \in G$ such that $h = \phi(g)$. Then
    \[h\inv = \phi(g)\inv = \phi(g\inv) \in \phi(G)\]
    so that $\phi(G)$ is closed under taking inverses. Therefore, $\phi(G)$ is a subgroup of $H$.

    Now suppose $\phi$ is injective. Consider the mapping
    \[\psi : G \to \phi(G) \quad \text{by} \quad g \mapsto \phi(g)\]
    for all $g \in G$. For any $g_1, g_2 \in G$, we have
    \[\psi(g_1g_2) = \phi(g_1g_2) = \phi(g_1)\phi(g_2) = \psi(g_1)\psi(g_2)\]
    so that $\psi$ is a homomorphism. Moreover, suppose $g_1, g_2 \in G$ are such that $\psi(g_1) = \psi(g_2)$. Then $\phi(g_1) = \phi(g_2)$, and since $\phi$ is injective, then $g_1 = g_2$. Hence, $\psi$ is injective. Finally, for any $h \in \phi(G)$, there exists some $g \in G$ such that $h = \phi(g) = \psi(g)$ so that $\psi$ is surjective. Therefore, $\psi$ is an isomorphism and $G \cong \phi(G)$.
\end{sol}

\newpage

\begin{specialexercise} \label{ex1.6.14}
    Let $G$ and $H$ be groups and let $\phi : G \to H$ be a homomorphism. Define the \emph{kernel} of $\phi$ to be $\{ g \in G \mid \phi(g) = 1_H \}$ (so the kernel is the set of elements in $G$ which map to the identity of $H$, i.e., is the fiber over the identity of $H$). Prove that the kernel of $\phi$ is a subgroup of $G$. Prove that $\phi$ is injective if and only if the kernel of $\phi$ is the identity subgroup of $G$.
\end{specialexercise}

\begin{sol}
    Let $\ker\phi$ denote the kernel of $\phi$. Note that $1_G \in \ker\phi$ since $\phi(1_G) = 1_H$, so that $\ker\phi$ is nonempty. To show closure, let $g_1, g_2 \in \ker\phi$. Then
    \[\phi(g_1g_2) = \phi(g_1)\phi(g_2) = 1_H 1_H = 1_H\]
    so that $g_1g_2 \in \ker\phi$. Finally, for any $g \in \ker\phi$, we have
    \[\phi(g\inv) = \phi(g)\inv = 1_H\inv = 1_H\]
    so that $g\inv \in \ker\phi$. Therefore, $\ker\phi$ is a subgroup of $G$.
    \begin{ifandonlyif}
        \item [\rightimp] Suppose $\phi$ is injective. Let $g \in \ker\phi$. Then $\phi(g) = 1_H = \phi(1_G)$, and since $\phi$ is injective, then $g = 1_G$. Hence, $\ker\phi = \{1_G\}$.
        \item [\leftimp] Suppose $\ker\phi = \{1_G\}$. Let $g_1, g_2 \in G$ be such that $\phi(g_1) = \phi(g_2)$. Then
        \[1_H = \phi(g_1)\phi(g_2)\inv = \phi(g_1g_2\inv)\]
        so that $g_1g_2\inv \in \ker\phi$. By assumption, then $g_1g_2\inv = 1_G$, and hence $g_1 = g_2$. Therefore, $\phi$ is injective.
    \end{ifandonlyif}
\end{sol}

\begin{exercise}
    Define a map $\pi : \r^2 \to \r$ by $\pi((x,y)) = x$. Prove that $\pi$ is a homomorphism and find the kernel of $\pi$.
\end{exercise}

\begin{sol}
    Let $(x_1, y_1), (x_2, y_2) \in \r^2$. Then
    \[\pi((x_1, y_1) + (x_2, y_2)) = \pi((x_1 + x_2, y_1 + y_2)) = x_1 + x_2 = \pi((x_1, y_1)) + \pi((x_2, y_2))\]
    so that $\pi$ is a homomorphism. Moreover, if $(x, y) \in \ker(\pi)$, then $\pi((x, y)) = x = 0$. Hence,
    \[\ker(\pi) = \{(0, y) \mid y \in \r\} \qh\]
\end{sol}

\begin{exercise}
    Let $A$ and $B$ be groups and let $G$ be their direct product, $A \times B$. Prove that the maps $\pi_1 : G \to A$ and $\pi_2 : G \to B$ defined by $\pi_1((a,b)) = a$ and $\pi_2((a,b)) = b$ are homomorphisms and find their kernels.
\end{exercise}

\begin{sol}
    For any $(a_1, b_1), (a_2, b_2) \in G$, we have
    \[\pi_1((a_1, b_1)(a_2, b_2)) = \pi_1((a_1a_2, b_1b_2)) = a_1a_2 = \pi_1((a_1, b_1))\pi_1((a_2, b_2))\]
    and
    \[\pi_2((a_1, b_1)(a_2, b_2)) = \pi_2((a_1a_2, b_1b_2)) = b_1b_2 = \pi_2((a_1, b_1))\pi_2((a_2, b_2))\]
    so that both $\pi_1$ and $\pi_2$ are homomorphisms. Moreover, if $(a, b) \in \ker(\pi_1)$, then $\pi_1((a, b)) = a = 1_A$. Also, if $(a, b) \in \ker(\pi_2)$, then $\pi_2((a, b)) = b = 1_B$. Hence,
    \[\ker(\pi_1) = \{(1_A, b) \mid b \in B\} \quad \text{and} \quad \ker(\pi_2) = \{(a, 1_B) \mid a \in A\} \qh\]
\end{sol}

\newpage

\begin{exercise} \label{ex1.6.17}
    Let $G$ be any group. Prove that the map from $G$ to itself defined by $g \mapsto g^{-1}$ is a homomorphism if and only if $G$ is abelian.
\end{exercise}

\begin{solitem}
    \item [\rightimp] Suppose $\phi$ is a homomorphism. Then for $g, h \in G$, we have
    \[(gh)\inv = \phi(gh) = \phi(g)\phi(h) = g\inv h\inv\]
    Since inverses are unique, then $hg = gh$ so that $G$ is abelian.
    \item [\leftimp] Suppose $G$ is abelian. Then for $g, h \in G$, we have
    \[\phi(gh) = (gh)\inv = h\inv g\inv = g\inv h\inv = \phi(g)\phi(h)\]
    so that $\phi$ is a homomorphism.
\end{solitem}

\begin{exercise}
    Let $G$ be any group. Prove that the map from $G$ to itself defined by $g \mapsto g^2$ is a homomorphism if and only if $G$ is abelian.
\end{exercise}

\begin{solitem}
    \item [\rightimp] Suppose $\phi$ is a homomorphism. Then for $g, h \in G$, we have
    \[(gh)^2 = \phi(gh) = \phi(g)\phi(h) = g^2 h^2\]
    Then $ghgh = g^2h^2$, so that multiplying both sides on the left by $g\inv$ and on the right by $h\inv$ yields $hg = gh$ and $G$ is abelian.
    \item [\leftimp] Suppose $G$ is abelian. Then for $g, h \in G$, we have
    \[\phi(gh) = (gh)^2 = g^2 h^2 = \phi(g)\phi(h)\]
    so that $\phi$ is a homomorphism.
\end{solitem}

\begin{exercise}
    Let $G = \{ z \in \mathbb{C} \mid z^n = 1 \text{ for some } n \in \z^+ \}$. Prove that for any fixed integer $k > 1$ the map from $G$ to itself defined by $z \mapsto z^k$ is a surjective homomorphism but is not an isomorphism.
\end{exercise}

\begin{sol}
    Recall that the polar form of a complex number is $z = r(\cos\theta + i\sin\theta) = re^{i\theta}$, where $r$ is the modulus and $\theta = \arg(z)$ is the argument. By definition, every $z \in G$ has modulus $r = 1$ so that every element of $G$ is of the form $e^{i\theta}$. Moreover, $z \in G$ implies that $z^n = 1$ for some $n \in \zp$. Using the polar representation of $1 = e^{2\pi im}$ for some $m \in \z$, then we have $e^{in\theta} = e^{2\pi im}$. This implies that $\theta = 2\pi m/n$, and we can reformulate $G$ as the following:
    \[G = \{e^{2\pi im/n} \in \c \mid m \in \z, n \in \zp\}\]
    We show that $\phi$ is a homomorphism. For any $z, w \in G$, we have
    \[\phi(zw) = (zw)^k = z^k w^k = \phi(z)\phi(w)\]
    so that $\phi$ is a homomorphism. To show surjectivity, let $w \in G$. Then $w = e^{2\pi im/n}$ for some $m \in \z$ and $n \in \zp$. Since $z = e^{2\pi im/nk} \in G$ because $nk \in \zp$, then
    \[z^k = (e^{2\pi im/nk})^k = e^{2\pi im/n} = w\]
    so that $\phi(z) = w$ and $\phi$ is surjective. Lastly, we show that $\phi$ is not an isomorphism by finding its kernel. Note that every $z \in \ker\phi$ satisfies $z^k = 1$. Then $z = e^{2\pi im/k}$ for some $m \in \z$. Since there are $k$ distinct values of $m$ modulo $k$, then $\ker\phi$ contains $k$ distinct elements, namely
    \[\ker\phi = \{e^{2\pi im/k} \mid m = 0, 1, \ldots, k-1\}\]
    Since $k > 1$, then $\ker\phi$ contains more than just the identity element, so that $\phi$ is not injective by \hyperref[ex1.6.14]{Exercise 1.6.14}. Hence, $\phi$ is not an isomorphism.
\end{sol}

\begin{specialexercise}
    Let $G$ be a group and let $\aut(G)$ be the set of all isomorphisms from $G$ onto $G$. Prove that $\aut(G)$ is a group under function composition (called the \emph{automorphism group} of $G$ and the elements of $\aut(G)$ are called \emph{automorphisms} of $G$).
\end{specialexercise}

\begin{sol}
    Since function composition is associative, then it is also associative on $\aut(G)$. By the proof in \hyperref[ex1.6.10]{Exercise 1.6.10}, the composition of two bijections is a bijection. We now show that the composition of two isomorphisms is an isomorphism. Let $\phi, \psi \in \aut(G)$. Then for any $g_1, g_2 \in G$, we have
    \[(\phi \circ \psi)(g_1g_2) = \phi(\psi(g_1g_2)) = \phi(\psi(g_1)\psi(g_2)) = \phi(\psi(g_1))\phi(\psi(g_2)) = (\phi \circ \psi)(g_1)(\phi \circ \psi)(g_2)\]
    so that $\phi \circ \psi$ is a homomorphism. Since $\phi \circ \psi$ is also a bijection, then it is an isomorphism and $\phi \circ \psi \in \aut(G)$. The identity map $\id_G : G \to G$ defined by $\id_G(g) = g$ for all $g \in G$ is an isomorphism, so that $\id_G \in \aut(G)$ and is the identity element of $\aut(G)$. Finally, for any $\phi \in \aut(G)$, since $\phi$ is a bijection, then $\phi\inv$ exists. We show that $\phi\inv$ is also an isomorphism. For any $g_1, g_2 \in G$, we have
    \[\phi\inv(g_1g_2) = \phi\inv(\phi(\phi\inv(g_1g_2))) = \phi\inv(\phi(\phi\inv(g_1)\phi\inv(g_2))) = \phi\inv(g_1)\phi\inv(g_2)\]
    so that $\phi\inv$ is a homomorphism. Since $\phi\inv$ is also a bijection, then it is an isomorphism and $\phi\inv \in \aut(G)$. Therefore, $\aut(G)$ is a group under function composition.
\end{sol}

\begin{exercise} \label{ex1.6.21}
    Prove that for each fixed nonzero $k \in \q$ the map from $\q$ to itself defined by $q \mapsto kq$ is an automorphism of $\q$.
\end{exercise}

\begin{sol}
    Let $\phi : \q \to \q$ be defined by $\phi(q) = kq$ for all $q \in \q$. For any $q_1, q_2 \in \q$, we have
    \[\phi(q_1 + q_2) = k(q_1 + q_2) = kq_1 + kq_2 = \phi(q_1) + \phi(q_2)\]
    so that $\phi$ is a homomorphism. Moreover, suppose $q_1, q_2 \in \q$ are such that $\phi(q_1) = \phi(q_2)$. Then $kq_1 = kq_2$, and since $k \neq 0$, then $q_1 = q_2$. Hence, $\phi$ is injective. Finally, for any $q \in \q$, we have $\phi(q/k) = k(q/k) = q$ so that $\phi$ is surjective. Therefore, $\phi$ is an automorphism of $\q$.
\end{sol}

\begin{exercise}
    Let $A$ be an abelian group and fix some $k \in \z$. Prove that the map $a \mapsto a^k$ is a homomorphism from $A$ to itself. If $k = -1$ prove that this homomorphism is an isomorphism (i.e., is an automorphism of $A$).
\end{exercise}

\begin{sol}
    Let $\phi : A \to A$ be defined by $\phi(a) = a^k$ for all $a \in A$. For any $a_1, a_2 \in A$, we have
    \[\phi(a_1a_2) = (a_1a_2)^k = a_1^k a_2^k = \phi(a_1)\phi(a_2)\]
    so that $\phi$ is a homomorphism. For $k = -1$, see \hyperref[ex1.6.17]{Exercise 1.6.17} to conclude that $\phi \in \aut(A)$.
\end{sol}

\newpage

\begin{exercise}
    Let $G$ be a finite group which possesses an automorphism $\sigma$ such that $\sigma(g) = g$ if and only if $g = 1$. If $\sigma^2$ is the identity map from $G$ to $G$, prove that $G$ is abelian (such an automorphism $\sigma$ is called \emph{fixed point free} of order 2). [Show that every element of $G$ can be written in the form $x^{-1}\sigma(x)$ and apply $\sigma$ to such an expression.]
\end{exercise}

\begin{sol}
    To show that every $g \in G$ is of the form $x\inv \sigma(x)$ for some $x \in G$, we must show that the map
    \[\phi : G \to G \quad \text{by} \quad x \mapsto x\inv \sigma(x)\]
    is injective. Supppose $x, y \in G$ are such that $\phi(x) = \phi(y)$. Then
    \[x\inv \sigma(x) = y\inv \sigma(y)\]
    so that $yx\inv = \sigma(yx\inv)$. Since $\sigma$ is fixed point free, then $yx\inv = 1$ and $x = y$. Hence, $\phi$ is injective. Since $G$ is finite, then $\phi$ is also surjective, and every element of $G$ may be written in the form $x\inv \sigma(x)$ for some $x \in G$. 

    To show that $G$ is abelian, observe for any $g \in G$ that
    \[\sigma(g) = \sigma(x\inv \sigma(x)) = \sigma(x\inv) \sigma^2(x) = \sigma(x)\inv x = (x\inv \sigma(x))\inv = g\inv\]
    where $\sigma^2 = \id_G$ was used. Now, for any $g, h \in G$, we have
    \[\sigma(gh) = (gh)\inv = h\inv g\inv = \sigma(h) \sigma(g) = \sigma(hg)\]
    Since $\sigma$ is injective, then $gh = hg$ and $G$ is abelian.
\end{sol}

\begin{exercise}
    Let $G$ be a finite group and let $x$ and $y$ be distinct elements of order 2 in $G$ that generate $G$. Prove that $G \cong D_{2n}$, where $n = |xy|$.
\end{exercise}

\begin{sol}
    To show that $G$ is isomorphic to $D_{2n}$, we first need to produce an element $t \in G$ of order $n$ along with an element $x \in G$ of order 2 such that $xt = t\inv x$. The clear choice is to let $t = xy$ and keep $x$ as is. Note that $|t| = |xy| = n$ by assumption, and $|x| = 2$ by assumption. Moreover,
    \[xt = x(xy) = (xx)y = y = (xy)\inv x = t\inv x\]
    so that the elements $x, t \in G$ satisfy the relations of $D_{2n}$. Moreover, observe that $y = xt$ so that $t$ and $x$ generate $G$. Following this, it is easy to see that every element of $G$ can be written in the form $x^i t^j$ for $i \in \{0, 1\}$ and $0 \leq j < n$. For $x = 0$, we have the elements in the cyclic subgroup generated by $t$, and for $x = 1$, we have the elements $\set{x, xt, xt^2, \ldots, xt^{n-1}}$. Since $|t| = n$, then these are all distinct elements of $G$. Hence, $|G| = 2n$.

    Since the relations of $D_{2n}$ are satisfied by $x, t \in G$, $G$ is generated by $x$ and $t$, and $|G| = |D_{2n}| = 2n$, it follows that the map
    \[\phi : D_{2n} \to G \quad \text{by} \quad r \mapsto t, s \mapsto x\]
    extends to an isomorphism from $D_{2n}$ onto $G$. Therefore, $G \cong D_{2n}$.
\end{sol}

\begin{exercise}
    Let $n \in \z^+$, let $r$ and $s$ be the usual generators of $D_{2n}$ and let $\theta = 2\pi / n$.
    \begin{subproblems}
        \item Prove that the matrix 
        \[
        \begin{pmatrix}
        \cos \theta & -\sin \theta \\
        \sin \theta & \cos \theta
        \end{pmatrix}
        \]
        is the matrix of the linear transformation which rotates the $x,y$-plane about the origin in a counterclockwise direction by $\theta$ radians.
        \item Prove that the map $\phi : D_{2n} \to \gl_2(\r)$ defined on generators by
        \[
        \phi(r) = 
        \begin{pmatrix}
            \cos \theta & -\sin \theta \\
            \sin \theta & \cos \theta
        \end{pmatrix}
        \quad \text{and} \quad
        \phi(s) = 
        \begin{pmatrix}
            0 & 1 \\
            1 & 0
        \end{pmatrix}
        \]
        extends to a homomorphism of $D_{2n}$ into $\gl_2(\r)$.
        \item Prove that the homomorphism $\phi$ defined above is injective.
    \end{subproblems}
\end{exercise}

\begin{solalph}
    \item Let $\r^2$ be generated by the standard basis vectors $\b e_1 = (1 \quad 0)^T$ and $\b e_2 = (0 \quad 1)^T$. Let $T$ denote the linear transformation which rotates the $x,y$-plane about the origin in a counterclockwise direction by $\theta$ radians. Then
    \[T(\b e_1) = 
    \begin{pmatrix}
        \cos \theta \\
        \sin \theta
    \end{pmatrix}, \quad \text{and} \quad
    T(\b e_2) = 
    \begin{pmatrix}
        -\sin \theta \\
        \cos \theta
    \end{pmatrix}\]
    So the matrix of $T$ with respect to the standard basis is
    \[\begin{pmatrix}
        \cos \theta & -\sin \theta \\
        \sin \theta & \cos \theta
    \end{pmatrix}\]
    \item It is easy to show by induction that for any positive $k$, then $\phi(r)^k$ is given by the following matrix:
    \[\phi(r)^k = \begin{pmatrix}
        \cos (k\theta) & -\sin (k\theta) \\
        \sin (k\theta) & \cos (k\theta)
    \end{pmatrix}\]
    so that $\phi(r)^n = I$. It is fairly straightforward to see that the set of vectors $\set{T(\b e_1), T(\b e_2)}$ is orthonormal, so $\phi(r)$ is orthogonal. Hence, $\phi(r)\inv = \phi(r)^T$. Moreover, we can calculate that
    \[\phi(s)^2 =
    \begin{pmatrix}
        0 & 1 \\
        1 & 0
    \end{pmatrix}^2 =
    \begin{pmatrix}
        1 & 0 \\
        0 & 1
    \end{pmatrix}
    \]
    so that $\phi(s)^2 = I$. Lastly, we calculate that
    \[\phi(s)\phi(r) = 
    \begin{pmatrix}
        0 & 1 \\
        1 & 0
    \end{pmatrix}
    \begin{pmatrix}
        \cos \theta & -\sin \theta \\
        \sin \theta & \cos \theta
    \end{pmatrix} = 
    \begin{pmatrix}
        \sin \theta & \cos \theta \\
        \cos \theta & -\sin \theta
    \end{pmatrix} = 
    \begin{pmatrix}
        \cos \theta & \sin \theta \\
        -\sin \theta & \cos \theta
    \end{pmatrix}
    \begin{pmatrix}
        0 & 1 \\
        1 & 0
    \end{pmatrix} = \phi(r)\inv \phi(s)\]
    so that $\phi(s)\phi(r) = \phi(r)\inv \phi(s)$. Since these relations match those of $D_{2n}$, then $\phi$ extends to a homomorphism from $D_{2n}$ to $\gl_2(\r)$.
    \item To show that $\phi$ is injective, we show that $\ker\phi$ is trivial. Let $g \in \ker\phi$.
    \begin{itemize}
        \item If $g = r^k$ for some $0 \leq k < n$, then $\phi(r^k) \neq I$, since $\phi(r)^k$ is a rotation matrix and only $\phi(r)^n = I$.
        \item If $g = sr^k$ for some $0 \leq k < n$, then $\phi(sr^k) = \phi(s)\phi(r)^k$. Note that $\phi(s)$ has determinant $-1$ while $\phi(r)^k$ has determinant $1$ since it is a rotation matrix. Hence, $\phi(sr^k)$ has determinant $-1$ and cannot be the identity matrix.
    \end{itemize}
    Therefore, the only element in $\ker\phi$ is the identity element of $D_{2n}$, so that $\phi$ is injective.
\end{solalph}

\begin{exercise}
    Let $i$ and $j$ be the generators of $Q_8$ described in Section 5. Prove that the map $\phi$ from $Q_8$ to $\gl_2(\mathbb{C})$ defined on generators by
    \[\phi(i) = 
    \begin{pmatrix}
        \sqrt{-1} & 0 \\
        0 & -\sqrt{-1}
    \end{pmatrix}
    \quad \text{and} \quad
    \phi(j) = 
    \begin{pmatrix}
        0 & -1 \\
        1 & 0
    \end{pmatrix}
    \]
    extends to a homomorphism. Prove that $\phi$ is injective.
\end{exercise}

\begin{sol}
    It is easy to see that
    \[\phi(i)^2 = 
    \begin{pmatrix}
        \sqrt{-1} & 0 \\
        0 & -\sqrt{-1}
    \end{pmatrix}^2 = 
    \begin{pmatrix}
        -1 & 0 \\
        0 & -1
    \end{pmatrix} = -I
    \]
    and
    \[\phi(j)^2 =
    \begin{pmatrix}
        0 & -1 \\
        1 & 0
    \end{pmatrix}^2 = 
    \begin{pmatrix}
        -1 & 0 \\
        0 & -1
    \end{pmatrix} = -I
    \]
    Also note that
    \[\phi(i)\phi(j) = 
    \begin{pmatrix}
        \sqrt{-1} & 0 \\
        0 & -\sqrt{-1}
    \end{pmatrix}
    \begin{pmatrix}
        0 & -1 \\
        1 & 0
    \end{pmatrix} = 
    \begin{pmatrix}
        0 & -\sqrt{-1} \\
        -\sqrt{-1} & 0
    \end{pmatrix}\]
    So we may set $\phi(i)\phi(j) = \phi(k)$. We can calculate that $\phi(k)^2 = -I$. It then follows that
    \[\phi(i)^2 = \phi(j)^2 = \phi(k)^2 = \phi(i)\phi(j)\phi(k) = \phi(-1)\]
    The elements $\phi(i), \phi(j)$, and $\phi(k)$ satisfy the relations of $Q_8$ as described in \hyperref[ex1.5.3]{Exercise 1.5.3} so that $\phi$ extends to a homomorphism from $Q_8$ to $\gl_2(\c)$. Moreover, $\phi$ is injective since the only element that maps to the identity matrix is $1 \in Q_8$. Therefore, $\phi$ is an injective homomorphism.
\end{sol}

\newpage

\subsection{Group Actions}

\begin{exercise}
    Let $F$ be a field. Show that the multiplicative group of nonzero elements of $F$ (denoted by $F^\times$) acts on the set $F$ by $g\cdot a = g a$, where $g\in F^\times$, $a\in F$ and $ga$ is the usual product in $F$ of the two field elements (state clearly which axioms in the definition of a field are used). 
\end{exercise}

\begin{sol}
    Let $g, h \in F\unt$ and $f \in F$. Then
    \[g \cdot (h \cdot f) = g \cdot (hf) = g(hf) = (gh)f = (gh) \cdot f\]
    where the associativity of multiplication in $F$ was used to conclude that $g(hf) = (gh)f$. Moreover, $1 \cdot f = 1f = f$ where $1$ is the multiplicative identity in $F$. Therefore, the group action axioms are satisfied and $F\unt$ acts on $F$.
\end{sol}

\begin{exercise}
    Show that the additive group $\mathbb{Z}$ acts on itself by $z\cdot a = z + a$ for all $z,a\in\mathbb{Z}$.
\end{exercise}

\begin{sol}
    Let $z, w \in \z$ and $a \in \z$. Then
    \[z \cdot (w \cdot a) = z \cdot (w + a) = z + (w + a) = (z + w) + a = (z + w) \cdot a\]
    Moreover, $0 \cdot a = 0 + a = a$ so that the group action axioms are satisfied and $\z$ acts on itself.
\end{sol}

\begin{exercise}
    Show that the additive group $\mathbb{R}$ acts on the $x,y$ plane $\mathbb{R}\times\mathbb{R}$ by $r\cdot (x,y) = (x + r, y)$.
\end{exercise}

\begin{sol}
    Let $r, s \in \r$ and $(x, y) \in \r^2$. Then
    \[r \cdot (s \cdot (x, y)) = r \cdot (x + s, y) = (x + s + r, y) = (x + (r + s), y) = (r + s) \cdot (x, y)\]
    Moreover, $0 \cdot (x, y) = (x + 0, y) = (x, y)$ so that the group action axioms are satisfied and $\r$ acts on $\r^2$.
\end{sol}

\begin{specialexercise}
    Let $G$ be a group acting on a set $A$ and fix some $a\in A$. Show that the following sets are subgroups of $G$ (cf. \hyperref[ex1.1.26]{Exercise 1.1.26}):
    \begin{subproblems}
        \item the kernel of the action,
        \item $\{g\in G \mid g a = a\}$ — this subgroup is called the \emph{stabilizer} of $a$ in $G$.
    \end{subproblems}
\end{specialexercise}

\begin{solalph}
    \item Suppose $g, h$ are in the kernel of the action. Then for any $a \in A$, we have
    \[gh \cdot a = g \cdot (h \cdot a) = g \cdot a = a\]
    so that $gh$ is in the kernel of the action. Also, for any $g$ in the kernel of the action and any $a \in A$, we have
    \[g\inv \cdot a = g\inv \cdot (g \cdot a) = (g\inv g) \cdot a = 1 \cdot a = a\]
    so that $g\inv$ is in the kernel of the action. Therefore, the kernel of the action is a subgroup of $G$.
    \item Let $G_a$ denote the stabilizer of $a$ in $G$. Suppose $g, h \in G_a$. Then
    \[gh \cdot a = g \cdot (h \cdot a) = g \cdot a = a\]
    so that $gh \in G_a$. Also, for any $g \in G_a$, we have
    \[g\inv \cdot a = g\inv \cdot (g \cdot a) = (g\inv g) \cdot a = 1 \cdot a = a\]
    so that $g\inv \in G_a$. Therefore, $G_a$ is a subgroup of $G$.
\end{solalph}

\begin{exercise}
    Prove that the kernel of an action of the group $G$ on the set $A$ is the same as the kernel of the corresponding permutation representation $G\to S_A$ (cf. \hyperref[ex1.6.14]{Exercise 1.6.14}).
\end{exercise}

\begin{sol}
    Let $K$ denote the kernel of the action, and let $K'$ denote $\ker\phi$, where $\phi : G \to S_A$ is the permutation representation corresponding to the action. Suppose $g \in K$. Then $g \cdot a = a$ for every $a \in A$, hence $\phi(g)$ is the identity permutation in $S_A$ and $g \in K'$. Conversely, suppose $g \in K'$. Then $\phi(g)$ is the identity permutation in $S_A$, so that for every $a \in A$, we have $\phi(g)(a) = a$. But $\phi(g)(a) = g \cdot a$ by definition of the permutation representation, so that $g \cdot a = a$ for every $a \in A$ and $g \in K$. Therefore, $K = K'$.
\end{sol}

\begin{specialexercise}
    Prove that a group $G$ acts faithfully on a set $A$ if and only if the kernel of the action is the set consisting only of the identity.
\end{specialexercise}

\begin{solitem}
    \item [\rightimp] Suppose $G$ acts faithfully on $A$. Let $g$ be in the kernel of the action. Then for every $a \in A$, we have $g \cdot a = a$. Since the action is faithful, then $g$ must be the identity element of $G$. Hence, the kernel of the action is the set consisting only of the identity.
    \item [\leftimp] Suppose the kernel of the action is the set consisting only of the identity. Let $g, h \in G$ be such that $g \cdot a = h \cdot a$ for every $a \in A$. Then
    \[h\inv g \cdot a = h\inv \cdot (g \cdot a) = h\inv \cdot (h \cdot a) = (h\inv h) \cdot a = 1 \cdot a = a\]
    so that $h\inv g$ is in the kernel of the action. By assumption, then $h\inv g$ is the identity element of $G$, and $g = h$. Therefore, the action is faithful.
\end{solitem}

\begin{exercise}
    Prove that in Example 2 in this section the action is faithful.
\end{exercise}

\begin{sol}
    Let $V$ be a vector space over a field $F$. Let $F\unt$ act on $V$ by $r \cdot v = rv$ for all $r \in F\unt$ and $v \in V$. Suppose $r \in F\unt$ is in the kernel of the action. Then for every $v \in V$, we have $r \cdot v = v$, so that $rv = v$. Since $V$ is a vector space, then it contains the nonzero vector $v = 1$. Hence, $r(1) = 1$ so that $r = 1$. Therefore, the kernel of the action is the set consisting only of the identity, and by the previous exercise, the action is faithful.
\end{sol}

\begin{exercise}
    Let $A$ be a nonempty set and let $k$ be a positive integer with $k\le |A|$. The symmetric group $S_A$ acts on the set $B$ consisting of all subsets of $A$ of cardinality $k$ by
    \[\sigma(\{a_1, \dots, a_k\}) = \{\sigma(a_1), \dots, \sigma(a_k)\}.\]
    \begin{subproblems}
        \item Prove that this is a group action.
        \item Describe explicitly how the elements $(1\ 2)$ and $(1\ 2\ 3)$ act on the six $2$-element subsets of $\{1,2,3,4\}$.
    \end{subproblems}
\end{exercise}

\begin{solalph}
    \item Let $\sigma, \tau \in S_A$. Then for any cardinality $k$ subset $\{x_1, \ldots, x_k\}$ of $A$, we have
    \begin{align*}
        \sigma \cdot (\tau \cdot \{x_1, \ldots, x_k\}) & = \sigma \cdot \{\tau(x_1), \ldots, \tau(x_k)\} \\
        & = \{\sigma(\tau(x_1)), \ldots, \sigma(\tau(x_k))\} \\
        & = (\sigma \circ \tau) \cdot \{x_1, \ldots, x_k\}
    \end{align*}
    Moreover, $1 \cdot \{x_1, \ldots, x_k\} = \{1(x_1), \ldots, 1(x_k)\} = \{x_1, \ldots, x_k\}$. Then it is a group action.
    \item
    \[
    \begin{array}{c|c|c}
        A & (1\ 2) \cdot A & (1\ 2\ 3) \cdot A \\
        \hline
        \{1, 2\} & \{2, 1\} & \{2, 3\} \\
        \{1, 3\} & \{2, 3\} & \{2, 1\} \\
        \{1, 4\} & \{2, 4\} & \{2, 4\} \\
        \{2, 3\} & \{1, 3\} & \{3, 1\} \\
        \{2, 4\} & \{1, 4\} & \{3, 4\} \\
        \{3, 4\} & \{3, 4\} & \{1, 4\}
    \end{array} \qh
    \]
\end{solalph}

\begin{exercise}
    Do both parts of the preceding exercise with ``ordered $k$-tuples'' in place of ``$k$-element subsets,'' where the action on $k$-tuples is defined as above but with set braces replaced by parentheses (note that, for example, the $2$-tuples $(1,2)$ and $(2,1)$ are different even though the sets $\{1,2\}$ and $\{2,1\}$ are the same, so the sets being acted upon are different).
\end{exercise}

\begin{sol}
    The work is similar as above, except there are 12 2-tuples as $(1, 2) \neq (2, 1)$.
\end{sol}

\begin{exercise}
    With reference to the preceding two exercises determine:
    \begin{subproblems}
        \item for which values of $k$ the action of $S_n$ on $k$-element subsets is faithful, and
        \item for which values of $k$ the action of $S_n$ on ordered $k$-tuples is faithful.
    \end{subproblems}
\end{exercise}

\begin{solalph}
    \item If $k = \abs A$, then any subset $S$ of $A$ with cardinality $k$ is equal to $A$. Then any $\sigma \in S_n$ will fix $S$, so the action is not faithful.
    
    Let $\sigma \in S_n$ be a non-identity permutation. Then there exists some $a \in A$ such that $\sigma(a) \neq a$. If $k < \abs A$, then we can choose a $k$-element subset $S$ of $A$ such that $a \in S$ and $\sigma(a) \notin S$. Then $\sigma(S) \neq S$, hence no non-identity permutation fixes every $k$-element subset of $A$. Therefore, the action is faithful for all $1 \leq k < \abs A$.
    \item Let $\sigma \in S_n$ be a non-identity permutation. Then there exists some $a \in A$ such that $\sigma(a) \neq a$. Consider the ordered $k$-tuple $T = (a, x_2, x_3, \ldots, x_k)$ where $x_2, x_3, \ldots, x_k$ are distinct elements of $A$ different from $a$ and $\sigma(a)$. Then $\sigma(T) = (\sigma(a), \sigma(x_2), \sigma(x_3), \ldots, \sigma(x_k)) \neq T$ because the first element is different. Hence, no non-identity permutation fixes every ordered $k$-tuple of elements of $A$. Therefore, the action is faithful for all $1 \leq k \leq \abs A$.
\end{solalph}

\begin{exercise}
    Write out the cycle decomposition of the eight permutations in $S_4$ corresponding to the elements of $D_8$ given by the action of $D_8$ on the vertices of a square (where the vertices of the square are labeled as in Section 2).
\end{exercise}

\begin{sol}
    Let $\phi : D_8 \to S_4$ be the permutation representation of the action of $D_8$ on the vertices of a square $\{1, 2, 3, 4\}$, where the vertices are labeled in order around the square starting in the top left corner, then going clockwise. Then the cycle decompositions are as follows:
    \begin{align*}
        \phi(1) & = 1 & \phi(r) & = (1\ 2\ 3\ 4) \\
        \phi(r^2) & = (1\ 3)(2\ 4) & \phi(r^3) & = (1\ 4\ 3\ 2) \\
        \phi(s) & = (2\ 4) & \phi(sr) & = (1\ 4)(2\ 3) \\
        \phi(sr^2) & = (1\ 3) & \phi(sr^3) & = (1\ 2)(3\ 4) \qh
    \end{align*}
\end{sol}

\newpage

\begin{specialexercise} \label{ex1.7.12}
    Assume $n$ is an even positive integer and show that $D_{2n}$ acts on the set consisting of pairs of opposite vertices of a regular $n$-gon. Find the kernel of this action (label vertices as usual).
\end{specialexercise}

\begin{sol}
    Let the vertices of the regular $n$-gon be labeled $1, 2, \ldots, n$ in order around the polygon. Let $P_k$ denote the pair of opposite vertices, i.e.,
    \[P_k = \set{k, k + n/2} \quad \text{defined for all $1 \leq k \leq n/2$}\]
    Let $P$ denote the set of all such pairs of opposite vertices. Note that $\abs P = n/2$. Define a mapping from $D_{2n} \times P \to P$ by
    \[g \cdot P_i = g(P_i) = P_j\]
    where $P_j$ is the pair of opposite vertices containing the images of the vertices in $P_i$ under the action of $g$ on the vertices of the $n$-gon. It is straightforward to verify that this mapping satisfies the group action axioms, since any rotation or reflection of the $n$-gon will map pairs of opposite vertices to other pairs of opposite vertices. Hence, $D_{2n}$ acts on $P$.

    To find the kernel of this action, let $g \in D_{2n}$ be in the kernel. Then for every pair of opposite vertices $P_i \in P$, we have $g \cdot P_i = P_i$. In particular, consider the pair $P_1 = \set{1, 1 + n/2}$. Then $g$ must map vertex $1$ to either vertex $1$ or vertex $1 + n/2$. If $g$ maps vertex $1$ to vertex $1 + n/2$, then it must also map vertex $1 + n/2$ to vertex $1$, which is only possible if $g$ is a rotation by $180^\circ = r^{n/2}$. On the other hand, if $g$ maps vertex $1$ to vertex $1$, then $g$ is the identity element. Moreover, no reflection can be in the kernel:
    \begin{itemize}
        \item If the reflection is about an axis through vertex $k$ and $k + n/2$, then the vertex $k + 1$ is mapped to vertex $k - 1$, which in general is not equal to either $k + 1$ or $k + 1 + n/2$.
        \item If the reflection is about an axis through the midpoints of the edges between vertices $k$ and $k + 1$ and between vertices $k + n/2$ and $k + n/2 + 1$, then vertex $k$ is mapped to vertex $k + 1$, which in general is not equal to either $k$ or $k + n/2$.
    \end{itemize}
    Therefore, the kernel of this action is $\set{1, r^{n/2}}$. In particular, the kernel for $n = 2$ is the whole group $D_4$ since there is only one pair of opposite vertices.
\end{sol}

\begin{exercise}
    Find the kernel of the left regular action.
\end{exercise}

\begin{sol}
    If $g$ is in the kernel of the left regular action of $G$ on itself, then $g \cdot 1 = 1$ in particular. Then $g1 = 1$, implying that $g = 1$. Then the kernel of the left regular action is trivial.
\end{sol}

\begin{exercise}
    Let $G$ be a group and let $A = G$. Show that if $G$ is non-abelian then the maps defined by  
    \[ g \cdot a = ag \quad \text{for all } g,a\in G \]
    do not satisfy the axioms of a (left) group action of $G$ on itself.
\end{exercise}

\begin{sol}
    Since $G$ is non-abelian, there exists $g, h \in G$ such that $gh \neq hg$. Assume, by way of contradiction, that the above maps do satisfy the axioms of a left group action. Then for any $a \in A$, we have
    \[g \cdot (h \cdot a) = g \cdot (ah) = (ah)g = a(hg) \]
    On the other hand, $(gh) \cdot a = a(gh)$. Since $gh \neq hg$, then $a(hg) \neq a(gh)$ for some $a \in A$. This contradicts the assumption that the above maps satisfy the axioms of a left group action. Therefore, the maps do not satisfy the axioms of a left group action.
\end{sol}

\begin{exercise}
    Let $G$ be any group and let $A = G$. Show that the maps defined by  
    \[ g \cdot a = ag^{-1} \quad \text{for all } g,a\in G \]
    do satisfy the axioms of a (left) group action of $G$ on itself.
\end{exercise}

\begin{sol}
    Let $g, h \in G$. Then for any $a \in A$, we ahve
    \[g \cdot (h \cdot a) = g \cdot (ah\inv) = a(h\inv g\inv) = a(gh)\inv = (gh) \cdot a\]
    Moreover, $1 \cdot a = a1\inv = a1 = a$. Then the maps define a left group action.
\end{sol}

\begin{specialexercise}
    Let $G$ be any group and let $A = G$. Show that the maps defined by  
    \[ g \cdot a = gag^{-1} \quad \text{for all } g,a\in G \]
    do satisfy the axioms of a (left) group action (this action of $G$ on itself is called \emph{conjugation}).
\end{specialexercise}

\begin{sol}
    Let $g, h \in G$. Then for any $a \in A$, we have
    \[g \cdot (h \cdot a) = g \cdot (hah\inv) = g(hah\inv)g\inv = (gh)a(gh)\inv = (gh) \cdot a\]
    Moreover, $1 \cdot a = 1a1\inv = a$. Then the maps define a left group action.
\end{sol}

\begin{exercise} \label{ex1.7.17}
    Let $G$ be a group and let $G$ act on itself by left conjugation, so each $g\in G$ maps $G$ to $G$ by
    \[ x \mapsto gxg^{-1}. \]
    For fixed $g\in G$, prove that conjugation by $g$ is an isomorphism from $G$ onto itself (i.e.\ an automorphism of $G$). Deduce that $x$ and $gxg^{-1}$ have the same order for all $x\in G$ and that for any subset $A$ of $G$, $|A| = |gAg^{-1}|$ (here $gAg^{-1} = \{gag^{-1} \mid a\in A\}$).
\end{exercise}

\begin{sol}
    We prove that the associated permutation $\phi : G \to G$ defined by $\phi(x) = gxg\inv$ is an isomorphism. Let $x, y \in G$. Then
    \[\phi(xy) = g(xy)g\inv = (gxg\inv)(gyg\inv) = \phi(x)\phi(y)\]
    so that $\phi$ is a homomorphism. To show that $\phi$ is bijective, we show that it is injective. Suppose $\phi(x) = \phi(y)$ for some $x, y \in G$. Then
    \[gxg\inv = gyg\inv\]
    implying that $x = y$. Surjectivity is clear from conjugation by $g\inv$, so $\phi$ is a bijection. Therefore, $\phi$ is an isomorphism from $G$ onto itself.

    From \hyperref[ex1.6.2]{Exercise 1.6.2}, we know that isomorphisms preserve order, so $\abs x = \abs{gxg\inv}$ for all $x \in G$. If we consider the restriction of $\phi$ to a subset $A$ of $G$, then $\phi|_A : A \to gAg\inv$ remains a bijection so that $\abs A = \abs{gAg\inv}$.
\end{sol}

\newpage

\begin{specialexercise} \label{ex1.7.18}
    Let $H$ be a group acting on a set $A$. Prove that the relation $\sim$ on $A$ defined by  
    \[
    a \sim b \quad \text{if and only if} \quad a = hb \text{ for some } h\in H
    \]
    is an equivalence relation. (For each $x\in A$ the equivalence class of $x$ under $\sim$ is called the \textit{orbit} of $x$ under the action of $H$. The orbits under the action of $H$ partition the set $A$.)
\end{specialexercise}

\begin{sol}
    We verify that $\sim$ is an equivalence relation by checking the three properties:
    \begin{itemize}
        \item Reflexivity: For any $a \in A$, we have $a = 1 \cdot a$ where $1$ is the identity element of $H$. Hence, $a \sim a$.
        \item Symmetry: Suppose $a, b \in A$ such that $a \sim b$. Then there exists some $h \in H$ such that $a = hb$. Multiplying both sides by $h\inv$, we have $h\inv a = b$, so that $b = h\inv a$. Since $h\inv \in H$, then $b \sim a$.
        \item Transitivity: Suppose $a, b, c \in A$ such that $a \sim b$ and $b \sim c$. Then there exist some $h_1, h_2 \in H$ such that $a = h_1 b$ and $b = h_2 c$. Substituting the second equation into the first, we have
        \[a = h_1 (h_2 c) = (h_1 h_2) c\]
        Since $h_1 h_2 \in H$, then $a \sim c$.
    \end{itemize}
    Therefore, $\sim$ is an equivalence relation on $A$.
\end{sol}

\begin{specialexercise} \label{ex1.7.19}
    Let $H$ be a subgroup of the finite group $G$ and let $H$ act on $G$ (here $A = G$) by left multiplication. Let $x\in G$ and let $\mathcal{O}$ be the orbit of $x$ under this action of $H$. Prove that the map  
    \[
    H \to \mathcal{O}, \qquad h \mapsto hx
    \]
    is a bijection (hence all orbits have cardinality $|H|$). From this and the preceding exercise deduce \textit{Lagrange's Theorem}:  
    \[\textit{If $G$ is a finite group and $H$ is a subgroup of $G$ then $|H|$ divides $|G|$}\]
\end{specialexercise}

\begin{sol}
    Let $\phi : H \to \oo$ be the map defined by $\phi(h) = hx$ for all $h \in H$. We show that $\phi$ is a bijection. To show injectivity, suppose $\phi(h_1) = \phi(h_2)$ for some $h_1, h_2 \in H$. Then
    \[h_1 x = h_2 x\]
    implying that $h_1 = h_2$. To show surjectivity, let $y \in \oo$. Then by definition of the orbit, there exists some $h \in H$ such that $y = hx$. Hence, $\phi(h) = y$. Therefore, $\phi$ is a bijection and $\abs \oo = \abs H$.

    To deduce Lagrange's Theorem, note by the previous exercise that the orbits of the action of $H$ on $G$ partition $G$. Since each orbit has cardinality $\abs H$, then $\abs G$ is a sum of multiples of $\abs H$. Therefore, $\abs H$ divides $\abs G$.
\end{sol}

\begin{exercise} \label{ex1.7.20}
    Show that the group of rigid motions of a tetrahedron is isomorphic to a subgroup of $S_4$.
\end{exercise}

\begin{sol}
    Let $G$ be the group of rigit motions of a tetrahedron with vertices labeled $1, 2, 3, 4$. Then each $\alpha \in G$ sends some vertex to another vertex so that $G$ acts on the set of vertices $A = \{1, 2, 3, 4\}$.

    Since $G$ acts on $A$, this gives rise to a homomorphism 
    \[\phi : G \to S_4 \quad \text{where} \quad \phi(\alpha)(i) = \alpha(i) \quad \text{for all} \quad i \in A.\]
    To see that $\phi$ is injective, suppose $\phi(\alpha)$ is the identity permutation in $S_4$ for some $\alpha \in G$. Then for every vertex $i \in A$, we have $\phi(\alpha)(i) = i$, so that $\alpha(i) = i$. Since a rigid motion that fixes all vertices must be the identity rigid motion, then $\alpha$ is the identity element of $G$. Therefore, $\phi$ is injective and $G$ is isomorphic to a subgroup of $S_4$.
\end{sol}

\begin{exercise}
    Show that the group of rigid motions of a cube is isomorphic to $S_4$. (This group acts on the set of four pairs of opposite vertices.)
\end{exercise}

\begin{sol}
    Recall that $|G| = 24$ from \hyperref[ex1.2.10]{Exercise 1.2.10}. Let $G$ be the group of rigid motions of a cube, and let $A = \set{a_1, a_2, a_3, a_4}$ be the set of pairs of opposite vertices of the cube. Then each $\alpha \in G$ sends some pair of opposite vertices to another pair of opposite vertices so that $G$ acts on $A$.

    Using similar reasoning as in the exercise, we have an injective homomorphism
    \[\phi : G \to S_4 \quad \text{where} \quad \phi(\alpha)(a_i) = \alpha(a_i) \quad \text{for all} \quad a_i \in A.\]
    Since $|G| = 24 = |S_4|$, it follows that $\phi$ is surjective as well, hence $G \cong S_4$.
\end{sol}

\begin{exercise}
    Show that the group of rigid motions of an octahedron is isomorphic to a subgroup of $S_4$.  
    (This group acts on the set of four pairs of opposite faces.) Deduce that the groups of rigid motions of a cube and of an octahedron are isomorphic. (These two groups are isomorphic because these solids are “dual.”)
\end{exercise}

\begin{sol}
    Let $G$ be the group of rigid motions of an octahedron. Let $A = \set{f_1, f_2, f_3, f_4}$ be the set of pairs of opposite faces of the octahedron. Then each $\alpha \in G$ sends some pair of opposite faces to another pair of opposite faces so that $G$ acts on $A$.

    Using similar reasoning as in the previous exercises, we have an injective homomorphism
    \[\phi : G \to S_4 \quad \text{where} \phi(\alpha)(f_i) = \alpha(f_i) \quad \text{for all} \quad f_i \in A.\]
    Therefore, $G$ is isomorphic to a subgroup of $S_4$. From \hyperref[ex1.2.11]{Exercise 1.2.11}, we know that $|G| = 24$, hence $G \cong S_4$. From the previous exercise, we know that the group of rigid motions of a cube is also isomorphic to $S_4$, so the groups of rigid motions of a cube and of an octahedron are isomorphic.
\end{sol}

\begin{exercise}
    Explain why the action of the group of rigid motions of a cube on the set of three pairs of opposite faces is not faithful. Find the kernel of this action.
\end{exercise}

\begin{sol}
    The group of rigid motions of a cube has 24 elements, while permutations of the set of three pairs of opposing faces has 6 elements, so the action cannot be faithful as no homomorphism can be injective between finite groups of different sizes (if there was some injective homomorphism, it would imply bijectivity between the two groups of different cardinality, which is impossible).
    
    Consider a cube such that its center is at the origin of $\r^3$ space. Then any rotation around the axes fixes a pair of faces, while sends the other two pairs back to themselves (to visualize this explicitly, consider the cube with vertices $(\pm 1, \pm 1, \pm 1)$ and the $z$-axis. A rotation around the $z$-axis would fix the faces $(\pm 1, \pm 1, 1)$ and $(\pm 1, \pm 1, -1)$, while it rotates the pair of faces $(1, \pm 1, \pm 1)$ and $(-1, \pm 1, \pm 1)$ back to each other---these are the faces that vary along the $x$-axis. One can construct the same for the pair of faces that lie along the $y$-axis and deduce the same thing). There are exactly 3 of these $180^\circ$ rotations, so the kernel of this action consists of these rotations and the identity.
\end{sol}