\setcounter{section}{-1}

\section{Preliminaries}

\subsection{Basics}

For Exercises 1 to 4, let
$$M = 
\begin{pmatrix}
    1 & 1 \\
    0 & 1
\end{pmatrix}$$
and $\bb = \{X \in \ac \mid MX = XM\}$, where $\ac$ denotes the set of $2 \times 2$ matrices with real entries.

\begin{exercise}
    Determine which of the following elements of $\ac$ lie in $\bb$:
    $$
    \begin{pmatrix}
        1 & 1 \\
        0 & 1
    \end{pmatrix},
    \begin{pmatrix}
        1 & 1 \\
        1 & 1
    \end{pmatrix},
    \begin{pmatrix}
        0 & 0 \\
        0 & 0
    \end{pmatrix},
    \begin{pmatrix}
        1 & 1 \\
        1 & 0
    \end{pmatrix},
    \begin{pmatrix}
        1 & 0 \\
        0 & 1
    \end{pmatrix},
    \begin{pmatrix}
        0 & 1 \\
        1 & 0 
    \end{pmatrix}$$
\end{exercise}

\begin{sol}
    Note that the 1st matrix is $M$ itself so that it belongs to $\bb$, since $MX = XM = M^2$. Further note that the 3rd, 5th, and 6th matrices are the zero, identity, and exchange matrices respectively. Then the 3rd and 5th belong to $\bb$ while the 6th does not. Then only the remaining matrices to check are the 2nd and 4th matrices. For the 2nd matrix:
    \begin{align*}
        MX & = 
        \begin{pmatrix}
            1 & 1 \\
            0 & 1
        \end{pmatrix}
        \begin{pmatrix}
            1 & 1 \\
            1 & 1
        \end{pmatrix} = 
        \begin{pmatrix}
            1 \cdot 1 + 1 \cdot 1 & 1 \cdot 1 + 1 \cdot 1 \\
            0 \cdot 1 + 1 \cdot 1 & 0 \cdot 1 + 1 \cdot 1
        \end{pmatrix} = 
        \begin{pmatrix}
            2 & 2 \\ 1 & 1
        \end{pmatrix} \\
        XM & = 
        \begin{pmatrix}
            1 & 1 \\
            1 & 1 
        \end{pmatrix}
        \begin{pmatrix}
            1 & 1 \\
            0 & 1
        \end{pmatrix} = 
        \begin{pmatrix}
            1 \cdot 1 + 1 \cdot 0 & 1 \cdot 1 + 1 \cdot 1 \\
            1 \cdot 1 + 1 \cdot 0 & 1 \cdot 1 + 1 \cdot 1 
        \end{pmatrix} =
        \begin{pmatrix}
            2 & 1 \\
            2 & 1
        \end{pmatrix}
    \end{align*}
    Then $MX \neq XM$ and the 2nd matrix does not belong to $\bb$. For the 4th matrix:
    \begin{align*}
        MX & = 
        \begin{pmatrix}
            1 & 1 \\
            0 & 1 
        \end{pmatrix}
        \begin{pmatrix}
            1 & 1 \\
            1 & 0
        \end{pmatrix} =
        \begin{pmatrix}
            1 \cdot 1 + 1 \cdot 1 & 1 \cdot 1 + 1 \cdot 0 \\
            0 \cdot 1 + 1 \cdot 1 & 0 \cdot 1 + 1 \cdot 0
        \end{pmatrix} =
        \begin{pmatrix}
            2 & 1 \\
            1 & 0
        \end{pmatrix} \\
        XM & = 
        \begin{pmatrix}
            1 & 1 \\
            1 & 0
        \end{pmatrix}
        \begin{pmatrix}
            1 & 1 \\
            0 & 1
        \end{pmatrix} = 
        \begin{pmatrix}
            1 \cdot 1 + 1 \cdot 0 & 1 \cdot 1 + 1 \cdot 1 \\
            1 \cdot 1 + 0 \cdot 0 & 1 \cdot 1 + 0 \cdot 1
        \end{pmatrix} = 
        \begin{pmatrix}
            1 & 2 \\
            1 & 1
        \end{pmatrix}
    \end{align*}
    So that $MX \neq XM$.
\end{sol}

\begin{exercise}
    Prove that if $P, Q \in \bb$, then $P + Q \in \bb$ where $+$ denotes the usual sum of two matrices.
\end{exercise}

\begin{sol}
    We calculate the following:
    \begin{align*}
        M(P + Q) & = 
        \begin{pmatrix}
            1 & 1 \\
            0 & 1
        \end{pmatrix}
        \begin{pmatrix}
            a + e & b + f \\
            c + g & d + h
        \end{pmatrix} \\
        & = 
        \begin{pmatrix}
            a + e + c + g & b + f + d + h \\
            c + g & d + h
        \end{pmatrix} \\
        & = 
        \begin{pmatrix}
            a + c & b + d \\
            c & d
        \end{pmatrix} + 
        \begin{pmatrix}
            e + g & f + h \\
            g & h
        \end{pmatrix} \\
        & = MP + MQ \\
        & = PM + QM \\
        & = 
        \begin{pmatrix}
            a & a + b \\
            c & c + d
        \end{pmatrix} + 
        \begin{pmatrix}
            e & e + f \\
            g & g + h
        \end{pmatrix} \\
        & = 
        \begin{pmatrix}
            a + e & a + e + b + f \\
            c + g & c + g + d + h
        \end{pmatrix} \\
        & = 
        \begin{pmatrix}
            a + e & b + f \\
            c + g & d + h
        \end{pmatrix}
        \begin{pmatrix}
            1 & 1 \\
            0 & 1
        \end{pmatrix} \\
        & = (P + Q)M \qh
    \end{align*}
\end{sol}

\begin{exercise}
    Prove that if $P, Q \in \bb$, then $P\cdot Q \in \bb$ where $\cdot$ denotes the usual product of two matrices.
\end{exercise}

\begin{sol}
    Using $PQ$, proceed as we did above with $M(PQ)$. Rewriting the entries will result in $(MP)Q$. Since $P \in \bb$, then we have $(PM)Q$. We rewrite entries again to result in $P(MQ)$. Because $Q \in \bb$, we have $P(QM)$, and a final rewrite results in $(PQ)M$.
\end{sol}

\begin{exercise}
    Find conditions on $p, q, r, s$ which determine precisely when $
    \begin{pmatrix}
        p & q \\
        r & s
    \end{pmatrix} \in \bb$.

\end{exercise}

\begin{sol}
    Let $X$ be the matrix described above. Note that
    \begin{align*}
        MX & =
        \begin{pmatrix}
            1 & 1 \\
            0 & 1
        \end{pmatrix}
        \begin{pmatrix}
            p & q \\
            r & s
        \end{pmatrix} =
        \begin{pmatrix}
            p + r & q + s \\
            r & s
        \end{pmatrix} \\
        XM & = 
        \begin{pmatrix}
            p & q \\
            r & s
        \end{pmatrix}
        \begin{pmatrix}
            1 & 1 \\
            0 & 1
        \end{pmatrix} = 
        \begin{pmatrix}
            p & p + q \\
            r & r + s
        \end{pmatrix}
    \end{align*}
    Because $X \in \bb$, then we may compare entries to obtain the following:
    \[
    \begin{cases}
        p + r = p \\
        q + s = p + q \\
        r = r \\
        s = r + s
    \end{cases}
    \]
    The first and fourth equations force $r = 0$, and the second equation forces $p = s$. Then $\bb$ is classified as
    \[\bb = \set*{\left.
    \begin{pmatrix}
        p & p + q \\
        0 & p
    \end{pmatrix}\,\right|\, p, q \in \r} \qh\]
\end{sol}

\begin{exercise}
    Determine whether the following functions $f$ are well defined:
    \begin{subproblems}
        \item $f : \q \to \z$ defined by $f(a/b) = a$.
        \item $f : \q \to \q$ defined by $f(a/b) = a^2/b^2$.
    \end{subproblems}
\end{exercise}

\begin{solalph}
    \item No, because $1/2 = 2/4$, but $f(1/2) = 1$ and $f(2/4) = 2$.
    \item Yes; suppose $a/b = c/d$. Then $ad = bc$, or that $a^2d^2 = b^2c^2$. Then $a^2/b^2 = c^2/d^2$, or $f(a/b) = f(c/d)$. 
\end{solalph}

\begin{exercise}
    Determine whether the function $f: \r^+ \to \z$ defined by mapping a real number $r$ to the first digit to the right of the decimal point in a decimal expansion of $r$ is well defined.
\end{exercise}

\begin{sol}
    No. Note that $1=1.000\ldots = 0.999\ldots$, but $f(1.000\ldots) = 1$ and $f(0.999\ldots) = 9$.
\end{sol}

\newpage

\begin{exercise}
    Let $f: A \to B$ be a surjective map of sets. Prove that the relation
    $$a \sim b \iff f(a) = f(b)$$
    is an equivalence relation whose equivalence classes are the fibers of $f$.
\end{exercise}

\begin{sol}
    Since $=$ is an equivalence relation on $B$, then $\sim$ is an equivalence relation.

    Consider an equivalence class of some $a \in A$, which is the set $\set{x \in A \mid x \sim a}$. By definition of $\sim$, this is the set $\set{x \in A \mid f(x) = f(a)}$, which is precisely the fiber of $f$ over $f(a)$. Since $f$ is surjective, every fiber of $f$ is nonempty, and every fiber corresponds to some equivalence class of $\sim$.
\end{sol}

\newpage

\subsection{Properties of the Integers}

\begin{exercise} \label{ex0.2.1}
    For each of the following pairs of integers $a$ and $b$, determine their greatest common divisor, their least common multiple, and write their greatest common divisor in the form $ax + by$ for some integers $x$ and $y$.
    \begin{subproblems}
        \item $a = 20, b = 13$
        \item $a= 69, b = 372$
        \item $a = 792, b = 275$
        \item $a = 11391, b = 5673$
        \item $a= 1761, b = 1567$
        \item $a = 507885, b = 60808$
    \end{subproblems}
\end{exercise}

\begin{sol}
    For this exercise, we will only do (e) as that has the most steps in calculating both the $\gcd$ and the Euclidean Algorithm. The $\lcm$ is obtained by dividing the product $ab$ by $(a, b)$.
    \begin{subproblems}
        \item$(20, 13) = 1, \lcm(20, 13) = 260, 1 = 2(20) - 3(13)$
        \item $(69, 372) = 3,\lcm(69, 372) = 8556, 27(69) -5(372)$
        \item $(792, 275) = 11, \lcm(792, 275) = 19800, 8(792) - 23(275)$
        \item $(11391, 5673) = 3 ,\lcm(11391, 5673) = 21540381, 3 = 253(5673) - 126(11391)$
        \item Applying the Euclidean Algorithm to $a = 1761$ and $b = 1567$, we get:
        \begin{align*}
            1761 & = (1)1567 + 194 \\
            1567 & = (8)194 + 15 \\
            194 & = (12)15 + 14 \\
            15 & = (1)14 + 1
        \end{align*}
        Then $(1761, 1567) = 1$ so that $\lcm(1761, 1567) = 2759487$. Reversing the Euclidean Algorithm steps to solve for 1, we get:
        \begin{align*}
            1 & = 15 - 14 \\
            & = 15 - (194 - 12(15)) = 13(15) - 194 \\
            & = 13(1567 - 8(194)) - 194 = 13(1567) - 105(194) \\
            & = 13(1567) - 105(1761 - 1567) = -105(1761) + 118(1567)
        \end{align*}
        \item $(507885, 60808) = 691, \lcm(507885, 60808) = 44693880, 691 = 142(60808) - 17(507885)$ \qh
    \end{subproblems}
\end{sol}

\begin{exercise}
    Prove that if the integer $k$ divides the integers $a$ and $b$ then $k$ divides $as + bt$ for every pair of integers $s$ and $t$. 
\end{exercise}

\begin{sol}
    Since $k \mid a$ and $k \mid b$, there exists $x, y \in \z$ such that $a = kx$ and $b = ky$. Then for any $s, t \in \z$, we have $as + bt = kxs + kyt = k(xs + yt)$ which is divisible by $k$.
\end{sol}

\begin{exercise}
    Prove that if $n$ is composite then there are integers $a$ and $b$ such that $n$ divides $ab$ but $n$ does not divide either $a$ or $b$.
\end{exercise}

\begin{sol}
    By the Fundamental Theorem of Arithmetic, $n$ has at least two prime factors $a, b$ such that $1 < a, b < n$. Putting $n = ab$, then $n \mid ab$ but $n \nmid a$ and $n \nmid b$.
\end{sol}

\begin{exercise}
    Let $a, b$ and $N$ be fixed integers with $a$ and $b$ nonzero and let $d = (a, b)$ be the greatest common divisor of $a$ and $b$. Suppose $x_0$ and $y_0$ are particular solutions to $ax + by = N$ (i.e., $ax_0 + by_0 = N$). Prove for any integer $t$ that the integers
    $$x = x_0 + \frac bdt \text{ and } y = y_0 - \frac adt$$
    are also solutions to $ax + by = N$ (this is in fact the general solution).
\end{exercise}

\begin{sol}
    We have
    \begin{align*}
        ax + by & = a\left(x_0 + \frac bdt\right) + b\left(y_0 - \frac adt\right) \\
        & = ax_0 + \frac{ab}{d}t + by_0 - \frac{ab}{d}t \\
        & = ax_0 + by_0 = N \qh
    \end{align*}
\end{sol}

\begin{exercise}
    Determine the value $\phi(n)$  each integer $n \leq 30$ where $\phi$ denotes the Euler $\phi$-function.
\end{exercise}

\begin{sol}
    $$\begin{array}{c|c|c|c|c|c|c|c|c|c|c|c|c|c|c|c}
        n & 1 & 2 & 3 & 4 & 5 & 6 & 7 & 8 & 9 & 10 & 11 & 12 & 13 & 14 & 15 \\
        \hline
        \phi(n) & 1 & 1 & 2 & 2 & 4 & 2 & 6 & 4 & 6 & 4 & 10 & 4 & 12 & 6 & 8 \\
    \end{array}$$
    $$\begin{array}{c|c|c|c|c|c|c|c|c|c|c|c|c|c|c|c}
        n & 16 & 17 & 18 & 19 & 20 & 21 & 22 & 23 & 24 & 25 & 26 & 27 & 28 & 29 & 30 \\
        \hline
        \phi(n) & 8 & 16 & 6 & 18 & 8 & 12 & 10 & 22 & 8 & 20 & 12 & 18 & 12 & 28 & 8 \\
    \end{array}$$
\end{sol}

\begin{exercise}
    Prove the Well Ordering Property of $\z$ by induction and prove the minimal element is unique.
\end{exercise}

\begin{sol}
    Let $S \subseteq \zp$ be nonempty. We proceed by induction that $S$ has a minimal element. Assume, by way of contradiction, that $S$ has no minimal element, and let $S'$ be the set of elements that are not in $S$. Since the minimal element of $\zp$ is 0, then $0 \not\in S$ so that $0 \in S'$. Now for some $k \in \zp$, suppose every integer $j$ such that $0 \leq j \leq k$ is in $S'$. Then $k + 1 \not\in S$, since if it were, then it would be the minimal element of $S$ (as every integer less than or equal to $k$ is in $S'$). Thus, $k + 1 \in S'$. By induction, every integer in $\zp$ is in $S'$, contradicting that $S$ is nonempty. Thus, $S$ has a minimal element.

    To prove uniqueness of the minimal element, suppose $S$ has two minimal elements $x$ and $y$. Then by definition of minimality, $x \leq y$ and $y \leq x$, so that $x = y$.
\end{sol}

\begin{exercise}
    If $p$ is a prime, prove that there do not exist nonzero integers $a$ and $b$ such that $a^2 = pb^2$ (i.e., $\sqrt p$ is not a rational number).
\end{exercise}

\begin{sol}
    Assume, by way of contradiction, that there exist nonzero integers $a$ and $b$ such that $a^2 = pb^2$. Without loss of generality, we may assume that $a$ and $b$ have no common factors (otherwise, we could divide them both by their greatest common divisor). Then $p \mid a^2$, so that $p \mid a$. Then there exists some integer $k$ such that $a = pk$. We then have
    \[a^2 = (pk)^2 = p^2k^2 = pb^2\]
    so that $b^2 = pk^2$. Then $p \mid b^2$, so that $p \mid b$. However, this contradicts our assumption that $a$ and $b$ have no common factors. Thus, there do not exist nonzero integers $a$ and $b$ such that $a^2 = pb^2$.
\end{sol}

\begin{exercise} \label{ex0.2.8}
    Let $p$ be a prime, $n \in \zp$. Find a formula for the largest power of $p$ which divides $n! = n(n - 1)(n - 2) \ldots 2 \cdot 1$ (it involves the greatest integer function).
\end{exercise}

\begin{sol}
    Note that there exists some $k \in \zp$ such that $n \geq kp$. Then the multiples of $p$ less than or equal to $n$ are $p, 2p, 3p, \ldots, kp$, contributing $k$ factors of $p$. Further, the multiples of $p^2$ less than or equal to $n$ are $p^2, 2p^2, 3p^2, \ldots, lp^2$ where $l$ is the largest integer such that $lp^2 \leq n$. These contribute an additional $l$ factors of $p$. Continuing this process until we reach $p^m$ where $m$ is the largest integer such that $p^m \leq n$, we find that $m = \log_p n$. Then the largest power of $p$ which divides $n!$ is given by
    \[\floor*{\frac{n}{p}} + \floor*{\frac{n}{p^2}} + \floor*{\frac{n}{p^3}} + \cdots + \floor*{\frac{n}{p^m}} = \sum_{i=1}^m \floor*{\frac{n}{p^i}} \qh\]
\end{sol}

\begin{exercise} \label{ex0.2.9}
    Write a computer program to determine the greatest common divisor $(a, b)$ of two integers $a$ and $b$ and to express $(a, b)$ in the form $ax + by$ for some integers $x$ and $y$.
\end{exercise}

\begin{breakablealgorithm}
    \begin{algorithmic}[1]
        \Require{Integers $a, b$ with $a \geq b > 0$}
        \Ensure{Integers $g, x, y$ such that $g = (a, b)$ and $ax + by = g$}
        \Procedure{GCD}{$a, b$}
            \State {$A, B \gets a, b$}
            \State {$x_0, y_0 \gets 1, 0$} \Comment{$a = 1 \cdot A + 0 \cdot B$}
            \State {$x_1, y_1 \gets 0, 1$} \Comment{$b = 0 \cdot A + 1 \cdot B$}
            \While {$a \neq b$}:
                \If {$a > b$}
                    \State {$a \gets a - b$}
                    \State {$x_0, y_0 \gets x_0 - x_1, y_0 - y_1$} \Comment{Update coefficients for $a$}
                \Else
                    \State {$b \gets b - a$}
                    \State {$x_1, y_1 \gets x_1 - x_0, y_1 - y_0$} \Comment{Update coefficients for $b$}
                \EndIf
            \EndWhile
            \State {$g \gets a$}
            \State \Return {$(g, x_0, y_0)$}
        \EndProcedure
    \end{algorithmic}
\end{breakablealgorithm}

\np

\begin{exercise}
    Prove for any given positive integer $N$ there exist only finitely many integers $n$ with $\phi(n) = N$ where $\phi$ denotes Euler's $\phi$-function. Conclude in particular that $\phi$ tends to infinity as $n$ tends to infinity.
\end{exercise}

\begin{sol}
    Fix $N \in \zp$. Consider $n \in \zp$ such that $\phi(n) = N$. By the Fundamental Theorem of Arithmetic, we may express $n$ as a product of primes $p_1, p_2, \ldots, p_k$ such that $p_1 < p_2 < \cdots < p_k$. Moreover, we have exponents $\alpha_1, \alpha_2, \ldots, \alpha_k$ for each prime. Using the identity that $\phi(p^\alpha) = p^{\alpha - 1}(p - 1)$ for any prime $p$ and integer $\alpha \geq 1$, then
    \[\phi(n) = \prod_{i = 1}^k \phi(p_i^{\alpha_i}) = \prod_{i = 1}^k p_i^{\alpha_i - 1}(p_i - 1)\]
    Since $\phi(n) = N$, then $N$ may be expressed as a product of the terms $p_i^{\alpha_i - 1}(p_i - 1)$ for $1 \leq i \leq k$. Note that each term is at least 1, so that there are only finitely many ways to express $N$ as a product of such terms. Further, for each term, there are only finitely many choices of $p_i$ and $\alpha_i$ that yield that term. Thus, there are only finitely many integers $n$ such that $\phi(n) = N$.

    To see that $\phi$ tends to infinity as $n$ tends to infinity, assume, by way of contradiction, that $\phi$ does not tend to infinity as $n$ tends to infinity. Then there exists $M \in \zp$ such that $\phi(n) \leq M$ for infinitely many $n$. However, there are only finitely many $n$ such that $\phi(n) \leq K$ for all $K \leq M$ by the first part of this exercise, contradicting our assumption. Thus, $\phi$ tends to infinity as $n$ tends to infinity.
\end{sol}

\begin{exercise}
    Prove that if $d$ divides $n$ then $\phi(d)$ divides $\phi(n)$ where $\phi$ denotes Euler's $\phi$-function.
\end{exercise}

\begin{sol}
    By the Fundamental Theorem of Arithmetic, we may express $n$ as a product of primes $p_1, p_2, \ldots, p_k$ such that $p_1 < p_2 < \cdots < p_k$. Moreover, we have exponents $\alpha_1, \alpha_2, \ldots, \alpha_k$ for each prime. Since $d$ is a divisor of $n$, we may express $d$ as $p_1^{\beta_1} p_2^{\beta_2} \ldots p_k^{\beta_k}$ where $0 \leq \beta_i \leq \alpha_i$ for all $1 \leq i \leq k$. Using the identity that $\phi(p^\alpha) = p^{\alpha - 1}(p - 1)$ for any prime $p$ and integer $\alpha \geq 1$, then 
    \[\phi(n) = \prod_{i = 1}^k \phi(p_i^{\alpha_i}) = \prod_{i = 1}^k p_i^{\alpha_i - 1}(p_i - 1) \longand \phi(d) = \prod_{i = 1}^k \phi(p_i^{\beta_i}) = \prod_{i = 1}^k p_i^{\beta_i - 1}(p_i - 1)\]
    It is now clear that $\phi(d) \mid \phi(n)$, since $p_i - 1$ divides itself and $p_i^{\beta_i - 1}$ divides $p_i^{\alpha_i - 1}$ for all $1 \leq i \leq k$.
\end{sol}

\np

\subsection{\texorpdfstring{$\intmod$: The Integers Modulo $n$}{Z/nZ: The Integers Modulo n}}

\begin{exercise}
    Write down explicitly all the elements in the residue classes of $\intmod[18]$.
\end{exercise}

\begin{sol}
    \begin{align*}
        \bar 0 & = \{0 + 18k : k \in \z\} = \{0, 18, -18, 36, -36, \ldots\} \\
        \bar 1 & = \{1 + 18k : k \in \z\} = \{1, 19, -17, 37, -35, \ldots\} \\
        \bar 2 & = \{2 + 18k : k \in \z\} = \{2, 20, -16, 38, -34, \ldots\} \\
        \bar 3 & = \{3 + 18k : k \in \z\} = \{3, 21, -15, 39, -33, \ldots\} \\
        \bar 4 & = \{4 + 18k : k \in \z\} = \{4, 22, -14, 40, -32, \ldots\} \\
        \bar 5 & = \{5 + 18k : k \in \z\} = \{5, 23, -13, 41, -31, \ldots\} \\
        \bar 6 & = \{6 + 18k : k \in \z\} = \{6, 24, -12, 42, -30, \ldots\} \\
        \bar 7 & = \{7 + 18k : k \in \z\} = \{7, 25, -11, 43, -29, \ldots\} \\
        \bar 8 & = \{8 + 18k : k \in \z\} = \{8, 26, -10, 44, -28, \ldots\} \\
        \bar 9 & = \{9 + 18k : k \in \z\} =\{9, 27, -9, 45, -27, \ldots\} \\
        \bar{10} & = \{10 + 18k : k \in \z\} = \{10, 28, -8, 46, -26, \ldots\} \\
        \bar{11} & = \{11 + 18k : k \in \z\} = \{11, 29, -7, 47, -25, \ldots\} \\
        \bar{12} & = \{12 + 18k : k \in \z\} = \{12, 30, -6, 48, -24, \ldots\} \\
        \bar{13} & = \{13 + 18k : k \in \z\} = \{13, 31, -5, 49, -23, \ldots\} \\
        \bar{14} & = \{14 + 18k : k \in \z\} = \{14, 32, -4, 50, -22, \ldots\} \\
        \bar{15} & = \{15 + 18k : k \in \z\} = \{15, 33, -3, 51, -21, \ldots\} \\
        \bar{16} & = \{16 + 18k : k \in \z\} = \{16, 34, -2, 52, -20, \ldots\} \\
        \bar{17} & = \{17 + 18k : k \in \z\} = 
        \{17, 35, -1, 53, -19, \ldots\} \qh
    \end{align*}
\end{sol}

\begin{exercise}
    Prove that the distinct equivalence classes in $\intmod$ are precisely $\bar 0, \bar 1, \bar 2, \ldots, \bar{n - 1}$ (use the Division Algorithm).
\end{exercise}

\begin{sol}
    Let $a \in \z$. By the Division Algorithm, there exists unique integers $q$ and $r$ such that $a = nq + r$ where $0 \leq r < n$. Then $a \equiv r \bmod n$, so that $a \in \bar r$. Since $0 \leq r < n$, then $\bar r$ is one of $\bar 0, \bar 1, \ldots, \bar{n - 1}$. Thus, every equivalence class in $\intmod$ is one of $\bar 0, \bar 1, \ldots, \bar{n - 1}$. Moreover, these equivalence classes are distinct since if $\bar i = \bar j$ for some $0 \leq i, j < n$, then $i \equiv j \bmod n$, so that $n \mid (i - j)$. However, since $-n < i - j < n$, then $i - j = 0$, or $i = j$. Thus, the distinct equivalence classes in $\intmod$ are precisely $\bar 0, \bar 1, \ldots, \bar{n - 1}$.
\end{sol}

\begin{exercise}
    Prove that if $a = a_n10^n + a_{n - 1}10^{n - 1} + \cdots + a_110 + a_0$ is any positive integer then $a \equiv (a_n + a_{n - 1} + \cdots + a_1 + a_0) \bmod 9$ (note that this is the usual arithmetic rule that the remainder after division by 9 is the same as the sum of the decimal digits mod 9—in particular an integer is divisible by 9 if and only if the sum of its digits is divisible by 9) [note that $10 \equiv 1 \bmod 9$].
\end{exercise}

\begin{sol}
    Using the note, then
    \begin{align*}
        a & \equiv (a_n10^n + a_{n - 1}10^{n - 1} + \cdots + a_110 + a_0) \bmod 9 \\
        & \equiv (a_n1^n + a_{n - 1}1^{n - 1} + \cdots + a_11 + a_0) \bmod 9 \\
        & = (a_n + a_{n - 1} + \cdots + a_1 + a_0) \bmod 9 \qh
    \end{align*}
\end{sol}

\begin{exercise}
    Compute the remainder when $37^{100}$ is divided by $29$.
\end{exercise}

\begin{sol}
    Note the following:
    \begin{align*}
        37^2 & \equiv 6 \bmod 29 \\ 
        37^4 & \equiv 6^2 \bmod 29 \equiv 7 \bmod 29 \\
        37^8 & \equiv 7^2 \bmod 29 \equiv 20 \bmod 29 \\
        37^{16} & \equiv 20^2 \bmod 29 \equiv 23 \bmod 29 \equiv -6 \bmod 29 \\
        37^{32} & \equiv (-6)^2 \bmod 29 \equiv 7 \bmod 29 \\
        37^{64} & \equiv 20 \bmod 29
    \end{align*}
    Then we have
    $$37^{64}37^{32}37^4 \equiv 20 \cdot 7 \cdot 7 \bmod 29 \equiv 20^2 \bmod 29 \equiv 23 \bmod 29$$
    Hence the remainder when dividing $37^{100}$ by $29$ is $23$.
\end{sol}

\begin{exercise}
    Compute the last two digits of $9^{1500}$.
\end{exercise}

\begin{sol}
    Recall that we take a number mod 100 to find the last two digits. Note that $9^4 \equiv 61 \bmod 100$, and $9^3 \equiv 29 \bmod 100$. Then 
    \[9^6 = (9^3)^2 \bmod 100 \equiv 29^2 \bmod 100 \equiv 41 \bmod 100.\]
    Further, 
    \[9^{10} = 9^6 \cdot 9^4 \bmod 100 \equiv 41 \cdot 61 \bmod 100 \equiv 1 \bmod 100.\] 
    Then 
    \[9^{1500} = (9^{10})^{150} \bmod 100 \equiv 1^{150} \bmod 100 \equiv 1 \bmod 100.\]
    Thus, the last two digits of $9^{1500}$ are $01$.
\end{sol}

\begin{exercise} \label{ex0.3.6}
    Prove that the square of the elements in $\intmod[4]$ are just $\bar 0$ and $\bar 1$.
\end{exercise}

\begin{sol}
    \begin{align*}
        0^2 & = 0 \equiv 0 \bmod 4 \\
        1^2 & = 1 \equiv 1 \bmod 4 \\
        2^2 & = 4 \equiv 0 \bmod 4 \\
        3^2 & = 9 \equiv 1 \bmod 4 \qh
    \end{align*}
\end{sol}

\begin{exercise} \label{ex0.3.7}
    Prove that for any integers $a$ and $b$ that $a^2 + b^2$ never leaves a remainder of $3$ when divided by $4$ (use the previous exercise).
\end{exercise}

\begin{sol}
    Since $a^2 $ and $b^2$ is either $0 \bmod 4$ or $1 \bmod 4$, then we have $4$ potential sums:
    \begin{align*}
        0 + 0 \equiv 0 \bmod 4 \\
        0 + 1 \equiv 1 \bmod 4 \\
        1 + 0 \equiv 1 \bmod 4 \\
        1 + 1 \equiv 2 \bmod 4
    \end{align*}
    In any of the sums, there is no remainder of $3$ when dividing by $4$.
\end{sol}

\begin{exercise}
    Prove that the equation $a^2 + b^2 = 3c^2$ has no solutions in nonzero integers $a, b$ and $c$. [Consider the equation mod 4 as in the previous two exercises and show that $a, b$ and $c$ would all have to be divisible by $2$. Then each of $a^2, b^2$ and $c^2$ has a factor of $4$ and by dividing through by $4$ show that there would be a smaller set of solutions to the original equation. Iterate to reach a contradiction.]
\end{exercise}

\begin{sol}
    We consider the equation in mod 4. By \hyperref[ex0.3.7]{Exercise 0.3.7}, the left side of the equation can only be $0, 1$, or $2 \bmod 4$. However, the right side is $3c^2$, which is either $0 \bmod 4$ or $3 \bmod 4$ since $c^2$ is either $0 \bmod 4$ or $1 \bmod 4$ by \hyperref[ex0.3.6]{Exercise 0.3.6}. Thus, both sides must be $0 \bmod 4$, so that $4 \mid a^2 + b^2$ and $4 \mid 3c^2$.

    It is easy to see that if $4 \mid c^2$, then $2 \mid c$. We now consider the fact that $4 \mid a^2 + b^2$. If both $a$ and $b$ are odd, then $a^2 + b^2 \equiv 2 \bmod 4$, which is false. If one of $a$ or $b$ is odd and the other is even, then $a^2 + b^2 \equiv 1 \bmod 4$, which is false. Thus, both $a$ and $b$ are even, so that $2 \mid a$ and $2 \mid b$. Then there exist integers $a_0, b_0, c_0$ such that $a = 2a_0, b = 2b_0,$ and $c = 2c_0$. Substituting these into the original equation, we have
    \[4a_0^2 + 4b_0^2 = 12c_0^2 \implies a_0^2 + b_0^2 = 3c_0^2.\]
    However, this contradicts our assumption that $a, b,$ and $c$ are nonzero integers, since we have found a smaller set of integers $a_0, b_0,$ and $c_0$ that satisfy the same equation. Iterating this process leads to a contradiction, so there are no nonzero integers $a, b,$ and $c$ such that $a^2 + b^2 = 3c^2$.
\end{sol}

\begin{exercise}
    Prove that the square of any odd integer always leaves a remainder of $1$ when divided by $8$.
\end{exercise}

\begin{sol}
    Let $a$ be an odd integer. Then there exists some integer $k$ such that $a = 2k + 1$. Then
    \[a^2 = (2k + 1)^2 = 4k^2 + 4k + 1 = 4k(k + 1) + 1.\]
    Since one of $k$ or $k + 1$ is even, then $2 \mid k(k + 1)$, hence $8 \mid 4k(k + 1)$. Thus, $a^2 \equiv 1 \bmod 8$.
\end{sol}

\begin{specialexercise}
    Prove that the number of elements of $\units$ is $\phi(n)$ where $\phi$ denotes the Euler $\phi$-function.
\end{specialexercise}

\begin{sol}
    Recall that $\phi(n)$ is defined as the number of integers $a$ such that $1 \leq a \leq n$, and $(a, n) = 1$. Hence, to prove that the number of elements in $\units$ is $\phi(n)$, it suffices to prove that $\bar a \in \units$ if and only if $(a, n) = 1$.
    \begin{itemize}
        \item \noindent\rightimp Suppose $\bar a \in \units$. Then there exists some $\bar b \in \units$ such that $\bar a \cdot \bar b = \bar 1$, or $ab \equiv 1 \bmod n$. Then there exists some integer $k$ such that $ab - 1 = kn$, or $ab - kn = 1$. Thus, $(a, n) = 1$.
        \item \noindent\leftimp Suppose $(a, n) = 1$. Then there exist integers $x$ and $y$ such that $ax + ny = 1$. Then $ax \equiv 1 \bmod n$, so that $\bar a \cdot \bar x = \bar 1$. Thus, $\bar a \in \units$. \qh
    \end{itemize}
\end{sol}

\begin{exercise}
    Prove that if $\bar a, \bar b \in \units$, then $\bar a \cdot \bar b \in \units$.
\end{exercise}

\begin{sol}
    Since $\bar a, \bar b \in \units$, then there exist $\bar c, \bar d \in \units$ such that $\bar a \cdot \bar c = \bar 1$ and $\bar b \cdot \bar d = \bar 1$. Then
    \[(\bar a \cdot \bar b)(\bar c \cdot \bar d) = \bar a(\bar b \cdot \bar d) \cdot \bar c = \bar a \cdot \bar 1 \cdot \bar c = \bar a \cdot \bar c = \bar 1,\]
    so that $\bar a \cdot \bar b \in \units$.
\end{sol}

\newpage

\begin{exercise}
    Let $n \in \z, n > 1$ and let $a \in \z$ with $1 \leq a \leq n$. Prove if $a$ and $n$ are not relatively prime, there exists an integer $b$ with $1 \leq b < n$ such that $ab \equiv 0 \bmod n$ and deduce that there cannot be an integer $c$ such that $ac \equiv 1 \bmod n$.
\end{exercise}

\begin{sol}
    Since $(a, n) = d > 1$, then there exist integers $x$ and $y$ such that $ax + ny = d$. Let $b = n/d$. Then
    \[ab = a(n/d) = n(a/d) \equiv 0 \bmod n,\]
    so that such a $b$ exists, since $a/d, n/d \in \z$. Moreover, if there existed some integer $c$ such that $ac \equiv 1 \bmod n$, then we would have $(ac)b = (ab)c \equiv b \bmod n$. This contradicts that $ab \equiv 0 \bmod n$, so such a $c$ cannot exist.
\end{sol}

\begin{exercise}
    Let $n \in \z, n > 1$ and let $a \in \z$ with $1 \leq a \leq n$. Prove that if $a$ and $n$ are relatively prime then there is an integer $c$ such that $ac \equiv 1 \bmod n$ [use the fact that the g.c.d. of two integers is a $\z$-linear combination of the integers].
\end{exercise}

\begin{sol}
    Since $(a, n) = 1$, there exists $c, x \in \z$ such that $ac + nx = 1$. Then $ac \equiv 1 \bmod n$.
\end{sol}

\begin{exercise}
    Conclude from the previous two exercises that $\units$ is the set of elements $\bar a$ of $\intmod$ with $(a, n) = 1$ and hence prove Proposition 4. Verify this directly in the case $n = 12$.
\end{exercise}

\begin{sol}
    The previous two exercises show that $a$ and $n$ are relatively prime if and only if there exists $b$ such that $ab \equiv 1 \bmod n$, which is exactly the proposition. For $n = 12$, the elements $1, 5, 7, 11$ are relatively prime to $12$, so that $\units[11] = \{\bar 1, \bar 5, \bar 7, \bar{11}\} $ whose inverses are $\bar 1, \bar 7, \bar 5, \bar{11}$ respectively. 
\end{sol}

\begin{exercise}
    For each of the following pairs of integers $a$ and $n$, show that $a$ is relatively prime to $n$ and determine the multiplicative inverse of $\bar a$ in $\intmod$.
    \begin{subproblems}
        \item $a = 13, n = 20$
        \item $a = 69, n = 89$
        \item $a = 1891, n = 3797$
        \item $a = 6003722857, n = 77695236973$
    \end{subproblems}
\end{exercise}

\begin{sol}
    To proceed with these exercises, we will need to use the Euclidean Algorithm to find $(a, n)$, then use the series of quotients and remainders to express $1$ as a $\z$-linear combination of $a$ and $n$ in order to find the inverse. In particular, we will obtain a combination in the form of $ax + ny = 1$. Then $ax \equiv 1 \bmod n$ so that the inverse of $a$ is $x$. Refer to \hyperref[ex0.2.1]{Exercise 0.2.1} to see how we applied the Euclidean Algorithm.
    \begin{subproblems}
        \item $-3(13) + 2(20) = 1 \implies -3 \cdot 13 \equiv 1 \bmod 20$, so the inverse is $\bar{-3} = \bar{17}$.
        \item $40(69) - 31(89) = 1 \implies 40 \cdot 69 \equiv 1 \bmod 89$, so the inverse is $\bar{40}$.
        \item $253(1891) - 126(3797) = 1 \implies 253 \cdot 1891 \equiv 1 \bmod 3797$, so the inverse is $\bar{253}$.
        \item $17n - 220a = 1 \implies -220a \equiv 1 \bmod n \implies 77695237193a \equiv 1 \bmod n$, so the inverse is $\bar{77695237193}$.
    \end{subproblems}
\end{sol}

\newpage

\begin{exercise}
    Write a computer program to add and multiply mod $n$, for any $n$ given as input. The output of these operations should be the least residues of the sums and products of two integers. Also include the feature that if $(a, n) = 1$, an integer $c$ between $1$ and $n - 1$ such that $\bar a \cdot \bar c = \bar 1$ may be printed on request. (Your program should not, of course, simply quote "mod" functions already built into many systems).
\end{exercise}

\noindent \textbf{\textit{Solution.}}~~Using the function \textsc{GCD} from \hyperref[ex0.2.9]{Exercise 0.2.9}, we have the following algorithm:
\begin{breakablealgorithm}
    \begin{algorithmic}[1]
        \Require Integers $a, b, n$ with $n > 1$
        \Ensure $\bar a + \bar b$ and $\bar a \cdot \bar b$, reduced modulo $n$, and if $(a, n) = 1$, the multiplicative inverse of $\bar a$ mod $n$
        \Procedure{AddMod}{$a, b, n$}
            \State {$s \gets a + b$}
            \While {$s < 0$}:
                \State {$s \gets s + n$}
            \EndWhile
            \While {$s \geq 0$}:
                \State {$s \gets s - n$}
            \EndWhile
            \State \Return {$s$}
        \EndProcedure
        \Procedure{MultiplyMod}{$a, b, n$}
            \State {$p \gets a \cdot b$}
            \While {$p < 0$}:
                \State {$p \gets p + n$}
            \EndWhile
            \While {$p \geq 0$}:
                \State {$p \gets p - n$}
            \EndWhile
            \State \Return {$p$}
        \EndProcedure
        \Procedure{InverseMod}{$a, n$}
            \State {$(g, x, y) \gets$ \Call{\textsc{GCD}}{$a, n$}}
            \If {$g \neq 1$}
                \State \Return {``No inverse exists''}
            \Else
                \While {$x < 0$}:
                    \State {$x \gets x + n$}
                \EndWhile
                \While {$x \geq 0$}:
                    \State {$x \gets x - n$}
                \EndWhile
                \State \Return {$x$}
            \EndIf
        \EndProcedure
    \end{algorithmic}
\end{breakablealgorithm}